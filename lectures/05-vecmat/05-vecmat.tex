\documentclass[spanish, c]{beamer}

\usepackage[utf8]{inputenc}
\usepackage[spanish, mexico]{babel}
\usepackage{hyperref}
\usepackage{xcolor}
\usepackage{color}
\usepackage{ragged2e}
\usepackage{mathrsfs}
\usepackage{csquotes}
\usepackage{listings}
\usepackage[scaled]{beramono}
\usepackage[T1]{fontenc}
\usepackage{matlab-prettifier}
\usepackage{graphicx}

\renewcommand{\indent}{\hspace*{2em}}

% \usepackage{tikz}

% \usetikzlibrary{fit, shapes, arrows}

% \usepackage{courier}
% \usepackage{subfigure}
% \usepackage{enumerate}
% \usepackage{algorithmic}
% \usepackage{algorithm}

% \usepackage{listings}
% \usepackage{lstlinebgrd}

\usetheme{Boadilla}
\usefonttheme[onlymath]{serif}

\newcommand{\matlab}[1]{\lstinline[style=Matlab-editor]!#1!}
\newcommand\blfootnote[1]{%
\begingroup
\renewcommand\thefootnote{}\footnote{#1}%
\addtocounter{footnote}{-1}%
\endgroup
}

\lstset
{
    language = Matlab,
    style = Matlab-editor,
    basicstyle = \mlttfamily\scriptsize,
    escapechar = `,
    numbers = left,
    frame = tb,
}

\lstdefinestyle{output}
{
    language = {},
    basicstyle = \mlttfamily\scriptsize,
    escapechar = `,
    numbers = none,
    showtabs = false,
   	showstringspaces = false,
}

% Sets the templates
\definecolor{navyblue}{RGB}{0, 0, 128}
\definecolor{crimson}{RGB}{128, 16, 0}

\setbeamertemplate{navigation symbols}{} 
\setbeamertemplate{headline}{}
\setbeamertemplate{footline}[frame number]
\setbeamertemplate{bibliography item}[text]
\setbeamertemplate{theorems}[numbered]

\setbeamercolor{title}{fg=navyblue, bg=white}
\setbeamercolor{frametitle}{fg=navyblue, bg=white}
\setbeamercolor{structure}{fg=navyblue}
\setbeamercolor{button}{fg=white,bg=navyblue}

\setbeamercovered{transparent}

\title{Vectores y Matrices}
\subtitle{Solución de Problemas con Programación \\ (TC1017)}
\author{
    \texorpdfstring{
        \begin{center}
            M.C. Xavier Sánchez Díaz \\
            \href{mailto:mail@tec.mx}{\texttt{mail@tec.mx}}
        \end{center}
    }
    {M.C. Xavier Sánchez Díaz}
}

\institute[Tecnológico de Monterrey]{\includegraphics[scale=0.5]{../img/logo}}
\date{}

\begin{document}

\setlength{\rightskip}{0pt}

\begin{frame}[plain]
    \titlepage        
\end{frame}

\begin{frame}{Outline}
    \tableofcontents
\end{frame}

\section{Vectores}

\begin{frame}{Teoría aburrida}{Vectores}
    \begin{definition}
            Un vector $n$-dimensional es una lista de $n$ números donde:

            \begin{itemize}
                \item La adición vectorial es conmutativa.
                \item La adición vectorial es asociativa.
                \item Existe la identidad aditiva.
                \item Existe la adición inversa.
                \item La multiplicación escalar es asociativa.
                \item La adición escalar es distributiva.
                \item La adición vectorial es distributiva.
                \item Existe la identidad multiplicativa.
            \end{itemize}
    \end{definition}
\end{frame}

\begin{frame}{La adición vectorial es conmutativa}{Vectores}
    Es decir que no importa el orden al sumar dos vectores:

    $$\mathbf{x} + \mathbf{y} = \mathbf{y} + \mathbf{x}$$

    \bigskip

    \textbf{Ejemplo práctico}: Si te mueves dos cuadras hacia el sur y una hacia el oriente. llegas al mismo lugar que moviéndote una cuadra hacia el oriente y luego dos hacia el sur.
    
\end{frame}

\begin{frame}{La adición vectorial es asociativa}{Vectores}
    
    $$(\mathbf{x} + \mathbf{y}) + \mathbf{z} = \mathbf{x} + (\mathbf{y} + \mathbf{z})$$

    \textbf{Ejemplo Práctico}: Si te avientas un viaje de acá a la esquina, descansas, y luego a la farmacia y al parque de un jalón, llegas al mismo lugar que si te avientas un viaje de acá a la esquina y a la farmacia de un jalón, descansas, y luego vas al parque.

\end{frame}

\begin{frame}{Existe la identidad aditiva}{Vectores}
    
    $$\mathbf{x} + 0 = 0 + \mathbf{x} = \mathbf{x}$$

    \textbf{Ejemplo Práctico}: Si vas al parque, y te quedas ahí, llegas al mismo lugar que si vas al parque (duh).

\end{frame}

\begin{frame}{Existe la adición inversa}{Vectores}

    $$\forall \mathbf{x}, \exists -\mathbf{x} \text{ tal que } \mathbf{x} + (-\mathbf{x}) = 0$$

    \textbf{Ejemplo práctico}: Para cada ruta que existe para llegar a algún lugar, existe otra ruta (la misma, de regreso) que si la sigues, llegas al mismo punto de donde saliste.
\end{frame}

\begin{frame}{La multiplicación escalar es asociativa}{Vectores}    
    
    $$r(s \mathbf{x}) = (rs)\mathbf{x}$$

    \textbf{Ejemplo práctico}: Estás a 5 metros de mí en cierta dirección. Si te mueves el doble de distancia, y luego el triple, estarás en el mismo punto que si te mueves seis veces en esa dirección desde tu posición original.

\end{frame}

\begin{frame}{La adición escalar es distributiva}{Vectores}
    
    $$(r + s)\mathbf{x} = r\mathbf{x} + s\mathbf{x}$$

    \textbf{Ejemplo práctico}: Estás a 5 metros de mí en cierta dirección. Si te mueves 7 veces esa distancia, hacia allá, estarás a 35 metros de mí. El lugar al que llegarías, sería el mismo lugar al que llegarías si te mueves 20 metros para allá, y luego 15 metros más en la misma dirección.

\end{frame}

\begin{frame}{La adición vectorial es distributiva}{Vectores}
    
    $$r(\mathbf{x} + \mathbf{y}) = r\mathbf{x} + r\mathbf{y}$$

    \textbf{Ejemplo práctico}: Si vamos 2 cuadras al sur y dos al oriente llegamos a la farmacia, que está a 100 metros de la iglesia desde la que salimos. Si me muevo el doble de esa distancia, alejándome de la iglesia, llego a la escuela, que está en el mismo lugar al que llegaría si me moviera 4 cuadras al sur y luego cuatro cuadras al oriente.
\end{frame}

\begin{frame}{Existe la identidad multiplicativa}{Vectores}

    $$1\mathbf{x} = \mathbf{x}$$

    \textbf{Ejemplo práctico}: Si me muevo de la escuela a la iglesia, llegaría al mismo lugar que si recorriera una vez esa distancia, en esa dirección desde mi punto de salida (duh)

\end{frame}

\section{Matrices y transformaciones lineales}

\begin{frame}{Teoría aburrida}{Matrices y transformaciones lineales}    
    \begin{definition}
        Una \alert{matriz} es una manera conveniente de guardar una \textbf{transformación lineal}.
    \end{definition}

    \begin{definition}
        Una \alert{transformación lineal} es una función de la forma $T: V \to W$ de tal manera que siempre se cumple lo siguiente:

        \begin{itemize}
            \item $T(\mathbf{v_1} + \mathbf{v_2}) = T(\mathbf{v_1}) + T(\mathbf{v_2})$ \quad (Adición de vectores)
            \item $T(\alpha\mathbf{v}) = \alpha T(\mathbf{v})$ \quad (Multiplicación escalar)
        \end{itemize}

        considerando que $\mathbf{v} \in V$, o sea, $\mathbf{v}$ es un vector.
    \end{definition}

    \bigskip

    En otras palabras, una manera sencilla de guardar \textbf{qué hacer} a \textbf{cada uno de los componentes} de un vector.
\end{frame}

\begin{frame}[allowframebreaks]{Adición de vectores}{Matrices y transformaciones lineales}

    \textbf{Mi transformación}: \textit{el doble del primero y triple del segundo}
    
    \bigskip

    $$T = \begin{bmatrix}
        2 & 0 \\
        0 & 3
    \end{bmatrix} = $$
    
    $$
    \begin{bmatrix}
    \text{influencia de } x \text{ sobre } x & \text{influencia de } x \text{ sobre } y \\
    \text{influencia de } y \text{ sobre } x & \text{influencia de } y \text{ sobre } y
    \end{bmatrix}$$

    \bigskip

    \dots Algo así.

    \framebreak

    $$T(\mathbf{v_1} + \mathbf{v_2}) = T(\mathbf{v_1}) + T(\mathbf{v_2})$$

    \bigskip

    \begin{columns}
        \begin{column}{0.5\textwidth}
            \begin{align*}
                & T \left (
                \begin{bmatrix}
                    2 \\ 1
                \end{bmatrix}
                 +
                 \begin{bmatrix}
                    3 \\ 2 
                 \end{bmatrix} \right ) = \\[2ex]
                 & =
                 \begin{bmatrix}
                    2, 0 \\
                    0 , 3
                \end{bmatrix} \left(
                    \begin{bmatrix}
                        2 \\ 1
                    \end{bmatrix}
                     +
                     \begin{bmatrix}
                        3 \\ 2 
                     \end{bmatrix} \right )\\
                 & =
                 \begin{bmatrix}
                    2, 0 \\
                    0 , 3
                \end{bmatrix}
                 \begin{bmatrix}
                     5 \\ 3
                 \end{bmatrix} \\
                & =
                \begin{bmatrix}
                    10 \\ 9
                \end{bmatrix}
            \end{align*}
        \end{column}
        \begin{column}{0.5\textwidth}
            \begin{align*}
                & T \left( \begin{bmatrix}
                    2 \\ 1
                \end{bmatrix} \right) +
                T \left( \begin{bmatrix}
                    3 \\ 2
                \end{bmatrix} \right) = \\[2ex]
                & =
                \begin{bmatrix}
                    2, 0\\
                    0, 3
                \end{bmatrix}
                \begin{bmatrix}
                    2 \\ 1
                \end{bmatrix}
                 +
                \begin{bmatrix}
                    2, 0 \\
                    0, 3
                \end{bmatrix}
                \begin{bmatrix}
                    3 \\ 2
                \end{bmatrix} \\
                & =
                \begin{bmatrix}
                    4 \\ 3
                \end{bmatrix} +
                \begin{bmatrix}
                    6 \\ 6
                \end{bmatrix} \\
                & =
                \begin{bmatrix}
                    10 \\ 9
                \end{bmatrix}
            \end{align*}
        \end{column}
    \end{columns}
\end{frame}

\begin{frame}{Multiplicación escalar}{Matrices y transformaciones lineales}

    \textbf{Misma transformación}: \textit{el doble del primero y triple del segundo}
    
    \bigskip

    $$T = \begin{bmatrix}
        2 & 0 \\
        0 & 3
    \end{bmatrix}$$

    $$T(\alpha\mathbf{v}) = \alpha T(\mathbf{v})$$

    \bigskip

    \begin{columns}
        \begin{column}{0.5\textwidth}
            \begin{align*}
                & T \left ( 1.5 
                \begin{bmatrix}
                    2 \\ 1
                \end{bmatrix} \right ) = \\[2ex]
                & =
                \begin{bmatrix}
                    2 & 0 \\
                    0 & 3
                \end{bmatrix}
                \begin{bmatrix}
                    3 \\ 1.5
                \end{bmatrix} \\
                & = 
                \begin{bmatrix}
                    6 \\ 4.5
                \end{bmatrix}
            \end{align*}
        \end{column}

        \begin{column}{0.5\textwidth}
            \begin{align*}
            & 1.5 T \left(
            \begin{bmatrix}
                2 \\ 1
            \end{bmatrix}
                \right) = \\[2ex]
            & = 1.5 \left(
                \begin{bmatrix}
                    2 & 0 \\
                    0 & 3
                \end{bmatrix}
                \begin{bmatrix}
                    2 \\ 1
                \end{bmatrix} \right ) \\
            & = 1.5 \begin{bmatrix}
                4 \\ 3
            \end{bmatrix} =
            \begin{bmatrix}
                6 \\ 4.5
            \end{bmatrix}
            \end{align*}
        \end{column}
    \end{columns}
\end{frame}

\section{Ejemplo}

\begin{frame}{Matrices de rotación}{Ejemplo}
    El actuador de un robot está en la siguiente posición (coordenadas en el espacio en $x, y, z$):
    $$\begin{bmatrix}
        2 \\ 5 \\ -2
    \end{bmatrix}$$
    El cual debe rotar 90 grados en el eje $x$. Si $\theta = 90$°, entonces podemos representar su posición nueva `transformando' su vieja posición con una matriz de rotación:

    \bigskip

    $$\begin{bmatrix}
        1 & 0 & 0 \\
        0 & \cos 90 & -\sin 90 \\
        0 & \sin 90 & \cos 90
    \end{bmatrix} \begin{bmatrix}
        2 \\ 5 \\ -2
    \end{bmatrix} = \begin{bmatrix}
        1 & 0 & 0 \\
        0 & 0 & -1 \\
        0 & 1 & 0
    \end{bmatrix} \begin{bmatrix}
        2 \\ 5 \\ -2
    \end{bmatrix}
    = \begin{bmatrix}
        2 \\ -2 \\ -5
    \end{bmatrix}$$
\end{frame}

% What is control flow
% why is it important
% does it exist in math?
% how to represent it
% how to represent it in matlab
% practical cases

% \section*{Referencias}

% \begin{frame}[t]{Referencias}
    % \nocite{bibID01}
    % \nocite{bibID02}

    % \bibliographystyle{IEEE}
    % \bibliography{biblio}
% \end{frame}

\end{document}