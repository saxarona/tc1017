\documentclass[spanish, c, handout]{beamer}

\usepackage[utf8]{inputenc}
%\usepackage[spanish, mexico]{babel}
\usepackage{hyperref}
\usepackage{xcolor}
\usepackage{color}
\usepackage{ragged2e}
\usepackage{mathrsfs}
\usepackage{csquotes}
\usepackage{listings}
\usepackage[scaled]{beramono}
\usepackage[T1]{fontenc}
\usepackage{matlab-prettifier}
\usepackage{graphicx}

\renewcommand{\indent}{\hspace*{2em}}

% \usepackage{tikz}

% \usetikzlibrary{fit, shapes, arrows}

% \usepackage{courier}
% \usepackage{subfigure}
% \usepackage{enumerate}
% \usepackage{algorithmic}
% \usepackage{algorithm}

% \usepackage{listings}
% \usepackage{lstlinebgrd}

\usetheme{Boadilla}
\usefonttheme[onlymath]{serif}

\newcommand{\matlab}[1]{\lstinline[style=Matlab-editor]!#1!}
\newcommand\blfootnote[1]{%
\begingroup
\renewcommand\thefootnote{}\footnote{#1}%
\addtocounter{footnote}{-1}%
\endgroup
}

\lstset
{
    language = Matlab,
    style = Matlab-editor,
    basicstyle = \mlttfamily\scriptsize,
    escapechar = `,
    numbers = left,
    frame = tb,
}

\lstdefinestyle{output}
{
    language = {},
    basicstyle = \mlttfamily\scriptsize,
    escapechar = `,
    numbers = none,
    showtabs = false,
   	showstringspaces = false,
}

% Sets the templates
\definecolor{navyblue}{RGB}{0, 0, 128}
\definecolor{crimson}{RGB}{128, 16, 0}

\setbeamertemplate{navigation symbols}{} 
\setbeamertemplate{headline}{}
\setbeamertemplate{footline}[frame number]
\setbeamertemplate{bibliography item}[text]
\setbeamertemplate{theorems}[numbered]

\setbeamercolor{title}{fg=navyblue, bg=white}
\setbeamercolor{frametitle}{fg=navyblue, bg=white}
\setbeamercolor{structure}{fg=navyblue}
\setbeamercolor{button}{fg=white,bg=navyblue}

\setbeamercovered{transparent}

\title{Funciones y Control de Flujo I}
\subtitle{Solución de Problemas con Programación \\ (TC1017)}
\author{
    \texorpdfstring{
        \begin{center}
            M.C. Xavier Sánchez Díaz \\
            \href{mailto:mail@tec.mx}{\texttt{mail@tec.mx}}
        \end{center}
    }
    {M.C. Xavier Sánchez Díaz}
}

\institute[Tecnológico de Monterrey]{\includegraphics[scale=0.5]{../img/logo}}
\date{}

\begin{document}

\setlength{\rightskip}{0pt}

\begin{frame}[plain]
    \titlepage        
\end{frame}

\begin{frame}{Outline}
    \tableofcontents
\end{frame}

\section{¿Qué es una función?}

\begin{frame}{Definición formal}{¿Qué es una función?}
    \begin{definition}
        Una \alert{función} \textit{unitaria} de un conjunto $A$ en un conjunto $B$ es cualquier relación binaria $R$ de $A$ a $B$ que satisfaga la condición de que \textit{para todo} $a \in A$ existe \textit{exactamente un} $b \in B$ tal que $(a,b) \in R$.
    \end{definition} \pause
    \bigskip
    Podemos describir una función $f$ de $A$ en $B$ como $f : A \to B$. \pause
    \bigskip
    \begin{exampleblock}{Ejemplo}
        La relación \textit{sucesor} es una \textbf{función} de los naturales en los naturales $f : \mathbb{N} \to \mathbb{N}$
        \[\mathtt{suc}(n) = \{(1,2), (2,3), (3,4), (4,5), \dots\}\]
    \end{exampleblock}
\end{frame}

\begin{frame}{..¿Qué?}{Qué es una función}
    \begin{definition}
        Una caja mágica que al introducirle \textit{ingredientes} devuelve \textit{resultados}.
    \end{definition} \pause

    ... con las siguientes condiciones: \pause

    \begin{itemize}
        \item La caja mágica \alert{siempre devuelve un resultado}. \pause
        \item \alert{Por cada ingrediente}, la caja mágica generará \alert{solamente un resultado}.
        \begin{itemize}
            \scriptsize
            \item Sin embargo, ingredientes distintos pueden generar el mismo resultado.
        \end{itemize} \pause
        \item \alert{Por cada ingrediente}, la caja mágica generará \alert{siempre el mismo resultado}. \pause
    \end{itemize}

    A los ingredientes los llamamos \textbf{parámetros} y a los resultados \textbf{valores de retorno}.
\end{frame}

\begin{frame}{Ejemplos}{¿Qué es una función?}
    La función \only<1-4>{\alert{$f(x) = 2x - 3$}}\only<5>{\alert{$\sin(x)$}}\only<6>{\alert{$\sqrt{x}$}} es una función porque, \textbf{usando cualquier número real}\dots \pause

    \bigskip

    \begin{itemize}[<+->]
        \item Siempre devuelve un resultado
        \item Nos da solamente un resultado
        \item Nos da siempre el mismo resultado
    \end{itemize}
\end{frame}

\begin{frame}{Elementos de una función}{¿Qué es una función}
    Cuando la \textbf{definimos} \dots
$$\alert<4>{\alert<2>{f}(\alert<3>{x})} = \alert<5>{7x^2 + 17x - 3}$$ \pause

\begin{itemize}
    \item \alert<2>{Nombre de la función}
    \item \alert<3>{Parámetro}
    \item \alert<4>{Encabezado de la función}
    \item \alert<5>{Cuerpo de la función}    
\end{itemize}
\end{frame}

\begin{frame}{Elementos de una función}{¿Qué es una función}
    Cuando la \textbf{evaluamos} por ejemplo, con $x = 2$ \dots
$$\alert<4>{\alert<2>{f}(\alert<3>{2})} = 7(2)^2 + 17(2) - 3 = \alert<5>{59}$$ \pause

\begin{itemize}
    \item \alert<2>{Nombre de la función}
    \item \alert<3>{Argumento}
    \item \alert<4>{Llamada o evaluación de la función}
    \item \alert<5>{Valor de retorno}
\end{itemize}
\end{frame}

\section{Funciones en MATLAB}

\begin{frame}{Manos a la obra}{Funciones en MATLAB}

    Todos los comandos en MATLAB están dados de alta internamente en el software como \textbf{funciones}. \pause

    \bigskip

    \begin{itemize}[<+->]
        \itemsep2ex
        \item \matlab{sin} necesita un parámetro para operar: \matlab{sin 67} significa $\sin(67)$
        \item \matlab{sqrt} necesita un parámetro para operar: \matlab{squrt 2} significa $\sqrt{2}$
        \item Tanto \matlab{sin} como \matlab{sqrt} cumplen con las reglas de siempre devolver resultados, sólo devolver un resultado, y devolver siempre el mismo.
    \end{itemize}

\end{frame}

\begin{frame}[fragile]{La palabra reservada \matlab{function}}{Funciones en MATLAB}
    Para implementar funciones en MATLAB es necesario utilizar la palabra reservada \matlab{function}.
    
    \bigskip

    En el editor:

    \bigskip

    \begin{columns}
        \begin{column}{0.95\textwidth}
            \begin{lstlisting}
function `\mlplaceholder{return\_value}` = `\mlplaceholder{function\_name}`(`\mlplaceholder{parameter}`)
    `\mlplaceholder{body of the function}`
end
            \end{lstlisting}
        \end{column}
    \end{columns}

\end{frame}

\begin{frame}[fragile]{La palabra reservada \matlab{function}}{Funciones en MATLAB}
    \begin{columns}
        \begin{column}{0.95\linewidth}
            Editor:
            \bigskip
            \lstinputlisting{successor.m}

            \bigskip
            
            Command Window:
            \bigskip
\begin{lstlisting}[style=output]
>> successor(9)

ans =

     10
\end{lstlisting}
        \end{column}
    \end{columns}
\end{frame}

\begin{frame}[fragile]{La palabra reservada \matlab{function}}{Funciones en MATLAB}
    \begin{columns}
        \begin{column}{0.95\linewidth}
            Editor:
            \bigskip
            \lstinputlisting{donothing.m}

            \bigskip
            
            Command Window:
            \bigskip
\begin{lstlisting}[style=output]
>> donothing(5)

ans =

 5
\end{lstlisting}
        \end{column}
    \end{columns}
\end{frame}

               
\begin{frame}[fragile]{La palabra reservada \matlab{function}}{Funciones en MATLAB}
    \begin{columns}
        \begin{column}{0.95\linewidth}
            Editor:
            \bigskip
            \lstinputlisting{fancyname.m}

            \bigskip
            
            Command Window:
            \bigskip
\begin{lstlisting}[style=output]
>> fancyname(2)

ans =

     15
\end{lstlisting} 
        \end{column}
        
    \end{columns}
\end{frame}

\section{Consideraciones adicionales}

\begin{frame}{Funciones de más de un parámetro}{Consideraciones adicionales}
    Hasta ahora hemos visto funciones \textit{unitarias}:

    $$f(x) = 2x$$ \pause

    Sin embargo, también puede haber funciones con una \alert{aridad} mayor a uno:

    $$f(x, y) = 7x + 3y$$ \pause

    ¿Se te ocurre alguna función matemática \textit{con nombre} que reciba más de un parámetro?

\end{frame}

\begin{frame}{Más preguntas interesantes con funciones}{Consideraciones adicionales}

    Si podemos tener más de un parámetro, ¿podemos tener más de un valor de retorno? \pause

    \bigskip

    \begin{flushright}
        \footnotesize \textit{Sí, usualmente ordenados y agrupados. Esto lo veremos más adelante, en estructuras de datos básicas}.
    \end{flushright} \pause

    \bigskip

    ¿Puedo hacer una función que no me dé resultado alguno? \pause

    \bigskip

    \begin{flushright}
        \footnotesize \textit{Sí, de hecho son muy frecuentes. Sin embargo, si no devuelve nada, ya no es propiamente una función.
        A estas funciones sin valor de retorno les llamamos \alert{procedimientos} (procedures en inglés)}. \pause
    \end{flushright}

    \bigskip

    ¿Puedo hacer una función que no reciba un parámetro? \pause

    \begin{flushright}
        \footnotesize \textit{Sí, sí puedes. Sin embargo no tiene mucho sentido. Lo recomendable es simplemente hacer una secuencia de instrucciones. A esta serie de instrucciones les llamamos \alert{scripts}}.
    \end{flushright}

\end{frame}

\begin{frame}{En cómputo, todo es una función}{Consideraciones adicionales}
    Específicamente en MATLAB, tenemos algunas redundancias para facilitar la escritura: \pause
    
    \begin{itemize}
        \itemsep2.5ex
        \item \matlab{plus (1, 2)} es lo mismo que \matlab{1 + 2} \pause
        \item \matlab{a == 5} es lo mismo que \matlab{eq(a, 5)} \pause
        \item \matlab{mtimes(5, 2)} es otra manera de decir \matlab{5 * 2} \pause
    \end{itemize}

    \bigskip

    ¿Qué otros operadores conocemos?
\end{frame}

\begin{frame}{Funciones seccionadas (Piecewise)}{Consideraciones adicionales}
    Hasta ahora, todas las funciones que hemos visto son hermosas porque son continuas y por tanto diferenciables.
    Sin embargo, no todas las funciones son así\dots \pause

    \bigskip

    ¿Cuál es el valor absoluto de $2$? \pause ¿Y el valor absoluto de $-2$? \pause

    $$|x| =\left\{ \begin{aligned}
        -&x , &\text{si } x < 0 \\
        &x ,  & \text{si } x \ge 0
    \end{aligned} \right.
    $$ \pause
    
    \bigskip

    ¿Cómo representamos esto en MATLAB?
\end{frame}

\begin{frame}[fragile]{Primer encuentro con el control de flujo}{Consideraciones adicionales}
    Los \alert{condicionales} son las \textbf{estructuras de control} de flujo más comunes.

    \bigskip

    \begin{columns}
        \begin{column}{0.95\linewidth}
\begin{lstlisting}
if `\mlplaceholder{condition}`
  % Si se cumple
  `\mlplaceholder{do something}`
else
  % Si no se cumple
  `\mlplaceholder{do something else}`
end
\end{lstlisting}
        \end{column}
        
    \end{columns}

\bigskip

¿Puedes implementar la función del valor absoluto en MATLAB?
\end{frame}

% what is a function
% components of a function
% function and procedures
% procedures as scripts
% built-in functions
% documenting

% \section*{Referencias}

% \begin{frame}[t]{Referencias}
    % \nocite{bibID01}
    % \nocite{bibID02}

    % \bibliographystyle{IEEE}
    % \bibliography{biblio}
% \end{frame}

\end{document}