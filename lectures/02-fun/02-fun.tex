\documentclass[spanish]{beamer}

\usepackage[utf8]{inputenc}
%\usepackage[spanish, mexico]{babel}
\usepackage{hyperref}
\usepackage{xcolor}
\usepackage{color}
\usepackage{ragged2e}
\usepackage{mathrsfs}
\usepackage{csquotes}

% \usepackage{tikz}

% \usetikzlibrary{fit, shapes, arrows}

% \usepackage{courier}
% \usepackage{subfigure}
% \usepackage{enumerate}
% \usepackage{algorithmic}
% \usepackage{algorithm}

% \usepackage{listings}
% \usepackage{lstlinebgrd}

\usetheme{Boadilla}
\usefonttheme[onlymath]{serif}

\newcommand\blfootnote[1]{%
\begingroup
\renewcommand\thefootnote{}\footnote{#1}%
\addtocounter{footnote}{-1}%
\endgroup
}

% Sets the templates
\definecolor{navyblue}{RGB}{0, 0, 128}
\definecolor{crimson}{RGB}{128, 16, 0}

\setbeamertemplate{navigation symbols}{} 
\setbeamertemplate{headline}{}
\setbeamertemplate{footline}[frame number]
\setbeamertemplate{bibliography item}[text]
\setbeamertemplate{theorems}[numbered]

\setbeamercolor{title}{fg=navyblue, bg=white}
\setbeamercolor{frametitle}{fg=navyblue, bg=white}
\setbeamercolor{structure}{fg=navyblue}
\setbeamercolor{button}{fg=white,bg=navyblue}

\setbeamercovered{transparent}

\title{Conceptos matemáticos preliminares}
\subtitle{Solución de Problemas con Programación \\ (TC1017)}
\author{
    \texorpdfstring{
        \begin{center}
            M.C. Xavier Sánchez Díaz \\
            \href{mailto:sax@tec.mx}{\texttt{sax@tec.mx}}
        \end{center}
    }
    {M.C. Xavier Sánchez Díaz}
}

\institute[Tecnológico de Monterrey]{\includegraphics[scale=0.5]{../img/logo}}
\date{}

\begin{document}

\setlength{\rightskip}{0pt}

\begin{frame}[plain]
    \titlepage        
\end{frame}

\begin{frame}{Outline}
    \tableofcontents
\end{frame}

\section{¿Qué es una función?}

\begin{frame}{Definición formal}{¿Qué es una función?}
    \begin{definition}
        Una \alert{función} \textit{unitaria} de un conjunto $A$ en un conjunto $B$ es cualquier relación binaria $R$ de $A$ a $B$ que satisfaga la condición de que \textit{para todo} $a \in A$ existe \textit{exactamente un} $b \in B$ tal que $(a,b) \in R$.
    \end{definition} \pause
    \bigskip
    Podemos describir una función $f$ de $A$ en $B$ como $f : A \to B$. \pause
    \bigskip
    \begin{exampleblock}{Ejemplo}
        La relación \textit{sucesor} es una \textbf{función} de los naturales en los naturales $f : \mathbb{N} \to \mathbb{N}$
        \[\mathtt{suc}(n) = \{(1,2), (2,3), (3,4), (4,5), \dots\}\]
    \end{exampleblock}
\end{frame}

\begin{frame}{..¿Qué?}{Qué es una función}
    \begin{definition}
        Una caja mágica que al introducirle \textit{ingredientes} devuelve \textit{resultados}.
    \end{definition} \pause

    ... con las siguientes condiciones: \pause

    \begin{itemize}
        \item La caja mágica \alert{siempre devuelve un resultado}. \pause
        \item \alert{Por cada ingrediente}, la caja mágica generará \alert{solamente un resultado}.
        \vspace{-2ex}
        \begin{itemize}
            \scriptsize
            \item Sin embargo, ingredientes distintos pueden generar el mismo resultado.
        \end{itemize} \pause
        \item \alert{Por cada ingrediente}, la caja mágica generará \alert{siempre el mismo resultado}. \pause
    \end{itemize}

    A los ingredientes los llamamos \textbf{parámetros} y a los resultados \textbf{valores de retorno}.
\end{frame}

\begin{frame}{Ejemplos}{¿Qué es una función?}
    La función \only<1-4>{\alert{$f(x) = 2x - 3$}}\only<5>{\alert{$\sin(x)$}}\only<6>{\alert{$\sqrt{x}$}} es una función porque, \textbf{usando cualquier número real}\dots \pause

    \bigskip

    \begin{itemize}[<+->]
        \item Siempre devuelve un resultado
        \item Nos da solamente un resultado
        \item Nos da siempre el mismo resultado
    \end{itemize}
\end{frame}

\begin{frame}{Elementos de una función}{¿Qué es una función}
    Cuando la \textbf{definimos} \dots
$$\alert<4>{\alert<2>{f}(\alert<3>{x})} = \alert<5>{7x^2 + 17x - 3}$$ \pause

\begin{itemize}
    \item \alert<2>{Nombre de la función}
    \item \alert<3>{Parámetro}
    \item \alert<4>{Encabezado de la función}
    \item \alert<5>{Cuerpo de la función}    
\end{itemize}
\end{frame}

\begin{frame}{Elementos de una función}{¿Qué es una función}
    Cuando la \textbf{evaluamos} por ejemplo, con $x = 2$ \dots
$$\alert<4>{\alert<2>{f}(\alert<3>{2})} = 7(2)^2 + 17(2) - 3 = \alert<5>{59}$$ \pause

\begin{itemize}
    \item \alert<2>{Nombre de la función}
    \item \alert<3>{Argumento}
    \item \alert<4>{Llamada o evaluación de la función}
    \item \alert<5>{Valor de retorno}
\end{itemize}
\end{frame}

% what is a function
% components of a function
% function and procedures
% procedures as scripts
% built-in functions
% documenting

% \section*{Referencias}

% \begin{frame}[t]{Referencias}
    % \nocite{bibID01}
    % \nocite{bibID02}

    % \bibliographystyle{IEEE}
    % \bibliography{biblio}
% \end{frame}

\end{document}