\documentclass[spanish, c, handout]{beamer}

\usepackage[utf8]{inputenc}
%\usepackage[spanish, mexico]{babel}
\usepackage{hyperref}
\usepackage{xcolor}
\usepackage{color}
\usepackage{ragged2e}
\usepackage{mathrsfs}
\usepackage{csquotes}
\usepackage{listings}
\usepackage[scaled]{beramono}
\usepackage[T1]{fontenc}
\usepackage{matlab-prettifier}
\usepackage{graphicx}

\renewcommand{\indent}{\hspace*{2em}}

% \usepackage{tikz}

% \usetikzlibrary{fit, shapes, arrows}

% \usepackage{courier}
% \usepackage{subfigure}
% \usepackage{enumerate}
% \usepackage{algorithmic}
% \usepackage{algorithm}

% \usepackage{listings}
% \usepackage{lstlinebgrd}

\usetheme{Boadilla}
\usefonttheme[onlymath]{serif}

\newcommand{\matlab}[1]{\lstinline[style=Matlab-editor]!#1!}
\newcommand\blfootnote[1]{%
\begingroup
\renewcommand\thefootnote{}\footnote{#1}%
\addtocounter{footnote}{-1}%
\endgroup
}

\lstset
{
    language = Matlab,
    style = Matlab-editor,
    basicstyle = \mlttfamily\scriptsize,
    escapechar = `,
    numbers = left,
    frame = tb,
}

\lstdefinestyle{output}
{
    language = {},
    basicstyle = \mlttfamily\scriptsize,
    escapechar = `,
    numbers = none,
    showtabs = false,
   	showstringspaces = false,
}

% Sets the templates
\definecolor{navyblue}{RGB}{0, 0, 128}
\definecolor{crimson}{RGB}{128, 16, 0}

\setbeamertemplate{navigation symbols}{} 
\setbeamertemplate{headline}{}
\setbeamertemplate{footline}[frame number]
\setbeamertemplate{bibliography item}[text]
\setbeamertemplate{theorems}[numbered]

\setbeamercolor{title}{fg=navyblue, bg=white}
\setbeamercolor{frametitle}{fg=navyblue, bg=white}
\setbeamercolor{structure}{fg=navyblue}
\setbeamercolor{button}{fg=white,bg=navyblue}

\setbeamercovered{transparent}

\title{Control de Flujo}
\subtitle{Solución de Problemas con Programación \\ (TC1017)}
\author{
    \texorpdfstring{
        \begin{center}
            M.C. Xavier Sánchez Díaz \\
            \href{mailto:sax@tec.mx}{\texttt{sax@tec.mx}}
        \end{center}
    }
    {M.C. Xavier Sánchez Díaz}
}

\institute[Tecnológico de Monterrey]{\includegraphics[scale=0.5]{../img/logo}}
\date{}

\begin{document}

\setlength{\rightskip}{0pt}

\begin{frame}[plain]
    \titlepage        
\end{frame}

\begin{frame}{Outline}
    \tableofcontents
\end{frame}

\section{Control de Flujo}

\begin{frame}{¿Qué es control de flujo?}{Control de Flujo}
    
    Un lenguaje de programación nos permite expresar la \alert<2->{manera en la que los componentes de ejecución se entrelazan} para realizar un cálculo. \pause

    \bigskip \pause

    De este modo, tenemos distintas herramientas a nuestra disposición para poder ejecutar comandos y obtener los resultados que queremos de distintas maneras: \pause

    \bigskip

    \begin{enumerate}
        \item Usando \alert<8->{secuencias de instrucciones} \pause
        \item \alert<8->{Seleccionando} entre distintas opciones \pause
        \item \alert<8->{Repitiendo} procesos \pause
        \item Haciendo múltiples procesos a la vez \pause
    \end{enumerate}

\end{frame}

\begin{frame}{Ejemplos}{Control de Flujo}
    \begin{itemize}[<+->]
        \itemsep3ex
        \item ¿Me traes una coca? \alert<7>{Si} hay sin azúcar, mejor. \alert<7>{Si no}, la que sea.
        \item Tengo que ir al cajero, \alert<6>{luego} a la lavandería y \alert<6>{luego} pasar por algo de comer.
        \item Necesito que me ayudes a calificar \alert<8>{cada uno} de estos ejercicios.
        \item Ya hablaremos de eso \alert<7>{cuando} seas más grande.
        \item Vas a tener que buscar en \alert<8>{cada} rincón de la casa para encontrarlo.
    \end{itemize}
\end{frame}

\begin{frame}{Ejemplos}{Control de Flujo}

    Podemos identificar con claridad los tres tipos de \textit{maneras} en que podemos entrelazar los componentes de las oraciones: \pause

    \bigskip

    \begin{itemize}
        \item Instrucciones \alert{secuenciales} \only<2>{(Primero voy acá, luego allá\dots)} \pause
        \item Instrucciones \alert{condicionales} \only<3>{(Si esto, entonces aquello, si no\dots)} \pause
        \item Instrucciones \alert{cíclicas} \only<4>{(Para cada uno de estos, haz tal y tal\dots)} \pause
    \end{itemize}

    \bigskip
    
    Estos son las tres \textbf{estructuras de control de flujo} que revisaremos en este curso.
\end{frame}

\section{Representaciones del control de flujo}

\begin{frame}{Secuencias}{Representaciones del control de flujo}
    \blfootnote{Paráfrasis de Stover, C. y Weisstein, E., de \textit{Set} en MathWorld: \url{http://mathworld.wolfram.com/Set.html}}
    Para comprender qué es una \textbf{secuencia}, hay que estudiar primero qué es un \alert{conjunto}. \pause

    \bigskip

    \begin{block}{Conjunto}
        \begin{displayquote}
            A \alert{set} is a finite or infinite \textbf{collection} of objects in which order has no significance, and multiplicity is ignored.
            Members of a set are referred to as \textbf{elements} and the notation $a \in A$ is used to denote that $a$ is an element of a set $A$.
        \end{displayquote}
    \end{block}
\end{frame}

\begin{frame}{Secuencias}{Representaciones del control de flujo}
    \begin{exampleblock}{Ejemplos de conjuntos}
        \begin{itemize}[<+->]
            \itemsep3.5ex
            \item El conjunto de números naturales $\mathbb{N} = \{1, 2, 3, \dots\}$
            \item El conjunto de funciones que tiene el MATLAB
            \item El conjunto de estudiantes del campus
            \item El conjunto de estudiantes en la clase
            \item El conjunto de estudiantes en el salón
        \end{itemize}
    \end{exampleblock}
\end{frame}

\begin{frame}{Secuencias}{Representaciones del control de flujo}
    Una \alert{secuencia} es un \textbf{conjunto} el cual está \alert{ordenado}. \pause

    \begin{exampleblock}{Ejemplo}
        Los números naturales son un conjunto ordenado:

        $$\mathbb{N} = \langle 1, 2, 3, 4, \dots \rangle$$
        
    \end{exampleblock} \pause

    Una receta de cocina es una \textbf{secuencia de instrucciones} porque el orden es importante. \pause
    
    Una \textbf{guión} es también una secuencia de diálogos, pues una obra de teatro o una película tienen una naturaleza secuencial.
    Este \textbf{guión} se llama en inglés \textbf{script}.
\end{frame}

\begin{frame}[fragile]{Secuencias}{Representaciones del control de flujo}

    Los \alert{scripts} son naturalmente secuenciales:

    \bigskip

    \begin{columns}
        \begin{column}{0.95\linewidth}
            \lstinputlisting{script.m}
        \end{column}
    \end{columns}

\end{frame}

\begin{frame}{Condicionales}{Representaciones del control de flujo}
    \blfootnote{Paráfrasis de Weisstein, E., de \textit{Implies} en MathWorld: \url{http://mathworld.wolfram.com/Implies.html}}

    Los estatutos condicionales \textit{también} tienen una base matemática: \pause

    \bigskip

    \begin{block}{Implicación lógica}
        \begin{displayquote}
            ``Implies'' is the connective in \textbf{propositional calculus} which has the meaning that ``if $A$ is true, then $B$ is also true''.
            In formal terminology, the term conditional is often used to refer to this connective. The symbol used to denote the implication is $$A \implies B$$
        \end{displayquote}
    \end{block}
\end{frame}

\begin{frame}{Condicionales}{Representaciones del control de flujo}

    Los condicionales son de naturaleza \textbf{disyuntiva}:

    \bigskip

    \begin{columns}
        \begin{column}{0.9\linewidth}
            \lstinputlisting{positiveornot.m}
        \end{column}
    \end{columns}

\end{frame}

\begin{frame}{Ciclos}{Representaciones del control de flujo}
    Los \alert{ciclos}, también conocidos como \textbf{bucles}, son estructuras de flujo cuya representación matemática es más bien implícita. \pause

    \bigskip

    ¿Cuál es la suma de todos los números que están en un reloj análogo? \pause Pensemos un poco\dots
\end{frame}

\begin{frame}{Ciclos}{Representaciones del control de flujo}

    \begin{columns}
        \begin{column}{0.5\linewidth}
            \begin{align*}
                R = & 12  + 11 + 10 \\
                  + &  9 + 8 + 7 \\
                  + &  6 + 5 + 4 \\
                  + &  3 + 2 + 1 \\
                  = & \mathbf{78}
            \end{align*} \pause
        \end{column}

        \begin{column}{0.5\linewidth}
            Hay que sumar el 12, más el 11, más el 10, más el\dots hay un ciclo que podemos identificar, donde el \textbf{valor base} va cambiando de uno en uno\dots

            A este valor base le llamamos \alert{iterador}. \pause
        \end{column}
    \end{columns}

    Y esta suma, que se lee como \textit{la suma de $i$, desde 1 hasta 12}, se representa de la siguiente manera:
    
    $$\sum\limits_{i=1}^{i=12} i = 78$$
\end{frame}

\begin{frame}{Ciclos}{Representaciones del control de flujo}
    El \alert{iterador} va de uno en uno en los elementos de una \textbf{secuencia}, que en el caso anterior era la secuencia de los números del 1 al 12.

    \bigskip

    Estos ciclos en los cuales sabemos específicamente \textbf{cuántas veces} hay que repetir el proceso se conocen como \alert{ciclo PARA}---o en inglés, \textit{for}--- pues es \textit{para cada uno}\dots

    \bigskip

    \begin{columns}
        \begin{column}{0.95\linewidth}
            \lstinputlisting{basicfor.m}
        \end{column}
    \end{columns}
\end{frame}

\begin{frame}{Ciclos}{Representaciones del control de flujo}    
    Otro ejemplo del ciclo \matlab{for}\dots
    
    \bigskip
    \begin{columns}
        \begin{column}{0.95\linewidth}
            \lstinputlisting{advancedfor.m}
        \end{column}
    \end{columns}
\end{frame}

\begin{frame}[fragile]{Ciclos}{Representaciones del control de flujo}
    Como hemos visto, el \matlab{for} necesita de una \alert{secuencia} o \alert{lista} para poder \textbf{iterar} en ella.
    Si el patrón de esta secuencia es muy claro, podemos expresarla por \textit{comprensión} como si fuera un conjunto:

    $$X = \{x \mid 0 < x < 15, x \text{ es impar}\}$$

    Esto lo podemos expresar en el MATLAB de una manera similar:

    \begin{columns}
        \begin{column}{0.95\linewidth}
                \begin{lstlisting}[frame=none, numbers=none]
                        `\mlplaceholder{start}`:`\mlplaceholder{increment}`:`\mlplaceholder{stop}`
                \end{lstlisting}
    \bigskip

    \lstinputlisting{colonex.m}
        \end{column}
    \end{columns}
\end{frame}

% What is control flow
% why is it important
% does it exist in math?
% how to represent it
% how to represent it in matlab
% practical cases

% \section*{Referencias}

% \begin{frame}[t]{Referencias}
    % \nocite{bibID01}
    % \nocite{bibID02}

    % \bibliographystyle{IEEE}
    % \bibliography{biblio}
% \end{frame}

\end{document}