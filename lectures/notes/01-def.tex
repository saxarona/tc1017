% Template created by Karol Kozioł (www.karol-koziol.net) for ShareLaTeX

\documentclass[letter,spanish,9pt]{extarticle}
\usepackage[utf8]{inputenc}

\usepackage[T1]{fontenc}
\usepackage{verbatim}
\usepackage{graphicx}
\usepackage{xcolor}
\usepackage{pgf,tikz}
\usepackage{mathrsfs}

\usetikzlibrary{shapes, calc, shapes, arrows, babel}

\usepackage{amsmath,amssymb,textcomp}
\everymath{\displaystyle}

\usepackage{times}
\renewcommand\familydefault{\sfdefault}
\usepackage{tgheros}
\usepackage[defaultmono,scale=0.85]{droidmono}

\usepackage{multicol}
\setlength{\columnseprule}{0pt}
\setlength{\columnsep}{20.0pt}

\usepackage[utf8]{inputenc}
\usepackage[spanish]{babel}
\usepackage{eurosym}

\usepackage{graphicx}
\graphicspath{{../img/}}
\usepackage{hyperref}

\usepackage{geometry}
\geometry{
letterpaper,
total={210mm,297mm},
left=5mm,right=5mm,top=5mm,bottom=15mm}

\linespread{1.1}

\newcommand{\samedir}{\mathbin{\!/\mkern-5mu/\!}}

% custom title
\makeatletter
\renewcommand*{\maketitle}{%
\noindent
\begin{minipage}{0.72\textwidth}
\begin{tikzpicture}
\node[rectangle,rounded corners=6pt,inner sep=10pt,fill=blue!50!black,text width= 0.95\textwidth] {\color{white}{\Huge \@title}\\[2ex] \Large \@author};
\end{tikzpicture}
\end{minipage}
\hfill
\begin{minipage}{0.27\textwidth}
\includegraphics[width=0.95\columnwidth]{logo}
\end{minipage}
\bigskip\bigskip
}%
\makeatother

% custom section
\usepackage[explicit]{titlesec}
\newcommand*\sectionlabel{}
\titleformat{\section}
  {\gdef\sectionlabel{}
   \normalfont\sffamily\Large\bfseries\scshape}
  {\gdef\sectionlabel{\thesection\ }}{0pt}
  {
\noindent
\begin{tikzpicture}
\node[rectangle,rounded corners=3pt,inner sep=4pt,fill=blue!50!black,text width= 0.95\columnwidth] {\color{white}\sectionlabel#1};
\end{tikzpicture}
  }
\titlespacing*{\section}{0pt}{12pt}{10pt}


% custom footer
\usepackage{fancyhdr}
\makeatletter
\pagestyle{fancy}
\fancyhead{}
\fancyfoot[C]{\footnotesize \@author - \tec}
\fancyfoot[R]{\thepage}
\renewcommand{\headrulewidth}{0pt}
\renewcommand{\footrulewidth}{0pt}
\makeatother
\usepackage{multirow} % para las tablas


\title{TC1017 -- Solución de Problemas con Programación}
\author{Xavier Sánchez Díaz}
\date{2019} 
\newcommand{\tec}{Tecnológico de Monterrey}



\begin{document}

\maketitle



\begin{multicols*}{2}


\section*{Términos importantes}

\begin{description}
  \item [Arreglo.] Conjunto ordenado de datos, usualmente sinónimo de `Vector', que es un arreglo de una dimensión:
    $$\begin{bmatrix}
      1 & 2 & 3 & 4
    \end{bmatrix}$$
  \item [Bucle.] Estatuto específico para hacer ciclos.
  \item [Cadena de caracteres.] Un `arreglo de letras o números'.
  \item [Condicional.] Un estatuto específico para tomar decisiones.
  \item [Entero.] Todos los números sin parte decimal. Un número $z$ es parte de los enteros que se representan con $\mathbb{Z}$.
  \item [Estatuto.] Cualquier expresión válida y completa en un lenguaje de programación.
  \item [Estructura.] Abstracción para representar datos. Los arreglos, las matrices y los árboles son estructuras de datos.
  \item [Función.] Una función es una `caja' a la cuál metes \textit{algo} y se transforma a \textit{otra cosa}. Para la función $f(x) = 2x+3$, si metes $x=5$, entonces la función lo convierte en $13$, pues $f(5)=2(5)+3 = 13$.
  \item [Imprimir.] Mostrar información en pantalla.
  \item [Iteración.] Una vuelta (ejecución) en un ciclo.
  \item [Matriz.] Arreglo de dos dimensiones. Usualmente visualizado como una tabla:
    $$\begin{bmatrix}
      2 & 3 & 4 \\
      1 & 2 & 3 \\
      9 & 2 & 3
    \end{bmatrix}$$
  \item [Operador.] Símbolo reservado que es un alias para una función de dos parámetros. $+,-,/,*,:$ son operadores. La exprsión $2+5$ es la mismo que al escribir $+(2,5)$ en algunos lenguajes.
  \item [Palabra reservada]. Palabras que no pueden ser usadas como nombres, ya que significan algo para el lenguaje de programación.
  \item [Parámetro.] El \textit{algo} que metes a la función. En la función \texttt{max(A)}, \texttt{A} es el parámetro. Un sinónimo es \textit{argumento}.
  \item [Procedimiento.] Una rutina, específicamente sin valor de retorno.
  \item [Punto Flotante.] Representación computacional de cada uno de los números con parte decimal (incluso si esa parte es 0).
  \item [Rutina.] Una secuencia ordenada de instrucciones, usualmente con nombre, que no necesariamente tiene un valor de retorno.
  \item [Tipo.] Clasificación de una variable dependiendo del tipo de dato que vaya a guardar.
  \item [Valor de Retorno.] Aquello que se \textit{devuelve} al terminar una función. El resultado.
  \item [Variable.] Dirección de memoria que guarda un valor y a la cual se le asigna un nombre, para poder referenciarla fácilmente después.  
\end{description}

\section*{}

\end{multicols*}

\end{document}
