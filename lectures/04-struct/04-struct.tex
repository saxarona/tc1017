\documentclass[spanish, c]{beamer}

\usepackage[utf8]{inputenc}
%\usepackage[spanish, mexico]{babel}
\usepackage{hyperref}
\usepackage{xcolor}
\usepackage{color}
\usepackage{ragged2e}
\usepackage{mathrsfs}
\usepackage{csquotes}
\usepackage{listings}
\usepackage[scaled]{beramono}
\usepackage[T1]{fontenc}
\usepackage{matlab-prettifier}
\usepackage{graphicx}

\renewcommand{\indent}{\hspace*{2em}}

% \usepackage{tikz}

% \usetikzlibrary{fit, shapes, arrows}

% \usepackage{courier}
% \usepackage{subfigure}
% \usepackage{enumerate}
% \usepackage{algorithmic}
% \usepackage{algorithm}

% \usepackage{listings}
% \usepackage{lstlinebgrd}

\usetheme{Boadilla}
\usefonttheme[onlymath]{serif}

\newcommand{\matlab}[1]{\lstinline[style=Matlab-editor]!#1!}
\newcommand\blfootnote[1]{%
\begingroup
\renewcommand\thefootnote{}\footnote{#1}%
\addtocounter{footnote}{-1}%
\endgroup
}

\lstset
{
    language = Matlab,
    style = Matlab-editor,
    basicstyle = \mlttfamily\scriptsize,
    escapechar = `,
    numbers = left,
    frame = tb,
}

\lstdefinestyle{output}
{
    language = {},
    basicstyle = \mlttfamily\scriptsize,
    escapechar = `,
    numbers = none,
    showtabs = false,
   	showstringspaces = false,
}

% Sets the templates
\definecolor{navyblue}{RGB}{0, 0, 128}
\definecolor{crimson}{RGB}{128, 16, 0}

\setbeamertemplate{navigation symbols}{} 
\setbeamertemplate{headline}{}
\setbeamertemplate{footline}[frame number]
\setbeamertemplate{bibliography item}[text]
\setbeamertemplate{theorems}[numbered]

\setbeamercolor{title}{fg=navyblue, bg=white}
\setbeamercolor{frametitle}{fg=navyblue, bg=white}
\setbeamercolor{structure}{fg=navyblue}
\setbeamercolor{button}{fg=white,bg=navyblue}

\setbeamercovered{transparent}

\title{Estructuras de datos}
\subtitle{Solución de Problemas con Programación \\ (TC1017)}
\author{
    \texorpdfstring{
        \begin{center}
            M.C. Xavier Sánchez Díaz \\
            \href{mailto:mail@tec.mx}{\texttt{mail@tec.mx}}
        \end{center}
    }
    {M.C. Xavier Sánchez Díaz}
}

\institute[Tecnológico de Monterrey]{\includegraphics[scale=0.5]{../img/logo}}
\date{}

\begin{document}

\setlength{\rightskip}{0pt}

\begin{frame}[plain]
    \titlepage        
\end{frame}

\begin{frame}{Outline}
    \tableofcontents
\end{frame}

\section{Operaciones}

\begin{frame}{Tipos de datos}{Datos y operaciones}
    
    Como ya vimos, existen distintos \alert{tipos de datos} con los que podemos trabajar en la computadora: \pause

    \bigskip

    \begin{itemize}[<+->]
        \item Números enteros (\textit{integer numbers})
        \item Números decimales (\textit{floating point numbers})
        \item Cadenas de caracteres alfanuméricos (\textit{strings})
    \end{itemize}

    \bigskip
    
    Cuando \textbf{operamos} con estos datos, generamos nueva información que podríamos necesitar en el futuro.

\end{frame}

\begin{frame}{Datos como resultados}{Datos y operaciones}
    Antes de usar la computadora o la calculadora para hacer cálculos, solíamos hacer las operaciones a mano.
    Por ejemplo, si queremos calcular $1270 \times 35$, una manera de hacerlo podría ser\dots

\end{frame}

\begin{frame}{Datos como resultados}{Datos y operaciones}
    \begin{align*}
        1270 \times 35 & = \\ \pause
        & = (1200 + 70) \times (7)(5) \\ \pause
        & = (\alert<4>{12})(\alert<5,6>{7})(\alert<4>{5})(\alert<7>{100}) + (\alert<8>{7})(\alert<8>{7})(\alert<10>{5})(\alert<9>{10}) \pause
    \end{align*}
    \vspace{-2ex}
    \begin{align*}
        \alert<4>{12} \times \alert<4>{5} = \alert<5,6>{60} \\
        \alert<5,6>{60} \times \alert<5,6>{7} = \alert<6>{6} \times \alert<6>{7} \times \alert<6>{10} = \alert<7>{420} \\
        \alert<7>{420} \times \alert<7>{100} = \alert<11>{42000} \\[1.5ex]
        \alert<8>{7} \times \alert<8>{7} = \alert<9>{49} \\
        \alert<9>{49} \times \alert<9>{10} = \alert<10>{490} \\
        \alert<10>{490} \times \alert<10>{5} = \alert<10>{490} \times \alert<10>{10 / 2} = \alert<11>{2450}\\[1.5ex]
        \alert<11>{42000} + \alert<11>{2450} = 44450 \quad \square
    \end{align*}
\end{frame}

\begin{frame}{Datos como resultados}{Datos y operaciones}    
    La operación completa se hace poco a poco, y por tanto necesitamos ``recordar'' ciertos pasos intermedios que ya tenemos calculados. \pause

    \bigskip
    
    Así como nosotros tenemos que tener en claro cuáles son esos pasos intermedios, la computadora debe saber \textit{dónde está} la información que tiene que leer para trabajar y hacer cálculos más elaborados. \pause

    \bigskip

    Para eso, podemos usar las estructuras de datos, para \textbf{ordenarlos} de manera conveniente y poder tener acceso a ellos de manera que se vayan necesitando.

\end{frame}

\section{Estructuras matemáticas}

\begin{frame}{Variables}{Estructuras matemáticas}
    La forma más simple de guardar \textbf{un solo dato} es usando una \alert{variable}. \pause

    En álgebra, hemos usado estas \textit{variables} para expresar qué hace una función y saber su resultado: \pause

    $$\alert<5>{y} = 2\alert<4>{x}^2 + 3\alert<4>{x} + 5$$ \pause

    \begin{itemize}
        \item $\alert<4>{x}$ es una variable la cual no sabemos su valor en este momento, pero sabemos qué hacer con ella
        \item $\alert<5>{y}$ es otra variable (porque tiene distinto nombre) y no sabemos su valor ahora, pero sabemos que guardará el resultado de la operación
    \end{itemize}

\end{frame}

\begin{frame}[t]{Variables}{Estructuras matemáticas}    
    $$\alert<3>{y} = 2\alert<2>{x}^2 + 3\alert<2>{x} + 5$$

    \bigskip

    Si ahora le damos valor a $x$, por ejemplo, $x = 3$\dots \pause

    \bigskip

    \begin{itemize}
        \itemsep1.7ex
        \item $\alert<2>{x}$ guarda el valor de $3$
        \item $\alert<3>{y}$ guarda $2 (3)^2 + (3)(3) + 5 = 18 + 9 + 5 = 32$
    \end{itemize}

\end{frame}

\begin{frame}{Arreglos}{Estructuras matemáticas}
    Asumamos que quiero saber las calificaciones de las Tareas 1, 2 y 3 de uno de mis alumnos.
    Para esto, necesitaría un lugar para guardar esos \textbf{3 datos}: \pause

    \bigskip

    \begin{itemize}[<+->]
        \item $t_1 = 90$ será la variable para la Tarea 1
        \item $t_2 = 75$ será la variable para la Tarea 2
        \item $t_3 = 87$ será la variable para la Tarea 3
    \end{itemize}

    \bigskip

    Con esta información, ahora contesta: \pause

    \begin{itemize}[<+->]
        \item ¿Cuál fue la calificación para la Tarea 2?
        \item ¿Cuál fue el promedio del alumno?
        \item ¿Cuál es la tarea en la que mejor le fue?
    \end{itemize}
\end{frame}

\begin{frame}{Arreglos}{Estructuras matemáticas}
    ¿La pregunta ahora es\dots realmente necesito \textbf{3 variables} para guardar \textbf{3 datos}?
    Podemos \textit{arreglar} los datos de tal manera que \textbf{su posición} nos aporte algo más: \pause

    $$\mathbf{t} = \langle 90, 75, 87 \rangle$$ \pause

    La \textbf{posición} en esta \textit{lista ordenada} nos indica qué número de tarea fue, y el valor que haya en dicha posición guarda la calificación. Por lo mismo, podemos usar ``una sola variable'' para guardar de manera ordenada la información requerida, y referirnos sólo a la posición deseada:

    $$\mathbf{t}_2 = 75$$
\end{frame}

\begin{frame}{Vectores}{Estructuras matemáticas}
    Cuando \textit{arreglamos} los datos para usarlos de manera lineal, estamos creando una \textbf{lista} o \alert{vector}. \pause

    $$\mathbf{x} = \langle 1, 2, 10, 23, 2, -1, 70, 15\rangle$$

    \begin{itemize}[<+->]
        \item Usamos negritas para denotar la diferencia entre la variable $x$ que guarda un valor, y la variable $\mathbf{x}$ que guarda \textbf{múltiples} valores
        \item Usamos \textit{angle brackets} (\textit{cuñas} les dicen en español) para delimitar la lista de sus valores
        \item A diferencia de un conjunto, el orden de los valores \textbf{sí importa}
    \end{itemize}

\end{frame}

\begin{frame}{Matrices}{Estructuras matemáticas}
    Supongamos que ahora necesito saber las calificaciones de las tres tareas, pero ahora de \textbf{varios alumnos}.

    Esto significa que ahora necesitamos \textbf{varios vectores}, pero en su lugar podemos \textit{arreglar} los datos como \textbf{una lista de listas}: \pause

    \begin{columns}
        \begin{column}{0.5\textwidth}
            \begin{align*}
                & alumno_1 \\
                & alumno_2 \\
                & alumno_3 \\
                & alumno_4
            \end{align*}
        \end{column}

        \begin{column}{0.5\textwidth}
            \begin{align*}
                \begin{bmatrix}
                    90 & 75 & 87 \\
                    100 & 100 & 95 \\
                    90 & 70 & 88 \\
                    85 & 65 & 50
                \end{bmatrix}
            \end{align*}
        \end{column}
    \end{columns}
\end{frame}

\begin{frame}{Matrices}{Estructuras matemáticas}
    $$A = 
    \begin{bmatrix}
        90 & 75 & 87 \\
        \alert<3>{100} & \alert<3>{100} & \alert<3>{95} \\
        90 & \alert<4>{70} & 88 \\
        85 & 65 & 50
    \end{bmatrix}
    $$

    \bigskip

    \begin{itemize}[<+->]
        \item Una \alert{matriz} es una lista de listas y solemos usar mayúsculas para los nombres de variables
        \item En este caso, $A$ tiene 4 filas y 3 columnas, es decir que es de $4 \times 3$
        \item El elemento $\alert<3>{A_2}$ es $\langle 100, 100, 95 \rangle$
        \item El elemento $\alert<4>{A_{3,2}}$ es $70$
    \end{itemize}

\end{frame}

\section{Operaciones vectorizadas}

\begin{frame}[fragile]{De uno en uno\dots}{Operaciones vectorizadas}
    Los \textbf{arreglos} (ya sean vectores o matrices) tienen, por sí solos, una especie de orden.
    Este orden da pie a pensar en una \textbf{secuencia}, y entonces operar usando \textbf{ciclos} es \textit{natural}:

    \bigskip

    \begin{columns}
        \begin{column}{0.95\linewidth}
            \lstinputlisting{onebyone.m}
        \end{column}
    \end{columns}

    \bigskip

    ¿Cuál es el resultado de \texttt{r}?
\end{frame}


\begin{frame}[fragile]{\dots o todos a la vez}{Operaciones vectorizadas}
    Sin embargo, existen ciertas operaciones que están pensadas para operar directamente sobre \textbf{vectores}, y funcionan justo como lo esperaríamos:

    \bigskip

    \begin{columns}
        \begin{column}{0.95\linewidth}
            \begin{lstlisting}[title={Command Window}, numbers=none]
>> x = [1 2 3 4 5];
>> x + 10
ans =
   11  12  13  14  15
            \end{lstlisting}
        \end{column}
    \end{columns}

    \bigskip

    Estas operaciones son conocidas como \alert{operaciones vectorizadas}, y trabajan con \textbf{cada uno} de los valores al mismo tiempo, en lugar de uno por uno.
\end{frame}


% What is control flow
% why is it important
% does it exist in math?
% how to represent it
% how to represent it in matlab
% practical cases

% \section*{Referencias}

% \begin{frame}[t]{Referencias}
    % \nocite{bibID01}
    % \nocite{bibID02}

    % \bibliographystyle{IEEE}
    % \bibliography{biblio}
% \end{frame}

\end{document}