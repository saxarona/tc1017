\documentclass[spanish]{beamer}

\usepackage{hyperref}
\usepackage{color}
\usepackage{ragged2e}
\usepackage{mathrsfs}
\usepackage[T1]{fontenc}
\usepackage[utf8]{inputenc}
\usepackage{listings}

\usepackage{csquotes}

\usepackage[framed]{matlab-prettifier}

\newcommand{\matlab}[1]{\lstinline[style=Matlab-pyglike]!#1!}

\lstset{ %
    language = Matlab,
    style = Matlab-editor,
    basicstyle = \mlttfamily\scriptsize,
}

% \usepackage{tikz}

% \usetikzlibrary{fit, shapes, arrows}

% \usepackage{courier}
% \usepackage{subfigure}
% \usepackage{enumerate}
% \usepackage{algorithmic}
% \usepackage{algorithm}

% \usepackage{listings}
% \usepackage{lstlinebgrd}

\usetheme{Boadilla}
\usefonttheme[onlymath]{serif}

\newcommand\blfootnote[1]{%
\begingroup
\renewcommand\thefootnote{}\footnote{#1}%
\addtocounter{footnote}{-1}%
\endgroup
}

% Sets the templates
\definecolor{navyblue}{RGB}{0, 0, 128}
\definecolor{crimson}{RGB}{128, 16, 0}

\setbeamertemplate{navigation symbols}{} 
\setbeamertemplate{headline}{}
\setbeamertemplate{footline}[frame number]
\setbeamertemplate{bibliography item}[text]
\setbeamertemplate{theorems}[numbered]

\setbeamercolor{title}{fg=navyblue, bg=white}
\setbeamercolor{frametitle}{fg=navyblue, bg=white}
\setbeamercolor{structure}{fg=navyblue}
\setbeamercolor{button}{fg=white,bg=navyblue}

\setbeamercovered{transparent}

\title{Cálculos matemáticos y fórmulas}
\subtitle{Solución de Problemas con Programación \\ (TC1017)}
\author{
    \texorpdfstring{
        \begin{center}
            M.C. Xavier Sánchez Díaz \\
            \href{mailto:sax@tec.mx}{\texttt{sax@tec.mx}}
        \end{center}
    }
    {M.C. Xavier Sánchez Díaz}
}

\institute[Tecnológico de Monterrey]{\includegraphics[scale=0.5]{../img/logo}}
\date{}

\begin{document}

\setlength{\rightskip}{0pt}

\begin{frame}[plain]
    \titlepage        
\end{frame}

\begin{frame}{Outline}
    \tableofcontents
\end{frame}

\section{Lo básico}

\begin{frame}[fragile]{Command Window como calculadora}{Lo básico}
    
    Puedes hacer operaciones aritméticas en la ventana de comandos:

    \bigskip

    \begin{lstlisting}
        >> 2 + 3
    ans =
            5
    \end{lstlisting} \pause

    \begin{lstlisting}
    >> 2 + 3 * 5 / 8 + 16 / 2.5 * 10^3
    ans =
            6.4039e+03
    \end{lstlisting} \pause

    \begin{lstlisting}
    >> mod(20,7)
    ans =
            6
    \end{lstlisting}
\end{frame}

\begin{frame}[fragile]{Pidiendo ayuda}{Lo básico}
    Para cualquier duda, uno siempre debe pedir ayuda. Esto se logra con la función \matlab{help}:
    \pause

    \bigskip

    \begin{lstlisting}
    >> help mod
    mod    Modulus after division.
        mod(x,y) returns x - floor(x./y)
        
        The statement "x and y are con

        By convention:
            mod(x,0) is x.
            mod(x,x) is 0.
            mod(x,y)
        See also rem
    \end{lstlisting}
\end{frame}

\begin{frame}{Holi}{Lo básico}
    Boli
\end{frame}


% \section*{Referencias}

% \begin{frame}[t]{Referencias}
    % \nocite{bibID01}
    % \nocite{bibID02}

    % \bibliographystyle{IEEE}
    % \bibliography{biblio}
% \end{frame}

\end{document}