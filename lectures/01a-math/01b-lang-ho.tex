\documentclass[spanish, handout]{beamer}

\usepackage{hyperref}
\usepackage{color}
\usepackage{ragged2e}
\usepackage{mathrsfs}
\usepackage[T1]{fontenc}
\usepackage[utf8]{inputenc}
\usepackage{listings}

\usepackage{csquotes}

\usepackage[framed]{matlab-prettifier}

\newcommand{\matlab}[1]{\lstinline[style=Matlab-pyglike]!#1!}

\lstset{ %
    language = Matlab,
    style = Matlab-editor,
    basicstyle = \mlttfamily\scriptsize,
}

% \usepackage{tikz}

% \usetikzlibrary{fit, shapes, arrows}

% \usepackage{courier}
% \usepackage{subfigure}
% \usepackage{enumerate}
% \usepackage{algorithmic}
% \usepackage{algorithm}

% \usepackage{listings}
% \usepackage{lstlinebgrd}

\usetheme{Boadilla}
\usefonttheme[onlymath]{serif}

\newcommand\blfootnote[1]{%
\begingroup
\renewcommand\thefootnote{}\footnote{#1}%
\addtocounter{footnote}{-1}%
\endgroup
}

% Sets the templates
\definecolor{navyblue}{RGB}{0, 0, 128}
\definecolor{crimson}{RGB}{128, 16, 0}

\setbeamertemplate{navigation symbols}{} 
\setbeamertemplate{headline}{}
\setbeamertemplate{footline}[frame number]
\setbeamertemplate{bibliography item}[text]
\setbeamertemplate{theorems}[numbered]

\setbeamercolor{title}{fg=navyblue, bg=white}
\setbeamercolor{frametitle}{fg=navyblue, bg=white}
\setbeamercolor{structure}{fg=navyblue}
\setbeamercolor{button}{fg=white,bg=navyblue}

\setbeamercovered{transparent}

\title{Cálculos matemáticos y fórmulas}
\subtitle{Solución de Problemas con Programación \\ (TC1017)}
\author{
    \texorpdfstring{
        \begin{center}
            M.C. Xavier Sánchez Díaz \\
            \href{mailto:sax@tec.mx}{\texttt{sax@tec.mx}}
        \end{center}
    }
    {M.C. Xavier Sánchez Díaz}
}

\institute[Tecnológico de Monterrey]{\includegraphics[scale=0.5]{../img/logo}}
\date{}

\begin{document}

\setlength{\rightskip}{0pt}

\begin{frame}[plain]
    \titlepage        
\end{frame}

\begin{frame}{Outline}
    \tableofcontents
\end{frame}

\section{Lo básico}

\begin{frame}[fragile]{Command Window como calculadora}{Lo básico}
    
    Puedes hacer operaciones aritméticas en la ventana de comandos:

    \bigskip

    \begin{lstlisting}
        >> 2 + 3
    ans =
            5
    \end{lstlisting} \pause

    \begin{lstlisting}
    >> 2 + 3 * 5 / 8 + 16 / 2.5 * 10^3
    ans =
            6.4039e+03
    \end{lstlisting} \pause

    \begin{lstlisting}
    >> mod(20,7)
    ans =
            6
    \end{lstlisting}
\end{frame}

\begin{frame}[fragile]{Pidiendo ayuda}{Lo básico}
    En caso de cualquier duda, uno siempre debe pedir ayuda.
    Esto se logra con el comando \matlab{help}.
    Prueba a usar \matlab{help} en la command window, así:
    
    \pause

    \bigskip

    \begin{lstlisting}
        help mod
    \end{lstlisting} \pause

    \bigskip

    \begin{lstlisting}
        help multiply
    \end{lstlisting} \pause

    \bigskip

    \begin{lstlisting}
        help format
    \end{lstlisting}
    
\end{frame}

\section{Tipos de datos}

\begin{frame}{¿Datos?}{Tipos de datos}

    \blfootnote{\textit{Definition of data}, de Lexico: \url{https://www.lexico.com/en/definition/data}}

    ¿Qué es un dato? \pause

    \bigskip
    
    \begin{block}{Data}
        \begin{displayquote}
            The \alert<3->{quantities}, \alert<3->{characters}, or \alert<3->{symbols} on which \alert<3->{operations} are performed by a computer, which may be \textbf<4->{stored} and transmitted in the form of electrical signals and \textbf<4->{recorded} on magnetic, optical, or mechanical recording media.
        \end{displayquote}
    \end{block} \pause

\end{frame}

\begin{frame}{¿Datos?}{Tipos de datos}

    ¿Qué significa \textit{tipo de dato}? \pause

    \blfootnote{\textit{Definition of data type}, de Lexico: \url{https://www.lexico.com/en/definition/data\_type}}

    \begin{block}{Data type}
        \begin{displayquote}
            A particular kind of data item, as defined by \alert<3->{the values it can take}, the programming language used, or \alert<3->{the operations that can be performed on it}.
        \end{displayquote}
    \end{block} \pause
\end{frame}

\begin{frame}{¿Qué puedo guardar como dato?}{Tipos de datos}

    \begin{columns}
        \begin{column}{0.5\textwidth}
            \begin{itemize}[<+->]
                \itemsep2.5ex
                \item \alert<4>{Cantidades}
                \item \alert<5>{Caracteres}
                \item \alert<6>{Símbolos}
            \end{itemize}
        \end{column}
        \begin{column}{0.5\textwidth}
            \only<4>{
                $$1672$$
                $$728.354$$
                $$0.123486578$$
            }

            \only<5>{
                $$\text{``hola pueblo''}$$
                $$\text{265-ca32}$$
                $$\text{Á'qé 'dásç'd}$$
            }

            \only<6>{
                \large
                ¿Símbolos?
            }
        \end{column}
    \end{columns}
\end{frame}

\begin{frame}{\textit{Símbolos} o \textit{Variables}}{Tipos de datos}
Un \alert{símbolo} (también conocido como \alert{variable}) es un espacio \textit{específico} en la memoria de la computadora que almacena un dato. \pause

\bigskip

Para no tener que especificar exactamente qué celda de memoria dentro de la computadora es a la que hacemos referencia, usualmente le ponemos un nombre o \textit{alias}, para referirnos a ella cada vez que lo necesitemos. \pause

\bigskip

Los nombres de las variables en MATLAB tienen que comenzar con una letra. No pueden llevar acentos, ni tildes, ni espacios en blanco ni símbolos ortográficos.
\end{frame}


\begin{frame}{¿Qué puedo guardar en una variable?}{Tipos de datos}

    \begin{exampleblock}{Números enteros (\textit{int})}
        $$1, 2, 200, 1230984, \dots$$
    \end{exampleblock} \pause

    \begin{block}{Caracteres y cadenas (\textit{char} and \textit{string})}
        $$\text{\texttt{hola pueblo}, \texttt{c}, \texttt{c212-A\_2}, \texttt{12345}}$$
    \end{block} \pause

    \begin{alertblock}{Números de punto flotante (\textit{float})}
        $$1236548.12387, 0.126579, 1.239231e5, 1.414213562, \dots$$
    \end{alertblock}

\end{frame}

% \section*{Referencias}

% \begin{frame}[t]{Referencias}
    % \nocite{bibID01}
    % \nocite{bibID02}

    % \bibliographystyle{IEEE}
    % \bibliography{biblio}
% \end{frame}

\end{document}