\documentclass[spanish, handout]{beamer}

\usepackage[utf8]{inputenc}
%\usepackage[spanish, mexico]{babel}
\usepackage{hyperref}
\usepackage{xcolor}
\usepackage{color}
\usepackage{ragged2e}
\usepackage{mathrsfs}
\usepackage{csquotes}

% \usepackage{tikz}

% \usetikzlibrary{fit, shapes, arrows}

% \usepackage{courier}
% \usepackage{subfigure}
% \usepackage{enumerate}
% \usepackage{algorithmic}
% \usepackage{algorithm}

% \usepackage{listings}
% \usepackage{lstlinebgrd}

\usetheme{Boadilla}
\usefonttheme[onlymath]{serif}

\newcommand\blfootnote[1]{%
\begingroup
\renewcommand\thefootnote{}\footnote{#1}%
\addtocounter{footnote}{-1}%
\endgroup
}

% Sets the templates
\definecolor{navyblue}{RGB}{0, 0, 128}
\definecolor{crimson}{RGB}{128, 16, 0}

\setbeamertemplate{navigation symbols}{} 
\setbeamertemplate{headline}{}
\setbeamertemplate{footline}[frame number]
\setbeamertemplate{bibliography item}[text]
\setbeamertemplate{theorems}[numbered]

\setbeamercolor{title}{fg=navyblue, bg=white}
\setbeamercolor{frametitle}{fg=navyblue, bg=white}
\setbeamercolor{structure}{fg=navyblue}
\setbeamercolor{button}{fg=white,bg=navyblue}

\setbeamercovered{transparent}

\title{Conceptos matemáticos preliminares}
\subtitle{Solución de Problemas con Programación \\ (TC1017)}
\author{
    \texorpdfstring{
        \begin{center}
            M.C. Xavier Sánchez Díaz \\
            \href{mailto:mail@tec.mx}{\texttt{mail@tec.mx}}
        \end{center}
    }
    {M.C. Xavier Sánchez Díaz}
}

\institute[Tecnológico de Monterrey]{\includegraphics[scale=0.5]{../img/logo}}
\date{}

\begin{document}

\setlength{\rightskip}{0pt}

\begin{frame}[plain]
    \titlepage        
\end{frame}

\begin{frame}{Outline}
    \tableofcontents
\end{frame}

\section{¿Qué estudiamos?}

\begin{frame}{¿Qué estudia un ingeniero?}{¿Qué estudiamos?}
    La ingeniería se enfoca, en esencia, en la resolución de \alert{problemas}. \pause
    
    \bigskip

    \begin{itemize}
        \item<2-> ¿Qué es un problema?
        \item<3-> ¿Qué tipos de problemas existen?
        \item<4-> ¿Qué problemas podrían presentarse en tu área de trabajo?
    \end{itemize}

    \bigskip

    \onslide<5->{
        \begin{center}
            \Large ¿Qué necesito para poder resolverlos?
        \end{center}
    }
\end{frame}

\begin{frame}{Nuestras herramientas}{¿Qué estudiamos?}
    Para resolver los problemas del día a día, necesitamos de un conjunto de abstracciones que nos permita hacer tareas de todo tipo: \pause

    \bigskip

    \begin{columns}
        \begin{column}{0.33\textwidth}
            \begin{center}
                \begin{itemize}
                    \item Ordenar
                    \item Clasificar
                    \item Agrupar
                    \item Repetir
                \end{itemize}
            \end{center}
        \end{column}
        \begin{column}{0.33\textwidth}
            \begin{center}
                \begin{itemize}
                    \item Decidir
                    \item Medir
                    \item Visualizar
                    \item Interpretar
                \end{itemize}
            \end{center}
        \end{column}
        \begin{column}{0.33\textwidth}
            \begin{center}
                \begin{itemize}
                    \item Predecir
                    \item Controlar
                    \item Optimizar
                    \item Calcular
                \end{itemize}                
            \end{center}
        \end{column}
    \end{columns} \pause
    
    \bigskip

    \begin{center}
        \Large
        Las \alert{matemáticas} nos permiten hacer todo eso y más.
    \end{center}

\end{frame}

\begin{frame}[plain]
    \begin{center}
        \includegraphics[width=0.97\textwidth]{../img/purity.png}
    \blfootnote{\textit{Purity}, de xkcd: \url{https://xkcd.com/435/}}
    \end{center}
\end{frame}

\begin{frame}{Los sabores de las matemáticas}{¿Qué estudiamos?}
    
    Existen muchas áreas de estudio dentro de las matemáticas: \pause

    \bigskip

    \begin{columns}
        \begin{column}{0.5\textwidth}
            \begin{itemize}
                \item<2-> \alert<7>{Aritmética}
                \item<3-> \alert<8>{Álgebra}
                \item<4-> \alert<9>{Probabilidad}
                \item<5-> \alert<10>{Estadística}
            \end{itemize}
        \end{column}
        \begin{column}{0.5\textwidth}
            \only<7>{
                $2 + 6 / 3 - 12 + 70 = 62$
            }
            \only<8>{
                $2x + 3y = z^2$
            }
            \only<9>{
                $P(B|A) = \frac{P(A|B)P(B)}{P(A)}$
            }
            \only<10>{
                $\mathbb{E}[X] = \bar{x}$
            }
        \end{column}
    \end{columns}

    \bigskip
    
    \onslide<6->{Y otras más como lógica, cálculo, combinatoria, teoría de números, teoría de grupos, geometría, topología...}
\end{frame}

\begin{frame}{¿Y esto cómo lo aplicamos?}{¿Qué estudiamos?}
    Las \alert{ciencias computacionales} (que estudian el cómputo) se encargan de crear un puente entre los fundamentos matemáticos y las aplicaciones al mundo real: el \textbf{cómo}. \pause

    \bigskip

    Tener una noción de sus herramientas y conceptos ayudará a que seamos más eficientes en nuestras respectivas áreas del conocimiento. \pause

    \bigskip

    \begin{center}
        \Large
        ¿Qué es lo que más te llama la atención del índice analítico?
    \end{center}
\end{frame}

\section{Fundamentos de aritmética}

\begin{frame}{¿Aritmética?}{Fundamentos de aritmética}
    
    \blfootnote{Sherman, Lynda y Weisstein, Eric W. ``Arithmetic.'' De MathWorld--A Wolfram Web Resource: \url{http://mathworld.wolfram.com/Arithmetic.html}}

    \begin{block}{Definición}
        \begin{displayquote}
            Arithmetic is the branch of mathematics dealing with \textbf<2->{integers}, or more generally, \alert<2->{numerical computation}.
            Arithmetical operations include \textbf<2->{addition}, \textbf<2->{congruence} calculation, \textbf<2->{division}, factorization, \textbf<2->{multiplication}, \textbf<2->{power} computation, \textbf<2->{root} extraction, and \textbf<2->{subtraction}.
        \end{displayquote}
    \end{block} \pause
    
    \bigskip

    \begin{center}
        \Large
        ¿Qué significan estos conceptos clave?
    \end{center}
\end{frame}

\begin{frame}{Conceptos clave}{Fundamentos de aritmética}
    \begin{itemize}
        \item<1-> {\color<8->{magenta} \textbf{Adición} (Suma): \hfill $2 + 5 = 5 + 2 = 7$}
        \item<2-> {\color<8->{magenta} \textbf{Sustracción} (Resta): \hfill $10 - 12 = -12 + 10 = -2$}
        \item<3-> {\color<9->{blue} \textbf{Producto} (Multiplicación): \hfill $12 \times 5 = 5 * 12 = 12 \cdot 5 = (5)(4)(3) = 60$}
        \item<4-> {\color<9->{blue} \textbf{División}: \hfill $60 / 15 =  \frac{60}{15} = \frac{(5)(4)(3)}{(5)(3)} = \frac{4}{1}$}
        \item<5-> {\color<10->{orange} \textbf{Potencia}: \hfill $2^{10} = 2^5 \cdot 2^5 = 2 \cdot 2 \cdot 2 \cdot 2^7 = 1024$}
        \item<6-> {\color<10->{orange} \textbf{Raíz}: \hfill $\sqrt{100} = 100^{\frac{1}{2}} = \sqrt{2 \cdot 2 \cdot 5 \cdot 5} = \sqrt{2} \sqrt{2} \sqrt{5} \sqrt{5} = 10$}
        \item<7-> \textbf{Congruencia} (Módulo): \hfill $10 \mod 2 = 0, 15 \mod 4 = 3, 19 \mod 3 = 1$
    \end{itemize}
\end{frame}

\begin{frame}{Jerarquía de operaciones}{Fundamentos de aritmética}
    Es importante saber que no todas las operaciones tienen el mismo \textit{peso}.
    Algunas operaciones se hacen antes, y otras se hacen después. Por ejemplo: \pause

    \only<2>{$$2 \times 3 + 5 \neq 2 \times (3+5)$$} \pause

    \only<3->{
        \bigskip
        
        $2 \times 3 + 5 = 6 + 5 = 11$ \pause \hfill $2 \times (3+5) = 2 \times 8 = 16$
        
        \bigskip
        \pause}

    Claramente hay una diferencia en el resultado. La mejor manera de recordar qué va primero es separar por \alert{términos}. \pause Un \alert{término} es una \textit{parte} de una operación. En una \textbf{expresión}, los términos se separan usando los operadores de \textbf{\color{magenta} suma o resta}:

    $$x^2 {~\color{magenta}+~} 2x {~\color{magenta}+~} 1 = 3x^3 {~\color{magenta}+~} 2x^2 {~\color{magenta}+~} 7x {~\color{magenta}-~} 10$$ \pause

    \begin{center}
        \Large
        ¿Cuántos términos hay de cada lado de la ecuación?
    \end{center}
    
\end{frame}

\begin{frame}{Jerarquía de operaciones}{Fundamentos de aritmética}
    Al separar por \alert{términos}, podemos ver la nueva expresión como una \textbf{\color{magenta} suma o resta} de un montón de operaciones más pequeñas que podemos \textit{resolver} individualmente: \pause

    $$\alert<3>{x^2} + \alert<4>{2x} + \alert<5>{1} = \alert<6>{3x^3} + \alert<7>{2x^2} + \alert<8>{7x} - \alert<9>{10}$$ \pause

    \begin{columns}
        \begin{column}{0.5\textwidth}
            \begin{itemize}
                \item<3-> $x^2$
                \item<4-> $2x$
                \item<5-> $1$
            \end{itemize}
        \end{column}
        \begin{column}{0.5\textwidth}
            \begin{itemize}
                \item<6-> $\mathbf{3x^3}$
                \item<7-> $\mathbf{2x^2}$
                \item<8-> $7x$
                \item<9-> $-10$
            \end{itemize}
        \end{column}
    \end{columns}
\end{frame}

\begin{frame}{Jerarquía de operaciones}{Fundamentos de aritmética}
    ``El exponente y las raíces van primero''---suelen decir---pero realmente todo va a la vez: \pause

    \bigskip
    
    \begin{center}
        $3x^3 = 3 \cdot x \cdot x \cdot x$ \pause \hfill $2x^2 = 2 \cdot x \cdot x$
    \end{center} \pause
    
    Finalmente, agrupamos todos los términos \textbf{\color{magenta} sumando o restando} (según sea el caso). \pause

    \bigskip

    \begin{center}
        \Large
        ¿Tenemos que resolver las operaciones manualmente?
    \end{center}
\end{frame}

\section{Conceptos básicos de programación}

\begin{frame}{¿Programación?}{Conceptos básicos de programación}
    \begin{displayquote}
        Falling in love with code means falling in love with problem solving and being a part of a forever ongoing conversation.
    \end{displayquote} \hfill Kathryn Barrett, O'Reily Media. \pause

    \bigskip

    Un lenguaje de programación es una herramienta computacional para resolver problemas. \pause

    \bigskip

    \begin{center}
        \Large
        ¿Cuáles de los problemas que mencionamos antes se pueden resolver con un lenguaje de programación?
    \end{center}
\end{frame}

\begin{frame}{Términos comunes}{Conceptos básicos de programación}
    \begin{columns}
        \begin{column}{0.33\textwidth}
            \begin{itemize}
                \item Estatuto (\textit{Statement})
                \item Función (\textit{Function})
                \item Rutina (\textit{Routine})
                \item Procedimiento (\textit{Procedure})
            \end{itemize}
        \end{column}
        \begin{column}{0.33\textwidth}
            \begin{itemize}
                \item Parámetro (\textit{Parameter})
                \item Argumento (\textit{Argument})
                \item Tipo (\textit{Type})
                \item Retorno (\textit{Return})
            \end{itemize}
        \end{column}
        \begin{column}{0.3\textwidth}
            \begin{itemize}
                \item Estructura (\textit{Structure})
                \item Arreglo (\textit{Array})
                \item Comentario (\textit{Comment})
                \item Bucle (\textit{Loop})
            \end{itemize}
        \end{column}
    \end{columns}
    \begin{columns}
        \begin{column}{0.33\textwidth}
            \begin{itemize}
                \item Condicional (\textit{Conditional})
                \item Palabra reservada (\textit{Reserved keyword})
                \item Archivo (\textit{File})
                \item Directorio (\textit{Directory})
            \end{itemize}
        \end{column}
        \begin{column}{0.33\textwidth}
            \begin{itemize}
                \item Gráfico (\textit{Plot})
                \item Operador (\textit{Operator})
                \item Iteración (\textit{Iteration})
                \item Variable (\textit{Variable})
            \end{itemize}
        \end{column}
        \begin{column}{0.33\textwidth}
            \begin{itemize}
                \item Entero (\textit{Integer})
                \item Punto Flotante (\textit{Floating-point})
                \item Cadena de caracteres (\textit{String})
                \item Imprimir (\textit{Print})
            \end{itemize}
        \end{column}
    \end{columns}
\end{frame}

% What we study
% % Arithmetic
% % Algebra
% % Stats and Probability
% % Other interesting things like Logic, Calculus, Number Theory, Geometry, Topology

% Arithmetic basics

% \section*{Referencias}

% \begin{frame}[t]{Referencias}
    % \nocite{bibID01}
    % \nocite{bibID02}

    % \bibliographystyle{IEEE}
    % \bibliography{biblio}
% \end{frame}

\end{document}