\documentclass[11pt, onside]{article}
%\usepackage[a4paper]{geometry}
\usepackage{fullpage}
\usepackage[utf8]{inputenc}
\usepackage[spanish, mexico]{babel}
\usepackage{lipsum}
\usepackage{bm}
\usepackage{upgreek}
\usepackage{enumitem}
\usepackage{mathrsfs}
\usepackage{amsmath}
\usepackage{amssymb}
\usepackage{tikz}
\usepackage{tcolorbox}
\usepackage{csquotes}
\usepackage{listings}
\usepackage{booktabs}
%\usepackage{helvet}
\usepackage[numbered]{matlab-prettifier}


% mathtools for: Aboxed (put box on last equation in align envirenment)
\usepackage{microtype} %improves the spacing between words and letters

%% COLOR DEFINITIONS

\usepackage{xcolor} % Enabling mixing colors and color's call by 'svgnames'

\definecolor{MyColor1}{rgb}{0.2,0.4,0.6} %mix personal color
\newcommand{\textb}{\color{Black} \usefont{OT1}{lmss}{m}{n}}
\newcommand{\blue}{\color{MyColor1} \usefont{OT1}{lmss}{m}{n}}
\newcommand{\blueb}{\color{MyColor1} \usefont{OT1}{lmss}{b}{n}}
\newcommand{\red}{\color{LightCoral} \usefont{OT1}{lmss}{m}{n}}
\newcommand{\green}{\color{Turquoise} \usefont{OT1}{lmss}{m}{n}}

\DeclareMathOperator{\trace}{trace}
\DeclareMathOperator{\diag}{diag}

%% FONTS AND COLORS

%    SECTIONS

\usepackage{titlesec}
\usepackage{sectsty}
%%%%%%%%%%%%%%%%%%%%%%%%
%set section/subsections HEADINGS font and color
%\sectionfont{\color{Black}}  % sets colour of sections
%\subsectionfont{\color{Black}}  % sets colour of sections

%set section enumerator to arabic number (see footnotes markings alternatives)
\renewcommand\thesection{\arabic{section}} %define sections numbering
\renewcommand\thesubsection{\thesection\arabic{subsection}} %subsec.num.

%define new section style
\newcommand{\mysection}{
\titleformat{\section} [runin] {\usefont{OT1}{lmss}{b}{n}\color{MyColor1}} 
{\thesection} {3pt} {} } 


% %	CAPTIONS
% \usepackage{caption}
% \usepackage{subcaption}
% %%%%%%%%%%%%%%%%%%%%%%%%
% \captionsetup[figure]{labelfont={color=Turquoise}}


%		!!!EQUATION (ARRAY) --> USING ALIGN INSTEAD
%using amsmath package to redefine eq. numeration (1.1, 1.2, ...) 
\renewcommand{\theequation}{\thesection\arabic{equation}}

\setlength\parindent{0pt}




\makeatletter
\let\reftagform@=\tagform@
\def\tagform@#1{\maketag@@@{(\ignorespaces\textcolor{red}{#1}\unskip\@@italiccorr)}}
\renewcommand{\eqref}[1]{\textup{\reftagform@{\ref{#1}}}}
\makeatother
% \usepackage{hyperref}
% \hypersetup{colorlinks=true}

% For labeling top of page on every page but first one:
%\usepackage{fancyhdr}

\newcommand{\myclass}{TC1017 -- Solución de Problemas con Programación} % Class name?
\newcommand{\mytitle}{Examen 2} % Title of document?
\newcommand{\mydate}{25.03.19} % The date?
\newcommand{\myheader}{
    \begin{flushleft}
        \large
        Name: \rule{13 cm}{0.4mm} \\
        Student ID: \rule{5 cm}{0.4mm} \hfill Date: \mydate
    \end{flushleft}
}

\newcommand{\matlab}[1]{\lstinline[style=Matlab-editor]!#1!}
\newcommand{\shortresponserule}{{\large\rule{5 cm}{0.3mm}}}

\title{
    \myclass \\
    \textbf{\mytitle} \\
    \myheader
    \date{}
}

% You can set the date automatically by replacing "date goes here" with "\today"

% \renewcommand{\rmdefault}{phv} % Arial Font
\renewcommand{\familydefault}{\sfdefault}

% \pagestyle{fancy}
% \fancyhead{}
% \fancyhead[CO,CE]{{\small{{\bf{\mytitle}} -- \myclass}}}

\begin{document}
\maketitle

\vspace{-1.5cm}

Read carefully before answering the questions.
This exam is meant to be solved individually.

When answering, try to be as explicit as you can: I will grade based on what it's written.
Remember that you can check class materials, the book or your lecture notes.
Be mindful of your time.
Good luck!

\vspace{1.5ex}

The type of your exam will depend on your Student ID:

$$ student = 
\begin{cases}
   Minara \, Anemina & = [1, 1, 1, 0.2, 0.5] \text{ if the last digit of your student ID } \mathtt{mod} \, 3 = 0 \\
   Igna \, Phoenix & = [0.2, 2, 0.5, 1, 1] \text{ if the last digit of your student ID } \mathtt{mod} \, 3 = 1 \\
   Ultima \, Thule & = [2, 0.2, 0.5, 2, 1] \text{ if the last digit of your student ID } \mathtt{mod} \, 3 = 2 \\
\end{cases}
$$

\vspace{1.5ex}

My student ID is \shortresponserule,
so the last digit of my ID $\mathtt{mod}$ 3 is \rule{1cm}{0.4mm}.

Therefore, my student is \shortresponserule \quad (2 \%)

\vspace{1.5ex}

\section{First Year Spells (60\%)}

Analyse each of the following codes, which are used to calculate the cost of the magic spells that students need to learn on their first year on the magical institute.

\subsection{Fireball}

\lstinputlisting[style=Matlab-editor]{fireball.m}

\begin{itemize}
    \item \matlab{fireball(student(1))} \hfill \shortresponserule
\end{itemize}

\subsection{Freezing Pulse}

\lstinputlisting[style=Matlab-editor]{freezing_pulse.m}

\begin{itemize}
    \item \matlab{freezing_pulse(student(2))} \hfill \shortresponserule
\end{itemize}

\subsection{Arc}

\lstinputlisting[style=Matlab-editor]{arc.m}
\begin{itemize}
    \item \matlab{arc(student(3))} \hfill \shortresponserule
\end{itemize}

\subsection{Essence Drain}

\lstinputlisting[style=Matlab-editor]{essence_drain.m}
\begin{itemize}
    \item \matlab{essence_drain(student(4))} \hfill \shortresponserule
\end{itemize}

\subsection{Summon Skeleton}

\lstinputlisting[style=Matlab-editor]{skeleton.m}
\begin{itemize}
    \item \matlab{skeleton(student(5))} \hfill \shortresponserule
\end{itemize}

\pagebreak

\section{Second Year Spells (30 \%)}

All previous codes generate the mana cost of each of the magic spells that $student$ must learn during the core subjects in their first year.
However, from the second year onwards, they can choose between \textbf{one of the following courses, or both}:

\subsection{Raise Zombie}

Summoning a zombie is a very complicated task, since body parts from all over the place are needed.
Furthermore, fresh meat is always preferred.
To keep all things legal with no funny business going on, the Magical Institute has an agreement with the town's morgue,
who provide a table which looks like the following, weekly (consider the numbers will change from week to week!):

\begin{table}[htbp]
    \centering
    \begin{tabular}{@{}cccc@{}}
    \toprule
    \textbf{ID} & \textbf{Weight} & \textbf{Height} & \textbf{Cause of Death} \\ \midrule
    1 & 78 & 1.72 & Natural \\
    2 & 97 & 1.86 & Asphyxia \\
     &  & $\vdots$ &  \\ \bottomrule
    \end{tabular}
\end{table}

To determine if a body is useful for reanimation, its potential (calculated as $Weight \times Height$) must be 115 or more.

\bigskip

Write \textbf{two functions}:

\begin{enumerate}
    \item A function which receives the matrix, and returns as a result the number of available bodies for reanimation for a given week (15 \%)
    \item A function which receives the matrix, and returns the total potential of a given week (15 \%)
\end{enumerate}

\subsection{Magma Orb}

To practice their aim, students need some shooting targets which will be irremediably consumed by the flames of their spells.
For that reason, the magical institute has an agreement with the town's Material and Recycling Centre, who provide a table with the following form, weekly (consider the numbers will change from week to week!):

% Please add the following required packages to your document preamble:
% \usepackage{booktabs}
\begin{table}[htbp]
    \centering
    \begin{tabular}{@{}cccc@{}}
    \toprule
    \textbf{ID} & \textbf{Units} & \textbf{Material} & \textbf{Compressible} \\ \midrule
    1 & 2100 & Cardboard & Yes \\
    2 & 3500 & Wax & No \\
    3 & 2700 & Oil & No \\
    4 & 7000 & Paper & Yes \\ \bottomrule
    \end{tabular}
\end{table}

To craft shooting targets, 65\% of the \textbf{weekly available} cardboard is used, as well as 10\% of paper, 10\% of oil and 15\% of wax (the rest is used for candles).

Furthermore, we know that to make a shooting target, a single unit of each material is needed.

Reflect on the following question...
\textit{¿Qué limita el número de dianas que puedo hacer esta semana?}

And now, answer...
How many shooting targets can we craft this week? (10 \%) \shortresponserule

\bigskip

Write \textbf{a function} which receives a matrix like above, and returns the number of shooting targets that can be crafted in a given week(20 \%)
\textit{Hint: if you don't want to make comparisons, you can use} \matlab{min}

\section{Feedback (8 \%)}

\begin{enumerate}[label=\alph*)]
    \item Draw the student you were assigned to (1 \%)
    \item Write a name for your magical institute (1 \%)
    \item Write a brief opinion on this exam model (3 \%)
    \item If fantasy situations can be tackled with programming, do you think programming is useful for real-life scenarios? Why? (3 \%)
\end{enumerate}

\section{Challenge (+8 $\bigstar$)}

Write the Flowchart of \textbf{any} of the First Year Spells:

\vfill

\textbf{In accordance with the Tecnológico de Monterrey Student Code of Honor, my performance in this exam will be guided by academic honesty.}
\end{document}