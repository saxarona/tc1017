\documentclass[11pt, onside]{article}
%\usepackage[a4paper]{geometry}
\usepackage{fullpage}
\usepackage[utf8]{inputenc}
\usepackage[spanish, mexico]{babel}
\usepackage{lipsum}
\usepackage{bm}
\usepackage{upgreek}
\usepackage{enumitem}
\usepackage{mathrsfs}
\usepackage{amsmath}
\usepackage{amssymb}
\usepackage{tikz}
\usepackage{tcolorbox}
\usepackage{csquotes}
\usepackage{listings}
\usepackage{booktabs}
%\usepackage{helvet}
\usepackage[numbered]{matlab-prettifier}


% mathtools for: Aboxed (put box on last equation in align envirenment)
\usepackage{microtype} %improves the spacing between words and letters

%% COLOR DEFINITIONS

\usepackage{xcolor} % Enabling mixing colors and color's call by 'svgnames'

\definecolor{MyColor1}{rgb}{0.2,0.4,0.6} %mix personal color
\newcommand{\textb}{\color{Black} \usefont{OT1}{lmss}{m}{n}}
\newcommand{\blue}{\color{MyColor1} \usefont{OT1}{lmss}{m}{n}}
\newcommand{\blueb}{\color{MyColor1} \usefont{OT1}{lmss}{b}{n}}
\newcommand{\red}{\color{LightCoral} \usefont{OT1}{lmss}{m}{n}}
\newcommand{\green}{\color{Turquoise} \usefont{OT1}{lmss}{m}{n}}

\DeclareMathOperator{\trace}{trace}
\DeclareMathOperator{\diag}{diag}

%% FONTS AND COLORS

%    SECTIONS

\usepackage{titlesec}
\usepackage{sectsty}
%%%%%%%%%%%%%%%%%%%%%%%%
%set section/subsections HEADINGS font and color
%\sectionfont{\color{Black}}  % sets colour of sections
%\subsectionfont{\color{Black}}  % sets colour of sections

%set section enumerator to arabic number (see footnotes markings alternatives)
\renewcommand\thesection{\arabic{section}} %define sections numbering
\renewcommand\thesubsection{\thesection\arabic{subsection}} %subsec.num.

%define new section style
\newcommand{\mysection}{
\titleformat{\section} [runin] {\usefont{OT1}{lmss}{b}{n}\color{MyColor1}} 
{\thesection} {3pt} {} } 


% %	CAPTIONS
% \usepackage{caption}
% \usepackage{subcaption}
% %%%%%%%%%%%%%%%%%%%%%%%%
% \captionsetup[figure]{labelfont={color=Turquoise}}


%		!!!EQUATION (ARRAY) --> USING ALIGN INSTEAD
%using amsmath package to redefine eq. numeration (1.1, 1.2, ...) 
\renewcommand{\theequation}{\thesection\arabic{equation}}

\setlength\parindent{0pt}




\makeatletter
\let\reftagform@=\tagform@
\def\tagform@#1{\maketag@@@{(\ignorespaces\textcolor{red}{#1}\unskip\@@italiccorr)}}
\renewcommand{\eqref}[1]{\textup{\reftagform@{\ref{#1}}}}
\makeatother
% \usepackage{hyperref}
% \hypersetup{colorlinks=true}

% For labeling top of page on every page but first one:
%\usepackage{fancyhdr}

\newcommand{\myclass}{TC1017 -- Solución de Problemas con Programación} % Class name?
\newcommand{\mytitle}{Examen 2} % Title of document?
\newcommand{\mydate}{25.03.19} % The date?
\newcommand{\myheader}{
    \begin{flushleft}
        \large
        Nombre: \rule{13 cm}{0.4mm} \\
        Matrícula: \rule{5 cm}{0.4mm} \hfill Fecha: \mydate
    \end{flushleft}
}

\newcommand{\matlab}[1]{\lstinline[style=Matlab-editor]!#1!}
\newcommand{\shortresponserule}{{\large\rule{5 cm}{0.3mm}}}

\title{
    \myclass \\
    \textbf{\mytitle} \\
    \myheader
    \date{}
}

% You can set the date automatically by replacing "date goes here" with "\today"

% \renewcommand{\rmdefault}{phv} % Arial Font
\renewcommand{\familydefault}{\sfdefault}

% \pagestyle{fancy}
% \fancyhead{}
% \fancyhead[CO,CE]{{\small{{\bf{\mytitle}} -- \myclass}}}

\begin{document}
\maketitle

\vspace{-1.5cm}

Lee cuidadosamente y contesta lo que se te pide.
Este examen es individual.

Al momento de contestar, intenta ser lo más explícito posible: se calificará con base en lo que esté escrito. %y se considerará el proceso aún cuando la respuesta final esté errada.
Recuerda que puedes revisar material de la clase, el libro de texto o tus notas.
Administra bien tu tiempo.
Buena suerte.

\vspace{1.5ex}

El tipo de tu examen dependerá de tu matrícula.

$$ student = 
\begin{cases}
   Minara \, Anemina & = [1, 1, 1, 0.2, 0.5] \text{ si el último dígito de tu matrícula } \mathtt{mod} \, 3 = 0 \\
   Igna \, Phoenix & = [0.2, 2, 0.5, 1, 1] \text{ si el último dígito de tu matrícula } \mathtt{mod} \, 3 = 1 \\
   Ultima \, Thule & = [2, 0.2, 0.5, 2, 1] \text{ si el último dígito de tu matrícula } \mathtt{mod} \, 3 = 2 \\
\end{cases}
$$

\vspace{1.5ex}

Mi matrícula es \shortresponserule,
así que el último dígito de mi ID $\mathtt{mod}$ 3 es \rule{1cm}{0.4mm}.

Por tanto me toca usar a \shortresponserule \quad (2 \%)

\vspace{1.5ex}

\section{Hechizos de primer año (60\%)}

Analiza cada uno de los siguientes códigos que sirven para calcular el costo de los hechizos que los estudiantes deben aprender el primer año de su estancia en el instituto.

\subsection{Fireball}

\lstinputlisting[style=Matlab-editor]{fireball.m}

\begin{itemize}
    \item \matlab{fireball(student(1))} \hfill \shortresponserule
\end{itemize}

\subsection{Freezing Pulse}

\lstinputlisting[style=Matlab-editor]{freezing_pulse.m}

\begin{itemize}
    \item \matlab{freezing_pulse(student(2))} \hfill \shortresponserule
\end{itemize}

\subsection{Arc}

\lstinputlisting[style=Matlab-editor]{arc.m}
\begin{itemize}
    \item \matlab{arc(student(3))} \hfill \shortresponserule
\end{itemize}

\subsection{Essence Drain}

\lstinputlisting[style=Matlab-editor]{essence_drain.m}
\begin{itemize}
    \item \matlab{essence_drain(student(4))} \hfill \shortresponserule
\end{itemize}

\subsection{Summon Skeleton}

\lstinputlisting[style=Matlab-editor]{skeleton.m}
\begin{itemize}
    \item \matlab{skeleton(student(5))} \hfill \shortresponserule
\end{itemize}

\pagebreak

\section{Hechizos de segundo año (30 \%)}

Los códigos anteriores generan los costos de cada uno de los hechizos que $student$ debe aprender en el tronco común de su estancia en la escuela de magia.
Sin embargo, a partir del segundo año pueden escoger \textbf{uno de los siguientes dos cursos, o ambos}:

\subsection{Raise Zombie}

Reanimar a un zombie es muy complicado pues se utilizan restos de aquí y de allá,
y se necesita carne fresca.
Para tener todo el proceso dentro del marco de la legalidad, la escuela tiene un convenio con la morgue municipal, quienes proveen semanalmente una tabla como la siguiente, y cuyos números cambian semana a semana:

\begin{table}[htbp]
    \centering
    \begin{tabular}{@{}cccc@{}}
    \toprule
    \textbf{ID} & \textbf{Peso} & \textbf{Altura} & \textbf{Cause de la Muerte} \\ \midrule
    1 & 78 & 1.72 & Natural \\
    2 & 97 & 1.86 & Asfixia \\
     &  & $\vdots$ &  \\ \bottomrule
    \end{tabular}
\end{table}

Para saber si un cuerpo puede ser usado para la reanimación, se busca que su potencial (calculado como $Peso \times Altura$) sea 115 o más.

\bigskip

Escribe \textbf{dos funciones}:
\begin{enumerate}
    \item Una función que reciba la matriz y devuelva como resultado el número de cuerpos disponibles para reanimación para esa semana (15 \%)
    \item Una función que reciba la matriz y devuelva el potencial total de la semana (15 \%)
\end{enumerate}

\subsection{Magma Orb}

Para practicar su puntería, los estudiantes necesitan blancos y dianas que irremediablemente serán consumidos por las flamas de sus hechizos.
Por eso, la escuela tiene un convenio con el centro municipal de materiales y reciclaje, quienes proveen semanalmente una tabla con la siguiente forma y cuyos números cambian semana a semana:

% Please add the following required packages to your document preamble:
% \usepackage{booktabs}
\begin{table}[htbp]
    \centering
    \begin{tabular}{@{}cccc@{}}
    \toprule
    \textbf{ID} & \textbf{Unidades} & \textbf{Material} & \textbf{Comprimible} \\ \midrule
    1 & 2100 & Cartón & Sí \\
    2 & 3500 & Cera & No \\
    3 & 2700 & Aceite & No \\
    4 & 7000 & Papel & Sí \\ \bottomrule
    \end{tabular}
\end{table}

Para fabricar blancos y dianas, a la semana se utiliza el 65\% del cartón, 10\% del papel, 10\% del aceite y 15\% de la cera (el resto se utiliza para velas) que entra al instituto.

Además, se sabe que para hacer una diana es necesario una unidad de cada material.

Contesta la siguiente pregunta retórica, y finalmente calcula...
\textit{¿Qué limita el número de dianas que puedo hacer esta semana?}

¿Cuántas dianas pueden hacerse esta semana? (10 \%) \shortresponserule

\bigskip

Escribe \textbf{una función} que reciba una matriz como la de arriba, y que devuelva el número de dianas que pueden fabricarse en la semana (20 \%)
\textit{Hint: si no quieres hacer comparaciones, podrías usar} \matlab{min}

\section{Feedback (8 \%)}

\begin{enumerate}[label=\alph*)]
    \item Dibuja al estudiante que te tocó (1 \%)
    \item Ponle nombre a su instituto (1 \%)
    \item Escribe una opinión breve sobre el modelo de examen (3 \%)
    \item Si se pueden atacar situaciones fantásticas con programación, ¿crees que sirva para situaciones reales? ¿Por qué? (3 \%)
\end{enumerate}

\section{Reto (+8 $\bigstar$)}

Escribe el Diagrama de flujo de cualquiera de los hechizos de primer año:

\vfill

\textbf{De acuerdo con el Código de Ética del Tecnológico de Monterrey, mi desempeño en esta actividad estará guiado por la integridad académica.}
\end{document}