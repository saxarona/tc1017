\documentclass[11pt, onside]{article}
%\usepackage[a4paper]{geometry}
\usepackage{fullpage}
\usepackage[utf8]{inputenc}
\usepackage[spanish, mexico]{babel}
\usepackage{lipsum}
\usepackage{bm}
\usepackage{upgreek}
\usepackage{enumitem}
\usepackage{mathrsfs}
\usepackage{amsmath}
\usepackage{amssymb}
\usepackage{tikz}
\usepackage{tcolorbox}
\usepackage{csquotes}
\usepackage{listings}
%\usepackage{helvet}
\usepackage[numbered]{matlab-prettifier}


% mathtools for: Aboxed (put box on last equation in align envirenment)
\usepackage{microtype} %improves the spacing between words and letters

%% COLOR DEFINITIONS

\usepackage{xcolor} % Enabling mixing colors and color's call by 'svgnames'

\definecolor{MyColor1}{rgb}{0.2,0.4,0.6} %mix personal color
\newcommand{\textb}{\color{Black} \usefont{OT1}{lmss}{m}{n}}
\newcommand{\blue}{\color{MyColor1} \usefont{OT1}{lmss}{m}{n}}
\newcommand{\blueb}{\color{MyColor1} \usefont{OT1}{lmss}{b}{n}}
\newcommand{\red}{\color{LightCoral} \usefont{OT1}{lmss}{m}{n}}
\newcommand{\green}{\color{Turquoise} \usefont{OT1}{lmss}{m}{n}}

\DeclareMathOperator{\trace}{trace}
\DeclareMathOperator{\diag}{diag}

%% FONTS AND COLORS

%    SECTIONS

\usepackage{titlesec}
\usepackage{sectsty}
%%%%%%%%%%%%%%%%%%%%%%%%
%set section/subsections HEADINGS font and color
%\sectionfont{\color{Black}}  % sets colour of sections
%\subsectionfont{\color{Black}}  % sets colour of sections

%set section enumerator to arabic number (see footnotes markings alternatives)
\renewcommand\thesection{\arabic{section}} %define sections numbering
\renewcommand\thesubsection{\thesection\arabic{subsection}} %subsec.num.

%define new section style
\newcommand{\mysection}{
\titleformat{\section} [runin] {\usefont{OT1}{lmss}{b}{n}\color{MyColor1}} 
{\thesection} {3pt} {} } 


% %	CAPTIONS
% \usepackage{caption}
% \usepackage{subcaption}
% %%%%%%%%%%%%%%%%%%%%%%%%
% \captionsetup[figure]{labelfont={color=Turquoise}}


%		!!!EQUATION (ARRAY) --> USING ALIGN INSTEAD
%using amsmath package to redefine eq. numeration (1.1, 1.2, ...) 
\renewcommand{\theequation}{\thesection\arabic{equation}}

\setlength\parindent{0pt}




\makeatletter
\let\reftagform@=\tagform@
\def\tagform@#1{\maketag@@@{(\ignorespaces\textcolor{red}{#1}\unskip\@@italiccorr)}}
\renewcommand{\eqref}[1]{\textup{\reftagform@{\ref{#1}}}}
\makeatother
\usepackage{hyperref}
\hypersetup{colorlinks=true}

% For labeling top of page on every page but first one:
%\usepackage{fancyhdr}

\newcommand{\myclass}{TC1017 -- Solución de Problemas con Programación} % Class name?
\newcommand{\mytitle}{Examen 1} % Title of document?
\newcommand{\mydate}{18.02.19} % The date?
\newcommand{\myheader}{
    \begin{flushleft}
        \large
        Nombre: \rule{13 cm}{0.4mm} \\
        Matrícula: \rule{5 cm}{0.4mm} \hfill Fecha: \mydate
    \end{flushleft}
}

\newcommand{\matlab}[1]{\lstinline[style=Matlab-bw]!#1!}
\newcommand{\shortresponserule}{{\large\rule{5 cm}{0.3mm}}}

\title{
    \myclass \\
    \textbf{\mytitle} \\
    \myheader
    \date{}
}

% You can set the date automatically by replacing "date goes here" with "\today"

% \renewcommand{\rmdefault}{phv} % Arial Font
\renewcommand{\familydefault}{\sfdefault}

% \pagestyle{fancy}
% \fancyhead{}
% \fancyhead[CO,CE]{{\small{{\bf{\mytitle}} -- \myclass}}}

\begin{document}
\maketitle

\vspace{-1.5cm}

Lee cuidadosamente y contesta lo que se te pide.
Este examen es individual.

Al momento de contestar, intenta ser lo más explícito posible: se calificará con base en lo que esté escrito. %y se considerará el proceso aún cuando la respuesta final esté errada.
Recuerda que puedes revisar material de la clase, el libro de texto o tus notas.
Administra bien tu tiempo.
Buena suerte.

\section{Analiza el código (50\%)}

Analiza cada uno de los siguientes códigos y escribe el resultado para cada caso.

\subsection*{Ignis}

\lstinputlisting[style=Matlab-bw]{ignis.m}

\begin{itemize}
    \item \matlab{ignis(5)} \hfill \shortresponserule
    \item \matlab{ignis(20)} \hfill \shortresponserule
\end{itemize}

\subsection*{Lux}

\lstinputlisting[style=Matlab-bw]{lux.m}

\begin{itemize}
    \item \matlab{lux(5)} \hfill \shortresponserule
    \item \matlab{lux(8)} \hfill \shortresponserule
\end{itemize}

\pagebreak

\subsection*{Aer}

\lstinputlisting[style=Matlab-bw]{aer.m}
\begin{itemize}
    \item \matlab{aer(5)} \hfill \shortresponserule
    \item \matlab{aer(6)} \hfill \shortresponserule
\end{itemize}

\subsection*{Aqua}

\lstinputlisting[style=Matlab-bw]{aqua.m}
\begin{itemize}
    \item \matlab{aqua(2,2,2)} \hfill \shortresponserule
    \item \matlab{aqua(10,3,2)} \hfill \shortresponserule
\end{itemize}

\subsection*{Cognitio}

\lstinputlisting[style=Matlab-bw]{cognitio.m}
\begin{itemize}
    \item \matlab{cognitio(5)} \hfill \shortresponserule
    \item \matlab{cognitio(25)} \hfill \shortresponserule
\end{itemize}

\section{Operaciones aritméticas (30\%)}

Realiza correctamente las siguientes operaciones.

\begin{enumerate}[label=\alph*)]
    \item $2 + 3 * 7 - 2 =$ \hfill \shortresponserule
    \item $5 + 4 + 3 + 2 - 10 / 2 =$ \hfill \shortresponserule
    \item $2^3 + 8 / 2 + 4 - 30 / 5 =$ \hfill \shortresponserule
    \item $((7 * 2 + 10 / 2 + 1) / 2 + 10)^2 =$ \hfill \shortresponserule
    \item $(\sqrt{2} * 2 * \sqrt{2} + 2 / 2 + (2 + 2)^2)^2 =$ \hfill \shortresponserule
    \item $27 + 15 * (3 / 3) - (3)(3^2) - 5 * 3 + 1 =$ \hfill \shortresponserule
\end{enumerate}


\section{Diseño de solución (20 \%)}

{\small \it
Este problema es un poco más complejo de lo que hemos resuelto hasta el momento.
Se recomienda que avances lo más que puedas en el resto del examen antes de comenzarlo.
}

\bigskip

El \textit{Biologic Space Lab} (BSL) es un laboratorio espacial que alberga especies extraterrestres para fines taxonómicos.
El Sector 1 (SRX) del BSL tiene 3500 individuos en su base de datos, algunos de ellos necesitan oxígeno para sobrevivir, mientras que otros necesitan nitrógeno.
Si la base de datos contiene, para cada individuo, el gas que necesita para sobrevivir (oxígeno o nitrógeno) y su consumo de gas por hora ¿Cómo calcularías cuánto oxígeno se consume en el sector SRX por segundo?
Sugerencias para diseñar tu solución:
\begin{enumerate}[label=\alph*)]
    \item ¿Cuántas variables necesitas para guardar la información de un solo individuo? ¿Cuáles son? (4\%)
    \item ¿Qué condición debe cumplirse para que tomes en cuenta el consumo de gas de un individuo? (4\%)
    \item Haz un diagrama de flujo muy breve para ejemplificar tu procedimiento  (6\%)
    \item Describe brevemente tu procedimiento, como si platicaras cómo resolver el problema (6\%)
\end{enumerate}

\section{Reto A (+5\%)}

\begin{enumerate}[label=\alph*)]
    \item ¿Cómo se llama la función para borrar la \textit{Command Windows}? (+2 \%)
    \item Haz el diagrama de flujo para el \texttt{holapueblo.m} (Tarea 2) (+3 \%)
\end{enumerate}

\section{Reto B (+10\%)}

Escribe el programa que escribirías en MATLAB para implementar la siguiente función:

$$f(x) = %
\begin{cases}
    &x, \text{ si } x > 0 \\
    & -x, \text{ si } x < 0 \\
    &0, \text{ si } x = 0
\end{cases}
$$

\vfill

\textbf{De acuerdo con el Código de Ética del Tecnológico de Monterrey, mi desempeño en esta actividad estará guiado por la integridad académica.}
\end{document}