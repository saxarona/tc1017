\documentclass[]{article}

%These tell TeX which packages to use.
\usepackage{array,epsfig}
\usepackage{amsmath}
\usepackage{amsfonts}
\usepackage{amssymb}
\usepackage{amsxtra}
\usepackage{amsthm}
\usepackage{mathrsfs}
\usepackage{color}
\usepackage[spanish, mexico]{babel}
\usepackage[utf8]{inputenc}
\usepackage{enumitem}
\usepackage{helvet}
\usepackage[framed]{matlab-prettifier}
\newcommand{\matlab}[1]{\lstinline[style=Matlab-pyglike]!#1!}

%Here I define some theorem styles and shortcut commands for symbols I use often
\theoremstyle{definition}
\newtheorem{defn}{Definition}
\newtheorem{thm}{Theorem}
\newtheorem{cor}{Corollary}
\newtheorem*{rmk}{Remark}
\newtheorem{lem}{Lemma}
\newtheorem*{joke}{Joke}
\newtheorem{ex}{Example}
\newtheorem*{sol}{Solution}
\newtheorem{prop}{Proposition}

\renewcommand{\familydefault}{\sfdefault}

\newcommand{\lra}{\longrightarrow}
\newcommand{\ra}{\rightarrow}
\newcommand{\surj}{\twoheadrightarrow}
\newcommand{\graph}{\mathrm{graph}}
\newcommand{\bb}[1]{\mathbb{#1}}
\newcommand{\Z}{\bb{Z}}
\newcommand{\Q}{\bb{Q}}
\newcommand{\R}{\bb{R}}
\newcommand{\C}{\bb{C}}
\newcommand{\N}{\bb{N}}
\newcommand{\M}{\mathbf{M}}
\newcommand{\m}{\mathbf{m}}
\newcommand{\MM}{\mathscr{M}}
\newcommand{\HH}{\mathscr{H}}
\newcommand{\Om}{\Omega}
\newcommand{\Ho}{\in\HH(\Om)}
\newcommand{\bd}{\partial}
\newcommand{\del}{\partial}
\newcommand{\bardel}{\overline\partial}
\newcommand{\textdf}[1]{\textbf{\textsf{#1}}\index{#1}}
\newcommand{\img}{\mathrm{img}}
\newcommand{\ip}[2]{\left\langle{#1},{#2}\right\rangle}
\newcommand{\inter}[1]{\mathrm{int}{#1}}
\newcommand{\exter}[1]{\mathrm{ext}{#1}}
\newcommand{\cl}[1]{\mathrm{cl}{#1}}
\newcommand{\ds}{\displaystyle}
\newcommand{\vol}{\mathrm{vol}}
\newcommand{\cnt}{\mathrm{ct}}
\newcommand{\osc}{\mathrm{osc}}
\newcommand{\LL}{\mathbf{L}}
\newcommand{\UU}{\mathbf{U}}
\newcommand{\support}{\mathrm{support}}
\newcommand{\AND}{\;\wedge\;}
\newcommand{\OR}{\;\vee\;}
\newcommand{\Oset}{\varnothing}
\newcommand{\st}{\ni}
\newcommand{\wh}{\widehat}
\newcommand{\markthis}[1]{{\color{blue}\textbf{#1}}}

%Pagination stuff.
\setlength{\topmargin}{-.3 in}
\setlength{\oddsidemargin}{0in}
\setlength{\evensidemargin}{0in}
\setlength{\textheight}{9.in}
\setlength{\textwidth}{6.5in}
\setlength{\itemsep}{0.45in}
\setlength{\parindent}{0pt}
\pagestyle{empty}



\begin{document}

\begin{center}
{\huge Solución de Problemas con Programación (TC1017)}\\[1.5ex]
{\large \textbf{Proyecto: Análisis del cambio climático}\\[1.5ex] %You should put your name here
11.11.19} %You should write the date here.
\end{center}

\vspace{0.2 cm}

{%
\small
Este proyecto es \textbf{en parejas}, y consiste en dos partes que valen 15 \% del final cada una.
El máximo de puntos a obtener en cada una de las partes es de 90 y 100 puntos respectivamente, de un total de 200.
El 10\% restante es por la inclusión de bibliografía correctamente citada (en formato IEEE, ACM, APA o MLA), y se sugiere que sea al menos una fuente para cada una de las oraciones marcadas con \markthis{este color}.
}

\section{Proyecto A: Mediciones y creación de herramientas de análisis}

\subsection{Mediciones (30 \%)}

El primer paso consiste en obtener las mediciones necesarias para poder generar nuestro análisis.
\markthis{Investiguen las temperaturas {\large máximas y mínimas} en Monterrey para cada día de los meses de julio y agosto de 2019, 2018 y 2017}.
Se recomienda que las guarden en archivos \texttt{csv} distintos para facilitar su manejo.

\bigskip

\markthis{Repitan el mismo proceso para alguna otra ciudad: temperaturas máximas y mínimas de cada día de julio y agosto de 2019, 2018 y 2017}.
Se sugiere que la ciudad sea de importancia para ustedes, su ciudad de origen o alguna ciudad en la que les gustaría vivir o trabajar.
De nuevo, también se sugiere que las guarden en archivos \texttt{csv} separados para facilitar el acceso.

\subsection{Preguntas previas al análisis (10 \%)}

En un archivo de texto (\texttt{txt, rtf}) contesten brevemente las siguientes preguntas \textbf{para cada una de las ciudades}.
Estas preguntas guiarán la creación de su herramienta de análisis. Piensen... ¿cómo calcularíamos estos puntos? ¿Qué pasos deberíamos seguir?

\begin{enumerate}
    \item De todas las mediciones que tomaron para este lugar, ¿Cuál es la temperatura máxima y cuándo fue?
    \item En este lugar, ¿cuándo fue la temperatura mínima y de cuánto fue?
    \item En esta ciudad, ¿cuál es la temperatura promedio \textbf{para cada día, por año} de sus mediciones?
    \item Ya con las temperaturas promedio... en esta ciudad, ¿En qué año fue más alta la temperatura? ¿En qué año fue más baja?
    \item Si \textbf{por cada año} sumamos la temperatura promedio de todos los días de julio en nuestros registros en esta ciudad, ¿qué año tiene la mayor suma?
\end{enumerate}

En el mismo archivo incluyan las respuestas \textbf{para ambas ciudades}.

\subsection{Creación de la herramienta de análisis (50 \%)}

Hagan \textbf{dos scripts} de MATLAB (uno para cada ciudad) para poder contestar las preguntas de la sección anterior.
Consideren que para contestar cada pregunta, es necesario hacer al menos un cálculo, por lo que debe ser incluido en su script.
El script debe incluir también la generación de una gráfica para al menos la pregunta 3.
Recuerda que la gráfica debe ser generada por ustedes en MATLAB.

\section{Proyecto B: Análisis y presentación de resultados}

Usando las respuestas de la sección anterior, generen un reporte de resultados (\texttt{PDF}) con sus \textbf{propias palabras}y que contenga lo siguiente:

\bigskip

\begin{itemize}
    \item \textbf{Portada}: con título del reporte, fecha y datos de los autores (2 \%)
    \item \textbf{Abstract}: resumen del trabajo de máximo 200 palabras y en la misma hoja que la portada. (10 \%)
    \item \textbf{Introducción}: de qué trata el trabajo, qué puntos cubre, por qué es importante y qué hay en cada sección del documento (10 \%)
    \item \textbf{Marco Teórico}: \markthis{muy breve investigación del cambio climático y su efecto en el planeta y la sociedad}. Se recomienda que sea breve pero conciso---no citar más de 2 fuentes, ya que no es el punto principal del trabajo (8 \%)
    \item \textbf{Metodología}: qué datos consiguieron, dónde, y en qué consistió su análisis (20 \%)
    \item \textbf{Resultados}: el análisis, sus gráficas (hechas en MATLAB) y una explicación de lo encontrado en sus datos (20 \%)
    \item \textbf{Conclusiones}: resumen breve de lo expuesto sin incluir información adicional (10 \%)
    \item \textbf{Bibliografía}: las fuentes que revisaron en cualquier formato estándar---IEEE, ACM, APA o MLA (20\%)
    \item \textbf{Feedback}: breve retroalimentación sobre el trabajo final y el curso; por cada integrante del equipo (10 \%)
\end{itemize}

\vspace{2 cm}

El entregable final es \textbf{una carpeta comprimida} (\texttt{ZIP}) por equipo, que debe contener:

\begin{enumerate}
    \item El reporte que generaron (\texttt{.pdf})
    \item Los dos scripts de MATLAB (\texttt{.m}) con los que trabajaron (uno por cada ciudad)
    \item El archivo de respuestas pre-análisis (\texttt{.txt} o \texttt{rtf}) (uno para las dos ciudades)
    \item Los archivos de mediciones de temperatura (\texttt{.csv}) (uno por cada ciudad)
\end{enumerate}

\end{document}