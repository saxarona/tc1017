\documentclass[11pt]{article}
%\usepackage[a4paper]{geometry}
\usepackage{fullpage}
\usepackage[utf8]{inputenc}
\usepackage[spanish, mexico]{babel}
\usepackage{lipsum}
\usepackage{bm}
\usepackage{upgreek}
\usepackage{enumitem}
\usepackage{mathrsfs}
\usepackage{amsmath}
\usepackage{amssymb}
\usepackage{tikz}
\usepackage{tcolorbox}
\usepackage{csquotes}
\usepackage{listings}
%\usepackage{helvet}
\usepackage[numbered]{matlab-prettifier}


% mathtools for: Aboxed (put box on last equation in align envirenment)
\usepackage{microtype} %improves the spacing between words and letters

%% COLOR DEFINITIONS

\usepackage{xcolor} % Enabling mixing colors and color's call by 'svgnames'

\definecolor{MyColor1}{rgb}{0.2,0.4,0.6} %mix personal color
\newcommand{\textb}{\color{Black} \usefont{OT1}{lmss}{m}{n}}
\newcommand{\blue}{\color{MyColor1} \usefont{OT1}{lmss}{m}{n}}
\newcommand{\blueb}{\color{MyColor1} \usefont{OT1}{lmss}{b}{n}}
\newcommand{\red}{\color{LightCoral} \usefont{OT1}{lmss}{m}{n}}
\newcommand{\green}{\color{Turquoise} \usefont{OT1}{lmss}{m}{n}}

\DeclareMathOperator{\trace}{trace}
\DeclareMathOperator{\diag}{diag}

%% FONTS AND COLORS

%    SECTIONS

\usepackage{titlesec}
\usepackage{sectsty}
%%%%%%%%%%%%%%%%%%%%%%%%
%set section/subsections HEADINGS font and color
%\sectionfont{\color{Black}}  % sets colour of sections
%\subsectionfont{\color{Black}}  % sets colour of sections

%set section enumerator to arabic number (see footnotes markings alternatives)
\renewcommand\thesection{\arabic{section}} %define sections numbering
\renewcommand\thesubsection{\thesection\arabic{subsection}} %subsec.num.

%define new section style
\newcommand{\mysection}{
\titleformat{\section} [runin] {\usefont{OT1}{lmss}{b}{n}\color{MyColor1}} 
{\thesection} {3pt} {} } 


% %	CAPTIONS
% \usepackage{caption}
% \usepackage{subcaption}
% %%%%%%%%%%%%%%%%%%%%%%%%
% \captionsetup[figure]{labelfont={color=Turquoise}}


%		!!!EQUATION (ARRAY) --> USING ALIGN INSTEAD
%using amsmath package to redefine eq. numeration (1.1, 1.2, ...) 
\renewcommand{\theequation}{\thesection\arabic{equation}}

\setlength\parindent{0pt}




\makeatletter
\let\reftagform@=\tagform@
\def\tagform@#1{\maketag@@@{(\ignorespaces\textcolor{red}{#1}\unskip\@@italiccorr)}}
\renewcommand{\eqref}[1]{\textup{\reftagform@{\ref{#1}}}}
\makeatother
\usepackage{hyperref}
\hypersetup{colorlinks=true}

% For labeling top of page on every page but first one:
%\usepackage{fancyhdr}

\newcommand{\myclass}{TC1017 -- Solución de Problemas con Programación} % Class name?
\newcommand{\mytitle}{Examen 1} % Title of document?
\newcommand{\mydate}{23.09.19} % The date?
\newcommand{\myheader}{
    \begin{flushleft}
        \large
        Nombre: \rule{13 cm}{0.4mm} \\
        Matrícula: \rule{5 cm}{0.4mm} \hfill Fecha: \mydate
    \end{flushleft}
}

\newcommand{\matlab}[1]{\lstinline[style=Matlab-bw]!#1!}
\newcommand{\shortresponserule}{{\large\rule{5 cm}{0.3mm}}}
\newcommand{\responserule}{{\large\rule{10 cm}{0.3mm}}}

\title{
    \myclass \\
    \textbf{\mytitle} \\
    \myheader
    \date{}
}

% You can set the date automatically by replacing "date goes here" with "\today"

% \renewcommand{\rmdefault}{phv} % Arial Font
\renewcommand{\familydefault}{\sfdefault}

% \pagestyle{fancy}
% \fancyhead{}
% \fancyhead[CO,CE]{{\small{{\bf{\mytitle}} -- \myclass}}}

\begin{document}
\maketitle

\vspace{-1.5cm}

Lee cuidadosamente y contesta lo que se te pide.
Este examen es individual.

Al momento de contestar, intenta ser lo más explícito posible: se calificará con base en lo que esté escrito. %y se considerará el proceso aún cuando la respuesta final esté errada.
Recuerda que puedes revisar material de la clase, el libro de texto o tus notas. % o el MATLAB directamente.
Administra bien tu tiempo.
Buena suerte.

\section{Analiza el código (60\%)}

Analiza cada uno de los siguientes códigos y escribe el resultado para cada caso.

\subsection*{Careful Planning}

\lstinputlisting[style=Matlab-bw]{careful_planning.m}

\begin{itemize}
    \item \matlab{careful_planning(5)} \hfill \shortresponserule
    \item \matlab{careful_planning(20)} \hfill \shortresponserule
    \item ¿Qué hace esta función? \hfill \responserule
\end{itemize}

\subsection*{Efficient Training}

\lstinputlisting[style=Matlab-bw]{efficient_training.m}

\begin{itemize}
    \item \matlab{efficient_training(5)} \hfill \shortresponserule
    \item \matlab{efficient_training(8)} \hfill \shortresponserule
    \item ¿Qué hace esta función? \hfill \responserule
\end{itemize}

\pagebreak

\subsection*{Hazardous Research}

\lstinputlisting[style=Matlab-bw]{hazardous_research.m}
\begin{itemize}
    \item \matlab{hazardous_research(5)} \hfill \shortresponserule
    \item \matlab{hazardous_research(6)} \hfill \shortresponserule
    \item ¿Qué hace esta función? \hfill \responserule
\end{itemize}

\subsection*{Inspired Learning}

\lstinputlisting[style=Matlab-bw]{inspired_learning.m}
\begin{itemize}
    \item \matlab{inspired_learning(1,2,1)} \hfill \shortresponserule
    \item \matlab{inspired_learning(2,3,1)} \hfill \shortresponserule
    \item ¿Qué hace esta función? \hfill \responserule
\end{itemize}

\section{Operaciones aritméticas (12\%)}

Realiza correctamente las siguientes operaciones.
En caso de obtener decimales o raíces, minimiza lo más que puedas tu expresión.

\begin{enumerate}[label=\alph*)]
    \item $1 + 3 + 5 - 2 - 4 * 3 =$ \hfill \shortresponserule
    \item $2 / 2 * 2 + 2 * 2 / 2 =$ \hfill \shortresponserule
    \item $25 * 4 - 100 + 12 + 3 * 4 =$ \hfill \shortresponserule
    \item $20 * 10 + \sqrt{400} + 15 + 3 * 5 =$ \hfill \shortresponserule
    \item $0 + 20 - 10 * 2 + 25 / 5 + 3 + 2 - 10 =$ \hfill \shortresponserule
    \item $280 - 300 + 45 / 9 / 5 + 2 * 5 + 10 =$ \hfill \shortresponserule
\end{enumerate}


\section{Diseño de solución (28 \%)}

{\footnotesize \it
Este problema es un poco más complejo de lo que hemos resuelto hasta el momento.
Se recomienda que avances lo más que puedas en el resto del examen antes de comenzarlo.
}

\bigskip

El \textit{Biologic Space Lab} (BSL) es un laboratorio espacial que alberga especies extraterrestres para fines taxonómicos.
El Sector 3 (PYR) del BSL tiene 7235 especies distintas en su base de datos. A pesar de que todas las especies son de clima cálido, algunos pueden soportar temperaturas extremas y otras no.
La base de datos contiene, por cada especie: el nombre científico, el peso promedio, la estatura promedio y su resistencia al calor en grados Celsius.

Se te ha asignado la tarea de etiquetar a cada especie de acuerdo con su resistencia al calor, en tres distintas categorías:

\begin{itemize}
    \item Clase A: Los que soportan menos de 120 °C
    \item Clase B: Los que soportan entre 120 y 280 °C
    \item Clase C: Los que soportan más de 280 °C
\end{itemize}

¿Cuál es el proceso que seguirías para etiquetar cada una de las especies?
Haz un diagrama de flujo de tu solución y contesta las siguientes preguntas:

\begin{enumerate}[label=\alph*)]
    \item ¿Cuántas variables necesitas para guardar la información de una sola especie? ¿Cuáles son? (2\%)
    \item ¿Qué condición debe cumplirse para que una especie sea Clase B? (4\%)
    \item ¿Qué comando o función del MATLAB debes usar para revisar de qué clase es? (2 \%)
    \item Haz un diagrama de flujo para ejemplificar tu procedimiento  (10\%)
    \item Describe brevemente tu diagrama de flujo, como si platicaras cómo resolver el problema (10\%)
\end{enumerate}

% \section{Reto A (+5\%)}

% \begin{enumerate}[label=\alph*)]
%     \item ¿Cómo se llama la función para borrar la \textit{Command Window}? (+2 \%)
%     \item Haz el diagrama de flujo para el \texttt{holapueblo.m} (Tarea 2) (+3 \%)
% \end{enumerate}

\section{Reto (+10\%)}

Escribe el programa que escribirías en MATLAB para implementar la siguiente función (+8\%):

$$f(x) = %
\begin{cases}
    &x, \text{ si } x > 0 \\
    & -x, \text{ si } x < 0 \\
    &0, \text{ si } x = 0
\end{cases}
$$

\vspace{2ex}

¿Qué está haciendo esta función? (+2\%) \responserule

\vfill

\textbf{De acuerdo con el Código de Ética del Tecnológico de Monterrey, mi desempeño en esta actividad estará guiado por la integridad académica.}
\end{document}