\documentclass[12pt, letterpaper, oneside]{article}
%\usepackage{geometry}
\usepackage[spanish, mexico]{babel}
\usepackage[utf8]{inputenc}
\usepackage{amssymb}
\usepackage[inner=1.5cm,outer=1.5cm,top=2.5cm,bottom=2.5cm]{geometry}
\pagestyle{empty}
\usepackage{graphicx}
\usepackage{fancyhdr, lastpage, bbding, pmboxdraw}
\usepackage[usenames,dvipsnames]{color}
\usepackage{multicol}
\definecolor{darkblue}{rgb}{0,0,.6}
\definecolor{darkred}{rgb}{.7,0,0}
\definecolor{darkgreen}{rgb}{0,.6,0}
\definecolor{red}{rgb}{.98,0,0}
\usepackage[colorlinks,pagebackref,pdfusetitle,urlcolor=darkblue,citecolor=darkblue,linkcolor=darkred,bookmarksnumbered,plainpages=false]{hyperref}
\renewcommand{\thefootnote}{\fnsymbol{footnote}}

\newcommand{\thecourse}{Solución de Problemas con Programación (TC1017--10-700)}
\newcommand{\thesemester}{Agosto--Diciembre 2019}
\newcommand{\theinstructor}{Xavier Sánchez Díaz}
\newcommand{\themail}{sax@tec.mx}
\newcommand{\thetime}{Lu 18:00--21:00 hrs}
\newcommand{\theplace}{\scriptsize A3-303 (MTY) AIII-3208 (SAL)}

\newcommand{\topic}{{\color{darkgreen}{\Rectangle}}}
\newcommand{\subtopic}{{\enskip \color{darkblue}{\Rectangle}}}

\pagestyle{fancyplain}
\fancyhf{}
\lhead{ \fancyplain{}{\thecourse} }
%\chead{ \fancyplain{}{} }
\rhead{ \fancyplain{}{\thesemester} }
%\rfoot{\fancyplain{}{page \thepage\ of \pageref{LastPage}}}
\fancyfoot[RO] {Página \thepage\ de \pageref{LastPage}}
\thispagestyle{plain}

%%%%%%%%%%%% LISTING %%%
\usepackage{listings}
\usepackage{caption}
\DeclareCaptionFont{white}{\color{white}}
\DeclareCaptionFormat{listing}{\colorbox{gray}{\parbox{\textwidth}{#1#2#3}}}
% \captionsetup[lstlisting]{format=listing,labelfont=white,textfont=white}
\usepackage{verbatim} % used to display code
\usepackage{fancyvrb}
\usepackage{acronym}
\usepackage{amsthm}
\VerbatimFootnotes % Required, otherwise verbatim does not work in footnotes!



\definecolor{OliveGreen}{cmyk}{0.64,0,0.95,0.40}
\definecolor{CadetBlue}{cmyk}{0.62,0.57,0.23,0}
\definecolor{lightlightgray}{gray}{0.93}

\lstset{
  %language=bash,                          % Code langugage
  basicstyle=\ttfamily,                   % Code font, Examples: \footnotesize, \ttfamily
  keywordstyle=\color{OliveGreen},        % Keywords font ('*' = uppercase)
  commentstyle=\color{gray},              % Comments font
  numbers=left,                           % Line nums position
  numberstyle=\tiny,                      % Line-numbers fonts
  stepnumber=1,                           % Step between two line-numbers
  numbersep=5pt,                          % How far are line-numbers from code
  backgroundcolor=\color{lightlightgray}, % Choose background color
  frame=none,                             % A frame around the code
  tabsize=2,                              % Default tab size
  captionpos=t,                           % Caption-position = bottom
  breaklines=true,                        % Automatic line breaking?
  breakatwhitespace=false,                % Automatic breaks only at whitespace?
  showspaces=false,                       % Dont make spaces visible
  showtabs=false,                         % Dont make tabls visible
  columns=flexible,                       % Column format
  morekeywords={__global__, __device__},  % CUDA specific keywords
}

%%%%%%%%%%%%%%%%%%%%%%%%%%%%%%%%%%%%
\begin{document}
  \begin{center}
  {\Large \textsc{\thecourse}}
  \end{center}
  \begin{center}
  \thesemester
  \end{center}

  \begin{center}
  \rule{6in}{0.4pt}
  \begin{minipage}[t]{.75\textwidth}
  \begin{tabular}{llcccll}
  \textbf{Instructor:} & \theinstructor & & &  & \textbf{Hora:} & \thetime \\
  \textbf{Email:} &  \href{mailto:sax@tec.mx}{\themail} & & & & \textbf{Lugar:} & \theplace
  \end{tabular}
  \end{minipage}
  \rule{6in}{0.4pt}
  \end{center}
  \vspace{.5cm}
  \setlength{\unitlength}{1in}
  \renewcommand{\arraystretch}{2}

  \section{Política de puntos extras (estrellas)}

  Algunas tareas y actividades en clase están pensadas para acumular puntos extras para mejorar la calificación final.
  Estos puntos extras son adicionales, y se miden en \textbf{estrellas}.
  10 estrellas pueden ser cambiadas por un punto para la calificación final.
  Considera que estos puntos no pueden ser utilizados en los exámenes parciales.

  Para poder hacer uso de las estrellas, debes seguir los siguientes pasos:

  \begin{itemize}
      \item Sólo puedes intercambiar estrellas en grupos de 10.
      De este modo, tener 10 estrellas te permite subir la calificación final por un punto, al igual que si tuvieras 19 estrellas.
      \item El número máximo de puntos extras que puedes conseguir tiene como límite el 5\% de la calificación final antes de los puntos extras.
      Por ejemplo, si obtienes un resultado final de 75, el máximo de puntos extras que puedes obtener es $75*0.05=3.75$, por lo que necesitas al menos 40 estrellas para ello.
      Asumiendo que lo conseguiste, entonces tu calificación final sería de $75 + 3.75 = 78.75 \approx 79$.
      \item Para poder hacer uso de las estrellas, considera que debes obtener una calificación aprobatoria.
      Por lo mismo, no consideres los puntos extras para aprobar el curso.
      Los puntos extras deben considerarse como una manera de premiar tu esfuerzo mejorando tu puntuación, no como un intento desesperado para pasar la materia.
      \textbf{Trabaja regularmente en las tareas y estudia para los exámenes, ya que es la clave para aprobar el curso}.
      \item Cualquier falta al Código de Honor del Tecnológico de Monterrey  repercutirá también en el programa de estrellas:
      \begin{itemize}
          \item El alumno que cometa la falta será excluido del programa de estrellas.
          \item El \textbf{número total de actividades de estrellas} será reducido en uno, afectando el máximo de estrellas que puede obtener cada estudiante durante el semestre.
      \end{itemize}
  \end{itemize}

  %%%%%% END 
\end{document} 