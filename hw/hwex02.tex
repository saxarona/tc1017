\documentclass[]{book}

%These tell TeX which packages to use.
\usepackage{array,epsfig}
\usepackage{amsmath}
\usepackage{amsfonts}
\usepackage{amssymb}
\usepackage{amsxtra}
\usepackage{amsthm}
\usepackage{mathrsfs}
\usepackage{color}
\usepackage[spanish, mexico]{babel}
\usepackage[utf8]{inputenc}
\usepackage{enumitem}
\usepackage{helvet}
\usepackage[framed]{matlab-prettifier}
\newcommand{\matlab}[1]{\lstinline[style=Matlab-pyglike]!#1!}

%Here I define some theorem styles and shortcut commands for symbols I use often
\theoremstyle{definition}
\newtheorem{defn}{Definition}
\newtheorem{thm}{Theorem}
\newtheorem{cor}{Corollary}
\newtheorem*{rmk}{Remark}
\newtheorem{lem}{Lemma}
\newtheorem*{joke}{Joke}
\newtheorem{ex}{Example}
\newtheorem*{sol}{Solution}
\newtheorem{prop}{Proposition}

\renewcommand{\familydefault}{\sfdefault}

\newcommand{\lra}{\longrightarrow}
\newcommand{\ra}{\rightarrow}
\newcommand{\surj}{\twoheadrightarrow}
\newcommand{\graph}{\mathrm{graph}}
\newcommand{\bb}[1]{\mathbb{#1}}
\newcommand{\Z}{\bb{Z}}
\newcommand{\Q}{\bb{Q}}
\newcommand{\R}{\bb{R}}
\newcommand{\C}{\bb{C}}
\newcommand{\N}{\bb{N}}
\newcommand{\M}{\mathbf{M}}
\newcommand{\m}{\mathbf{m}}
\newcommand{\MM}{\mathscr{M}}
\newcommand{\HH}{\mathscr{H}}
\newcommand{\Om}{\Omega}
\newcommand{\Ho}{\in\HH(\Om)}
\newcommand{\bd}{\partial}
\newcommand{\del}{\partial}
\newcommand{\bardel}{\overline\partial}
\newcommand{\textdf}[1]{\textbf{\textsf{#1}}\index{#1}}
\newcommand{\img}{\mathrm{img}}
\newcommand{\ip}[2]{\left\langle{#1},{#2}\right\rangle}
\newcommand{\inter}[1]{\mathrm{int}{#1}}
\newcommand{\exter}[1]{\mathrm{ext}{#1}}
\newcommand{\cl}[1]{\mathrm{cl}{#1}}
\newcommand{\ds}{\displaystyle}
\newcommand{\vol}{\mathrm{vol}}
\newcommand{\cnt}{\mathrm{ct}}
\newcommand{\osc}{\mathrm{osc}}
\newcommand{\LL}{\mathbf{L}}
\newcommand{\UU}{\mathbf{U}}
\newcommand{\support}{\mathrm{support}}
\newcommand{\AND}{\;\wedge\;}
\newcommand{\OR}{\;\vee\;}
\newcommand{\Oset}{\varnothing}
\newcommand{\st}{\ni}
\newcommand{\wh}{\widehat}

%Pagination stuff.
\setlength{\topmargin}{-.3 in}
\setlength{\oddsidemargin}{0in}
\setlength{\evensidemargin}{0in}
\setlength{\textheight}{9.in}
\setlength{\textwidth}{6.5in}
\setlength{\itemsep}{0.45in}
\pagestyle{empty}



\begin{document}

\begin{center}
{\huge Solución de Problemas con Programación (TC1017)}\\[1.5ex]
{\large \textbf{Tarea EX02}\\[1.5ex] %You should put your name here
27.02.19} %You should write the date here.
\end{center}

\vspace{0.2 cm}

\subsection*{Matrices y condiciones}

El \textit{Biologic Space Lab} (BSL) es un laboratorio espacial que alberga especies extraterrestres para fines taxonómicos.
El Sector 1 (SRX) del BSL tiene 3500 individuos en su base de datos, algunos de ellos necesitan oxígeno para sobrevivir, mientras que otros necesitan nitrógeno. El Sector 2 (TRO) tiene 12000 individuos, y el Sector 3 (PYR) tiene 2500.
Si la base de datos de cada sector contiene, por cada individuo, el gas que necesita para sobrevivir (oxígeno o nitrógeno) y su consumo de gas por hora ¿Cómo calcularías cuánto gas de determinado tipo se consume en \textbf{cierto sector} por segundo?

Haz una función que reciba \textbf{dos parámetros}: \texttt{m} la matriz que representa a un sector y \texttt{g} el tipo de gas que se va a calcular.
Dicha función calculará el consumo total de gas en el sector \textbf{por segundo} y lo \textbf{regresará} como \texttt{output} de la función.

Sugerencias para diseñar tu solución:
\begin{enumerate}[label=\alph*)]
    \item Piensa en que se utilizará una matriz para guardar la información de un sector. Cada fila será un individuo. Por ejemplo, algo así:
    $$A = \begin{bmatrix}
       1 & 800 \\
       0 & 267 \\
       0 & 90 \\
       1 & 576 \\
       1 & 2750
    \end{bmatrix}$$
    \item ¿Qué columna de la matriz debes revisar para saber si contar al individuo o no?
    \item ¿Qué condición debe cumplirse para hacer la suma del consumo si te pide contar el total de oxígeno? ¿Y si te pide nitrógeno?
    \item ¿Cuál debe ser el límite del ciclo para leer toda la matriz? ¿De qué depende este número?
\end{enumerate}

\bigskip

Deberás entregar \textbf{un archivo} de MATLAB con extensión \textbf{.m}: \texttt{gas.m}.
El archivo debe tener la función y su documentación \textbf{(en la segunda línea)} en el siguiente formato:

\bigskip

\begin{lstlisting}[style=Matlab-editor]
    function output = gas(x)
    % NAME: Arturo Gonzalez
    % STUDENT ID: A01170065
    % GAS output = gas (m, g)
    % GAS gets a matrix m and calculates the total gas consumption
    % per second, depending on the type of gas g it receives as parameter.
    % You can use it like this:
    % gas(A, 1)
    % assuming you have already declared a matrix A and
    % look for the total oxygen.
    % Use 0 if you want nitrogen consumption.
    ...
\end{lstlisting}

\pagebreak

{\Large Recomendaciones:}

\begin{itemize}
    \item Recuerda que es una función. No estamos buscando el resultado, sino una solución genérica, para calcular el consumo sin importar el número del sector ni cuántos individuos existen.
    \item Si lo crees necesario, haz un diagrama de flujo que te ayude a guiarte en el proceso.
    \item Asegúrate de estar corriendo el MATLAB en el mismo lugar donde guardaste tus archivos.
    \item Asegúrate de que tu archivo tiene nombre en minúsculas, sin espacios ni acentos o símbolos. 
    \item No te olvides de \textbf{documentar tu función} e incluir tu nombre y número de matrícula en el formato especificado arriba.
    \item \textbf{Reto A} (+1): ¿Quieres incluir los nombres de los individuos o una letra para el gas? Usa un \textit{cell array}, así:
    \begin{lstlisting}[style=Matlab-editor]
        % declare the array
        A = {'Dromiceiomimus', 'o', 1200; 'Ranunculus', 'n', 200}
        
        % Get the name of the second entry (Dromiceiomimus)
        A{2,1}

        % Get the gas consumption of the first entry
        A{1,3}
    \end{lstlisting}
    \item \textbf{Reto B} (+1): ¿Quieres usar \texttt{o} y \texttt{n} para el tipo de gas en lugar de 1 y 0? Puedes usar \matlab{switch} para definir qué hacer en caso de que se pida oxígeno o nitrógeno.
\end{itemize}

\vfill

\textbf{De acuerdo con el Código de Ética del Tecnológico de Monterrey, mi desempeño en esta actividad estará guiado por la integridad académica.}
\end{document}