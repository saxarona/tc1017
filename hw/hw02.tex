\documentclass[]{book}

%These tell TeX which packages to use.
\usepackage{array,epsfig}
\usepackage{amsmath}
\usepackage{amsfonts}
\usepackage{amssymb}
\usepackage{amsxtra}
\usepackage{amsthm}
\usepackage{mathrsfs}
\usepackage{color}
\usepackage[spanish, mexico]{babel}
\usepackage[utf8]{inputenc}
\usepackage{enumitem}
\usepackage{helvet}
\usepackage[framed]{matlab-prettifier}
\newcommand{\matlab}[1]{\lstinline[style=Matlab-pyglike]!#1!}

%Here I define some theorem styles and shortcut commands for symbols I use often
\theoremstyle{definition}
\newtheorem{defn}{Definition}
\newtheorem{thm}{Theorem}
\newtheorem{cor}{Corollary}
\newtheorem*{rmk}{Remark}
\newtheorem{lem}{Lemma}
\newtheorem*{joke}{Joke}
\newtheorem{ex}{Example}
\newtheorem*{sol}{Solution}
\newtheorem{prop}{Proposition}

\renewcommand{\familydefault}{\sfdefault}

\newcommand{\lra}{\longrightarrow}
\newcommand{\ra}{\rightarrow}
\newcommand{\surj}{\twoheadrightarrow}
\newcommand{\graph}{\mathrm{graph}}
\newcommand{\bb}[1]{\mathbb{#1}}
\newcommand{\Z}{\bb{Z}}
\newcommand{\Q}{\bb{Q}}
\newcommand{\R}{\bb{R}}
\newcommand{\C}{\bb{C}}
\newcommand{\N}{\bb{N}}
\newcommand{\M}{\mathbf{M}}
\newcommand{\m}{\mathbf{m}}
\newcommand{\MM}{\mathscr{M}}
\newcommand{\HH}{\mathscr{H}}
\newcommand{\Om}{\Omega}
\newcommand{\Ho}{\in\HH(\Om)}
\newcommand{\bd}{\partial}
\newcommand{\del}{\partial}
\newcommand{\bardel}{\overline\partial}
\newcommand{\textdf}[1]{\textbf{\textsf{#1}}\index{#1}}
\newcommand{\img}{\mathrm{img}}
\newcommand{\ip}[2]{\left\langle{#1},{#2}\right\rangle}
\newcommand{\inter}[1]{\mathrm{int}{#1}}
\newcommand{\exter}[1]{\mathrm{ext}{#1}}
\newcommand{\cl}[1]{\mathrm{cl}{#1}}
\newcommand{\ds}{\displaystyle}
\newcommand{\vol}{\mathrm{vol}}
\newcommand{\cnt}{\mathrm{ct}}
\newcommand{\osc}{\mathrm{osc}}
\newcommand{\LL}{\mathbf{L}}
\newcommand{\UU}{\mathbf{U}}
\newcommand{\support}{\mathrm{support}}
\newcommand{\AND}{\;\wedge\;}
\newcommand{\OR}{\;\vee\;}
\newcommand{\Oset}{\varnothing}
\newcommand{\st}{\ni}
\newcommand{\wh}{\widehat}

%Pagination stuff.
\setlength{\topmargin}{-.3 in}
\setlength{\oddsidemargin}{0in}
\setlength{\evensidemargin}{0in}
\setlength{\textheight}{9.in}
\setlength{\textwidth}{6.5in}
\setlength{\itemsep}{0.45in}
\pagestyle{empty}



\begin{document}

\begin{center}
{\huge Solución de Problemas con Programación (TC1017)}\\[1.5ex]
{\large \textbf{Tarea 02}\\[1.5ex] %You should put your name here
26.08.19} %You should write the date here.
\end{center}

\vspace{0.2 cm}

\subsection*{Funciones}

Programa en MATLAB/Octave las siguientes funciones:

\begin{enumerate}[label=\alph*)]
    \itemsep2.5ex
    \item La función \matlab{duplica}
    \begin{itemize}
        \item Debe \textbf{recibir} un parámetro, $x$.
        \item Debe \textbf{devolver} el doble del valor de $x$.
        \item Ejemplo: \matlab{duplica(20)} $\implies$ \matlab{40}.
    \end{itemize}
    \item La función \matlab{sumar}
    \begin{itemize}
        \item Debe \textbf{recibir} tres parámetros.
        \item Debe \textbf{devolver} la suma de los tres parámetros.
        \item Ejemplo: \matlab{sumar(2,3,5)} $\implies$ \matlab{10}.
    \end{itemize}
    \item La función \matlab{holapueblo}
    \begin{itemize}
        \item Debe \textbf{recibir} un parámetro, $x$.
        \item Debe \textbf{imprimir}: `Hola' \textbf{si} $x$ es divisible entre 3
        \item Debe \textbf{imprimir}: `Pueblo' \textbf{si} $x$ es divisible entre 5
        \item Debe \textbf{imprimir}: `Hola Pueblo' \textbf{si} $x$ es divisible entre 3 y entre 5
        \item Debe \textbf{imprimir}: `No' \textbf{si} $x$ no cae en ninguno de los casos anteriores.
        \item Ejemplo: \matlab{holapueblo(30)} $\implies$ \texttt{"Hola Pueblo"}
    \end{itemize}
\end{enumerate}

\bigskip

Deberás entregar un archivo comprimido en \textbf{ZIP} que contenga \textbf{tres archivos} de MATLAB con extensión \textbf{.m}: \texttt{duplica.m}, \texttt{sumar.m} y \texttt{holapueblo.m}.
Cada archivo debe tener la función y su documentación en el siguiente formato:

\bigskip

\begin{lstlisting}[style=Matlab-editor]
function output = lafuncion(x)
% NAME: Arturo Gonzalez
% STUDENT ID: A01170065
% LAFUNCION output = LAFUNCION (x)
% LAFUNCION returns some magic
% You can use it like this
% lafuncion(200) = 30000000
...
\end{lstlisting}

\pagebreak

{\Large Recomendaciones:}
\begin{itemize}
    \item Procura tener muy claro los comandos de \matlab{function}, \matlab{if}, \matlab{else}, \matlab{end} y \matlab{mod}.
    \item Es probable que necesites el comando \matlab{elseif}. Utiliza la función \matlab{help} para saber cómo utilizarlo.
    \item Para \textbf{imprimir} un valor puedes hacerlo con \matlab{sprintf} o \matlab{disp}. Para delimitar una palabra a imprimir, utiliza comillas dobles: \matlab{disp("Hola Pueblo Viejo y Sabio")}.
    \item Recuerda que si necesitas guardar un valor puedes asignar una variable, así: \matlab{out = x + 7}.
    \item Asegúrate de estar corriendo el MATLAB en el mismo lugar donde guardaste tus archivos.
    \item Asegúrate de que tus archivos se llaman igual que tus funciones.
    \item Asegúrate de que tus archivos tienen nombres en minúsculas, sin espacios ni acentos o símbolos. 
    \item No te olvides de \textbf{documentar tus funciones} e incluir tu nombre y número de matrícula en el formato especificado arriba.
\end{itemize}
\end{document}