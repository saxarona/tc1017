\documentclass[]{book}

%These tell TeX which packages to use.
\usepackage{array,epsfig}
\usepackage{amsmath}
\usepackage{amsfonts}
\usepackage{amssymb}
\usepackage{amsxtra}
\usepackage{amsthm}
\usepackage{mathrsfs}
\usepackage{color}
\usepackage[spanish, mexico]{babel}
\usepackage[utf8]{inputenc}
\usepackage{enumitem}
\usepackage{helvet}
\usepackage[framed]{matlab-prettifier}
\usepackage{booktabs}
\newcommand{\matlab}[1]{\lstinline[style=Matlab-pyglike]!#1!}

%Here I define some theorem styles and shortcut commands for symbols I use often
\theoremstyle{definition}
\newtheorem{defn}{Definition}
\newtheorem{thm}{Theorem}
\newtheorem{cor}{Corollary}
\newtheorem*{rmk}{Remark}
\newtheorem{lem}{Lemma}
\newtheorem*{joke}{Joke}
\newtheorem{ex}{Example}
\newtheorem*{sol}{Solution}
\newtheorem{prop}{Proposition}

\renewcommand{\familydefault}{\sfdefault}

\newcommand{\lra}{\longrightarrow}
\newcommand{\ra}{\rightarrow}
\newcommand{\surj}{\twoheadrightarrow}
\newcommand{\graph}{\mathrm{graph}}
\newcommand{\bb}[1]{\mathbb{#1}}
\newcommand{\Z}{\bb{Z}}
\newcommand{\Q}{\bb{Q}}
\newcommand{\R}{\bb{R}}
\newcommand{\C}{\bb{C}}
\newcommand{\N}{\bb{N}}
\newcommand{\M}{\mathbf{M}}
\newcommand{\m}{\mathbf{m}}
\newcommand{\MM}{\mathscr{M}}
\newcommand{\HH}{\mathscr{H}}
\newcommand{\Om}{\Omega}
\newcommand{\Ho}{\in\HH(\Om)}
\newcommand{\bd}{\partial}
\newcommand{\del}{\partial}
\newcommand{\bardel}{\overline\partial}
\newcommand{\textdf}[1]{\textbf{\textsf{#1}}\index{#1}}
\newcommand{\img}{\mathrm{img}}
\newcommand{\ip}[2]{\left\langle{#1},{#2}\right\rangle}
\newcommand{\inter}[1]{\mathrm{int}{#1}}
\newcommand{\exter}[1]{\mathrm{ext}{#1}}
\newcommand{\cl}[1]{\mathrm{cl}{#1}}
\newcommand{\ds}{\displaystyle}
\newcommand{\vol}{\mathrm{vol}}
\newcommand{\cnt}{\mathrm{ct}}
\newcommand{\osc}{\mathrm{osc}}
\newcommand{\LL}{\mathbf{L}}
\newcommand{\UU}{\mathbf{U}}
\newcommand{\support}{\mathrm{support}}
\newcommand{\AND}{\;\wedge\;}
\newcommand{\OR}{\;\vee\;}
\newcommand{\Oset}{\varnothing}
\newcommand{\st}{\ni}
\newcommand{\wh}{\widehat}

%Pagination stuff.
\setlength{\topmargin}{-.3 in}
\setlength{\oddsidemargin}{0in}
\setlength{\evensidemargin}{0in}
\setlength{\textheight}{9.in}
\setlength{\textwidth}{6.5in}
\setlength{\itemsep}{0.45in}
\pagestyle{empty}



\begin{document}

\begin{center}
{\huge Solución de Problemas con Programación (TC1017)}\\[1.5ex]
{\large \textbf{Tarea 04}\\[1.5ex] %You should put your name here
23.03.19} %You should write the date here.
\end{center}

\vspace{0.2 cm}

\subsection*{Matrices y operaciones}

Una empresa vende sacos valvulados rectangulares para su uso como empaque de ciertos productos industriales (productos granulares o \textit{pellets}, fertilizantes, alimentos, agregados para la industria de la construcción, plásticos y pinturas).
El costo unitario de un saco depende del peso del contenido, del material y de su volumen, usando la siguiente ecuación:

\bigskip

$$Costo_{unit} = Costo_{mater} * \left( 1.2 + \frac{Peso_{cont}}{Vol_{saco}} \right)$$

\bigskip

Asumiendo que las órdenes de compra te las dan en una matriz con el formato siguiente:

\begin{table}[htbp]
    \centering
    \begin{tabular}{@{}cccccc@{}}
    \toprule
    $ID_{ord}$ & $Alto$ & $Ancho$ & $Fondo$ & $Peso$ & $Costo_{mater}$ \\ \midrule
    1 & 150 & 220 & 25 & 2000 & 4.5 \\
     &  &  & $\vdots$ &  &  \\
    $x$ & 120 & 250 & 37 & 500 & 11.5 \\ \bottomrule
    \end{tabular}
\end{table}

Escribe una función que reciba como \textbf{único parámetro una matriz} en el formato anterior,
y que \textbf{devuelva} como resultado \textbf{una matriz de costos} con el siguiente formato:

\begin{table}[htbp]
    \centering
    \begin{tabular}{@{}cc@{}}
    \toprule
    $ID_{ord}$ & $Costo_{unit}$ \\ \midrule
    1 & 5.4101 \\
    $\vdots$ & $\vdots$ \\ \bottomrule
    \end{tabular}
\end{table}

\bigskip

Sugerencias para diseñar tu solución:
\begin{enumerate}[label=\alph*)]
    \item ¿Qué columnas de la matriz de \textit{input} debes revisar para obtener el volumen de la bolsa? ¿Cómo lo calculas?
    \item ¿Qué funciones u operaciones necesitas?
    \item \textbf{Reto:} ¿Puedes hacerlo sin usar un \matlab{for}? (Spoiler: sí puedes) (+1)
\end{enumerate}

\pagebreak

Deberás entregar \textbf{un archivo} de MATLAB con extensión \textbf{.m}: \texttt{bolsas.m}.
El archivo debe tener la función y su documentación \textbf{(en la segunda línea)} en el siguiente formato:

\bigskip

\begin{lstlisting}[style=Matlab-editor]
    function output = bolsas(A)
    % NAME: Arturo Gonzalez
    % STUDENT ID: A01170065
    % BOLSAS output = gas (A)
    % BOLSAS gets a matrix A and calculates the unit price of container bags,
    % depending on the type the volume, weight and materials.
    % You can use it like this:
    % bolsas(A)
    % assuming you have already declared a matrix A
    ...
\end{lstlisting}

\bigskip

{\Large Recomendaciones:}

\begin{itemize}
    \item Recuerda que es una función. No estamos buscando el resultado, sino una solución genérica, para calcular los precios unitarios sin importar el número de órdenes en la matriz.
    \item Si lo crees necesario, haz un diagrama de flujo que te ayude a guiarte en el proceso.
    \item Asegúrate de estar corriendo el MATLAB en el mismo lugar donde guardaste tus archivos.
    \item Asegúrate de que tu archivo tiene nombre en minúsculas, sin espacios ni acentos o símbolos. 
    \item No te olvides de \textbf{documentar tu función} e incluir tu nombre y número de matrícula en el formato especificado arriba.
\end{itemize}

\vfill

\textbf{De acuerdo con el Código de Ética del Tecnológico de Monterrey, mi desempeño en esta actividad estará guiado por la integridad académica.}
\end{document}