\documentclass[]{book}

%These tell TeX which packages to use.
\usepackage{array,epsfig}
\usepackage{amsmath}
\usepackage{amsfonts}
\usepackage{amssymb}
\usepackage{amsxtra}
\usepackage{amsthm}
\usepackage{mathrsfs}
\usepackage{color}
\usepackage[spanish, mexico]{babel}
\usepackage[utf8]{inputenc}
\usepackage{enumitem}
\usepackage{helvet}

%Here I define some theorem styles and shortcut commands for symbols I use often
\theoremstyle{definition}
\newtheorem{defn}{Definition}
\newtheorem{thm}{Theorem}
\newtheorem{cor}{Corollary}
\newtheorem*{rmk}{Remark}
\newtheorem{lem}{Lemma}
\newtheorem*{joke}{Joke}
\newtheorem{ex}{Example}
\newtheorem*{sol}{Solution}
\newtheorem{prop}{Proposition}

\renewcommand{\familydefault}{\sfdefault}

\newcommand{\lra}{\longrightarrow}
\newcommand{\ra}{\rightarrow}
\newcommand{\surj}{\twoheadrightarrow}
\newcommand{\graph}{\mathrm{graph}}
\newcommand{\bb}[1]{\mathbb{#1}}
\newcommand{\Z}{\bb{Z}}
\newcommand{\Q}{\bb{Q}}
\newcommand{\R}{\bb{R}}
\newcommand{\C}{\bb{C}}
\newcommand{\N}{\bb{N}}
\newcommand{\M}{\mathbf{M}}
\newcommand{\m}{\mathbf{m}}
\newcommand{\MM}{\mathscr{M}}
\newcommand{\HH}{\mathscr{H}}
\newcommand{\Om}{\Omega}
\newcommand{\Ho}{\in\HH(\Om)}
\newcommand{\bd}{\partial}
\newcommand{\del}{\partial}
\newcommand{\bardel}{\overline\partial}
\newcommand{\textdf}[1]{\textbf{\textsf{#1}}\index{#1}}
\newcommand{\img}{\mathrm{img}}
\newcommand{\ip}[2]{\left\langle{#1},{#2}\right\rangle}
\newcommand{\inter}[1]{\mathrm{int}{#1}}
\newcommand{\exter}[1]{\mathrm{ext}{#1}}
\newcommand{\cl}[1]{\mathrm{cl}{#1}}
\newcommand{\ds}{\displaystyle}
\newcommand{\vol}{\mathrm{vol}}
\newcommand{\cnt}{\mathrm{ct}}
\newcommand{\osc}{\mathrm{osc}}
\newcommand{\LL}{\mathbf{L}}
\newcommand{\UU}{\mathbf{U}}
\newcommand{\support}{\mathrm{support}}
\newcommand{\AND}{\;\wedge\;}
\newcommand{\OR}{\;\vee\;}
\newcommand{\Oset}{\varnothing}
\newcommand{\st}{\ni}
\newcommand{\wh}{\widehat}

%Pagination stuff.
\setlength{\topmargin}{-.3 in}
\setlength{\oddsidemargin}{0in}
\setlength{\evensidemargin}{0in}
\setlength{\textheight}{9.in}
\setlength{\textwidth}{6.5in}
\setlength{\itemsep}{0.45in}
\pagestyle{empty}



\begin{document}

\begin{center}
{\huge Solución de Problemas con Programación (TC1017)}\\[1.5ex]
{\large \textbf{Tarea 00}\\[1.5ex] %You should put your name here
19.08.19} %You should write the date here.
\end{center}

\vspace{0.2 cm}

\subsection*{Toma de datos estadísticos}

Para hacer más ameno el trabajar con datos, es recomendable conseguir datos de un tema que nos guste.
Por ello, se presentan las siguientes opciones. Léelas con detenimiento y escoge la que más te guste.

\begin{enumerate}[label=\alph*)]
    \item \textdf{Música}. Usando el reproductor de tu preferencia (ya sea en \textit{streaming} o en local) guarda los siguientes datos:
    \begin{itemize}
        \item \textdf{Fecha}, en formato \texttt{DD.MM.AA}. Por ejemplo: \texttt{12.09.19} para el 12 de septiembre de 2019.
        \item \textdf{Título de la canción}. Por ejemplo: \texttt{La vie en rose}
        \item \textdf{Artista}. Por ejemplo: \texttt{Edith Piaf}
        \item \textdf{¿La oíste completa?}. Por ejemplo: \texttt{Yes} si la escuchaste completa o \texttt{No} si le diste \textit{skip} antes de que terminara.
    \end{itemize}
    \item \textdf{Netflix}. Sean películas o series, guarda los siguientes datos:
    \begin{itemize}
        \item \textdf{Fecha}, en formato \texttt{DD.MM.AA}. Por ejemplo: \texttt{12.09.19} para el 12 de septiembre de 2019.
        \item \textdf{Título del episodio}. Por ejemplo: \texttt{The one when everybody finds out}. Si es película, pon \texttt{NA}.
        \item \textdf{Título de la serie/película}. Por ejemplo: \texttt{Chilling Adventures of Sabrina} o \texttt{Crimson Peak}.
        \item \textdf{¿Viste el opening/ending?}. Por ejemplo: \texttt{Yes} si viste el \textit{opening} o los créditos, y \texttt{No} si te los saltaste.
    \end{itemize}
    \item \textdf{Juegos}. En Steam, Overwatch, Fortnite, League of Legends, lo que sea... guarda los siguientes datos:
    \begin{itemize}
        \item \textdf{Fecha}, en formato \texttt{DD.MM.AA}. Por ejemplo: \texttt{12.09.19} para el 12 de septiembre de 2019.
        \item \textdf{Nombre del juego}. Por ejemplo: \texttt{Path of Exile} o \texttt{Starcraft II}.
        \item \textdf{Tipo de juego}. Nombre del mapa o tipo de \textit{match}, por ejemplo: \texttt{Summoner's Rift} o \texttt{Hanamura} o \texttt{Vaal Temple}. \texttt{NA} si no aplica.
        \item \textdf{¿Ganaste, perdiste, empataste?}. Por ejemplo: \texttt{Win} si ganaste o \texttt{Draw} si empataste o \texttt{Loss} si perdiste. \texttt{NA} si no aplica.
    \end{itemize}
\end{enumerate}

\pagebreak

\subsubsection*{¿Cómo lo guardo?}

Mi sugerencia es usar algún archivo de Excel o Google Sheets. Después, para el análisis, usaremos la versión en CSV.
Puedes usar un editor de texto y trabajar en un CSV directamente.
No hay problema alguno.

\subsubsection*{Recomendaciones adicionales}
\begin{itemize}
    \item No te estreses. Tenemos poco más de dos meses para juntar datos. Si se te pasa alguna medición, no importa.
    \item Necesitamos mínimo 50 mediciones.
    \item Sé consistente en tus datos. Si usas \texttt{Yes}, siempre usa \texttt{Yes}: con mayúscula, y sin espacios extras ni nada más.
    \item Ten siempre un respaldo de tus datos en línea. \textit{Step-up your game: puedes hacer un formulario usando Google Sheets + Forms y lo rellenas en tu celular sin tanto problema}.
\end{itemize}

Al final, utilizaremos los datos para hacer análisis estadístico básico, probablemente para el segundo parcial.
\end{document}