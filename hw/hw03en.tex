\documentclass[]{book}

%These tell TeX which packages to use.
\usepackage{array,epsfig}
\usepackage{amsmath}
\usepackage{amsfonts}
\usepackage{amssymb}
\usepackage{amsxtra}
\usepackage{amsthm}
\usepackage{mathrsfs}
\usepackage{color}
\usepackage[spanish, mexico]{babel}
\usepackage[utf8]{inputenc}
\usepackage{enumitem}
\usepackage{helvet}
\usepackage[framed]{matlab-prettifier}
\newcommand{\matlab}[1]{\lstinline[style=Matlab-pyglike]!#1!}

%Here I define some theorem styles and shortcut commands for symbols I use often
\theoremstyle{definition}
\newtheorem{defn}{Definition}
\newtheorem{thm}{Theorem}
\newtheorem{cor}{Corollary}
\newtheorem*{rmk}{Remark}
\newtheorem{lem}{Lemma}
\newtheorem*{joke}{Joke}
\newtheorem{ex}{Example}
\newtheorem*{sol}{Solution}
\newtheorem{prop}{Proposition}

\renewcommand{\familydefault}{\sfdefault}

\newcommand{\lra}{\longrightarrow}
\newcommand{\ra}{\rightarrow}
\newcommand{\surj}{\twoheadrightarrow}
\newcommand{\graph}{\mathrm{graph}}
\newcommand{\bb}[1]{\mathbb{#1}}
\newcommand{\Z}{\bb{Z}}
\newcommand{\Q}{\bb{Q}}
\newcommand{\R}{\bb{R}}
\newcommand{\C}{\bb{C}}
\newcommand{\N}{\bb{N}}
\newcommand{\M}{\mathbf{M}}
\newcommand{\m}{\mathbf{m}}
\newcommand{\MM}{\mathscr{M}}
\newcommand{\HH}{\mathscr{H}}
\newcommand{\Om}{\Omega}
\newcommand{\Ho}{\in\HH(\Om)}
\newcommand{\bd}{\partial}
\newcommand{\del}{\partial}
\newcommand{\bardel}{\overline\partial}
\newcommand{\textdf}[1]{\textbf{\textsf{#1}}\index{#1}}
\newcommand{\img}{\mathrm{img}}
\newcommand{\ip}[2]{\left\langle{#1},{#2}\right\rangle}
\newcommand{\inter}[1]{\mathrm{int}{#1}}
\newcommand{\exter}[1]{\mathrm{ext}{#1}}
\newcommand{\cl}[1]{\mathrm{cl}{#1}}
\newcommand{\ds}{\displaystyle}
\newcommand{\vol}{\mathrm{vol}}
\newcommand{\cnt}{\mathrm{ct}}
\newcommand{\osc}{\mathrm{osc}}
\newcommand{\LL}{\mathbf{L}}
\newcommand{\UU}{\mathbf{U}}
\newcommand{\support}{\mathrm{support}}
\newcommand{\AND}{\;\wedge\;}
\newcommand{\OR}{\;\vee\;}
\newcommand{\Oset}{\varnothing}
\newcommand{\st}{\ni}
\newcommand{\wh}{\widehat}

%Pagination stuff.
\setlength{\topmargin}{-.3 in}
\setlength{\oddsidemargin}{0in}
\setlength{\evensidemargin}{0in}
\setlength{\textheight}{9.in}
\setlength{\textwidth}{6.5in}
\setlength{\itemsep}{0.45in}
\pagestyle{empty}



\begin{document}

\begin{center}
{\huge Solución de Problemas con Programación (TC1017)}\\[1.5ex]
{\large \textbf{Homework 03}\\[1.5ex] %You should put your name here
30.01.19} %You should write the date here.
\end{center}

\vspace{0.2 cm}

\subsection*{Loops and conditionals}

Program in MATLAB/Octave the sums and products we saw in class:

\begin{enumerate}[label=\alph*)]
    \itemsep2.5ex
    \item {\Large $\sum \limits_{i=1}^{m=10} 2i + 3 =$}
    \item {\Large $\prod \limits_{i=1}^{k=6} i =$}
\end{enumerate}

\bigskip

Upload a \textbf{ZIP} compressed file containing \textbf{two MATLAB files} with \textbf{.m} extension:
\texttt{cicloM.m} y \texttt{cicloK.m}.
Each file should have the program and its documentation \textbf{(at the beginning)} in the following format:

\bigskip

\begin{lstlisting}[style=Matlab-editor]
% ------------------------------------
% NAME: Arturo Gonzalez
% STUDENT ID: A01170065
% NAMEOFYOURSCRIPT.m
% Describe the module in some words
% What happens and what do we expect
% ------------------------------------
% Program starts here!
...
\end{lstlisting}

\bigskip

{\Large Recommendations:}
\begin{itemize}
    \item You can design a flowchart to help in the process, if you find it convenient.
    \item Ensure you're running MATLAB in the same directory where your files are stored, take a look at the path in the top of your editor and the path in the top of the file explorer in the left.
    \item Use lowercase names for your files, with no spaces, symbols or accents.
    \item Don't forget to \textbf{document your functions} and include your name and student ID in the specified format above.
\end{itemize}
\end{document}