\documentclass[]{book}

%These tell TeX which packages to use.
\usepackage{array,epsfig}
\usepackage{amsmath}
\usepackage{amsfonts}
\usepackage{amssymb}
\usepackage{amsxtra}
\usepackage{amsthm}
\usepackage{mathrsfs}
\usepackage{color}
\usepackage[spanish, mexico]{babel}
\usepackage[utf8]{inputenc}
\usepackage{enumitem}
\usepackage{helvet}
\usepackage[framed]{matlab-prettifier}
\newcommand{\matlab}[1]{\lstinline[style=Matlab-pyglike]!#1!}

%Here I define some theorem styles and shortcut commands for symbols I use often
\theoremstyle{definition}
\newtheorem{defn}{Definition}
\newtheorem{thm}{Theorem}
\newtheorem{cor}{Corollary}
\newtheorem*{rmk}{Remark}
\newtheorem{lem}{Lemma}
\newtheorem*{joke}{Joke}
\newtheorem{ex}{Example}
\newtheorem*{sol}{Solution}
\newtheorem{prop}{Proposition}

\renewcommand{\familydefault}{\sfdefault}

\newcommand{\lra}{\longrightarrow}
\newcommand{\ra}{\rightarrow}
\newcommand{\surj}{\twoheadrightarrow}
\newcommand{\graph}{\mathrm{graph}}
\newcommand{\bb}[1]{\mathbb{#1}}
\newcommand{\Z}{\bb{Z}}
\newcommand{\Q}{\bb{Q}}
\newcommand{\R}{\bb{R}}
\newcommand{\C}{\bb{C}}
\newcommand{\N}{\bb{N}}
\newcommand{\M}{\mathbf{M}}
\newcommand{\m}{\mathbf{m}}
\newcommand{\MM}{\mathscr{M}}
\newcommand{\HH}{\mathscr{H}}
\newcommand{\Om}{\Omega}
\newcommand{\Ho}{\in\HH(\Om)}
\newcommand{\bd}{\partial}
\newcommand{\del}{\partial}
\newcommand{\bardel}{\overline\partial}
\newcommand{\textdf}[1]{\textbf{\textsf{#1}}\index{#1}}
\newcommand{\img}{\mathrm{img}}
\newcommand{\ip}[2]{\left\langle{#1},{#2}\right\rangle}
\newcommand{\inter}[1]{\mathrm{int}{#1}}
\newcommand{\exter}[1]{\mathrm{ext}{#1}}
\newcommand{\cl}[1]{\mathrm{cl}{#1}}
\newcommand{\ds}{\displaystyle}
\newcommand{\vol}{\mathrm{vol}}
\newcommand{\cnt}{\mathrm{ct}}
\newcommand{\osc}{\mathrm{osc}}
\newcommand{\LL}{\mathbf{L}}
\newcommand{\UU}{\mathbf{U}}
\newcommand{\support}{\mathrm{support}}
\newcommand{\AND}{\;\wedge\;}
\newcommand{\OR}{\;\vee\;}
\newcommand{\Oset}{\varnothing}
\newcommand{\st}{\ni}
\newcommand{\wh}{\widehat}

%Pagination stuff.
\setlength{\topmargin}{-.3 in}
\setlength{\oddsidemargin}{0in}
\setlength{\evensidemargin}{0in}
\setlength{\textheight}{9.in}
\setlength{\textwidth}{6.5in}
\setlength{\itemsep}{0.45in}
\pagestyle{empty}



\begin{document}

\begin{center}
{\huge Solución de Problemas con Programación (TC1017)}\\[1.5ex]
{\large \textbf{Tarea EX03}\\[1.5ex] %You should put your name here
11.11.19} %You should write the date here.
\end{center}

\vspace{0.2 cm}

\subsection*{Quick Problem Set}

\bigskip

Implementa una solución para cada uno de los siguientes problemas.

\bigskip

\begin{enumerate}[label=\alph*)]
    \itemsep1.5ex
    \item Una \textbf{función} que reciba un número y regrese el triple (+1)
    \item Una \textbf{función} que reciba una matriz y la regrese multiplicada por 10 (+2)
    \item Un \textbf{script} que imprima "Hola Pueblo" 1500 veces (+2)
    \item Una \textbf{función} que reciba un número entero y regrese su factorial (+3)
    \item Un \textbf{script} que produzca una gráfica de 100 números generados aleatoriamente (\textit{Hint}: \matlab{plot(y)}) (+2)
\end{enumerate}

\vspace{1in}

Deberás entregar \textbf{cinco} archivos \textbf{.m} directamente en Blackboard (sin carpeta comprimida).

\vspace{0.5in}

\bigskip

{\Large Recomendaciones:}
\begin{itemize}
    \item Si lo crees necesario, haz un diagrama de flujo que te ayude a guiarte en el proceso.
    \item Asegúrate de estar corriendo el MATLAB en el mismo lugar donde guardaste tus archivos.
    \item Asegúrate de que tu archivo tiene nombre en minúsculas, sin espacios ni acentos o símbolos. 
    \item No te olvides de incluir tu nombre y número de matrícula en los archivos fuente.
\end{itemize}
\end{document}