\documentclass[]{book}

%These tell TeX which packages to use.
\usepackage{array,epsfig}
\usepackage{amsmath}
\usepackage{amsfonts}
\usepackage{amssymb}
\usepackage{amsxtra}
\usepackage{amsthm}
\usepackage{mathrsfs}
\usepackage{color}
\usepackage[spanish, mexico]{babel}
\usepackage[utf8]{inputenc}
\usepackage{enumitem}
\usepackage{helvet}
\usepackage[framed]{matlab-prettifier}
\newcommand{\matlab}[1]{\lstinline[style=Matlab-pyglike]!#1!}

%Here I define some theorem styles and shortcut commands for symbols I use often
\theoremstyle{definition}
\newtheorem{defn}{Definition}
\newtheorem{thm}{Theorem}
\newtheorem{cor}{Corollary}
\newtheorem*{rmk}{Remark}
\newtheorem{lem}{Lemma}
\newtheorem*{joke}{Joke}
\newtheorem{ex}{Example}
\newtheorem*{sol}{Solution}
\newtheorem{prop}{Proposition}

\renewcommand{\familydefault}{\sfdefault}

\newcommand{\lra}{\longrightarrow}
\newcommand{\ra}{\rightarrow}
\newcommand{\surj}{\twoheadrightarrow}
\newcommand{\graph}{\mathrm{graph}}
\newcommand{\bb}[1]{\mathbb{#1}}
\newcommand{\Z}{\bb{Z}}
\newcommand{\Q}{\bb{Q}}
\newcommand{\R}{\bb{R}}
\newcommand{\C}{\bb{C}}
\newcommand{\N}{\bb{N}}
\newcommand{\M}{\mathbf{M}}
\newcommand{\m}{\mathbf{m}}
\newcommand{\MM}{\mathscr{M}}
\newcommand{\HH}{\mathscr{H}}
\newcommand{\Om}{\Omega}
\newcommand{\Ho}{\in\HH(\Om)}
\newcommand{\bd}{\partial}
\newcommand{\del}{\partial}
\newcommand{\bardel}{\overline\partial}
\newcommand{\textdf}[1]{\textbf{\textsf{#1}}\index{#1}}
\newcommand{\img}{\mathrm{img}}
\newcommand{\ip}[2]{\left\langle{#1},{#2}\right\rangle}
\newcommand{\inter}[1]{\mathrm{int}{#1}}
\newcommand{\exter}[1]{\mathrm{ext}{#1}}
\newcommand{\cl}[1]{\mathrm{cl}{#1}}
\newcommand{\ds}{\displaystyle}
\newcommand{\vol}{\mathrm{vol}}
\newcommand{\cnt}{\mathrm{ct}}
\newcommand{\osc}{\mathrm{osc}}
\newcommand{\LL}{\mathbf{L}}
\newcommand{\UU}{\mathbf{U}}
\newcommand{\support}{\mathrm{support}}
\newcommand{\AND}{\;\wedge\;}
\newcommand{\OR}{\;\vee\;}
\newcommand{\Oset}{\varnothing}
\newcommand{\st}{\ni}
\newcommand{\wh}{\widehat}

%Pagination stuff.
\setlength{\topmargin}{-.3 in}
\setlength{\oddsidemargin}{0in}
\setlength{\evensidemargin}{0in}
\setlength{\textheight}{9.in}
\setlength{\textwidth}{6.5in}
\setlength{\itemsep}{0.45in}
\pagestyle{empty}



\begin{document}

\begin{center}
{\huge Solución de Problemas con Programación (TC1017)}\\[1.5ex]
{\large \textbf{Homework 02}\\[1.5ex] %You should put your name here
30.01.19} %You should write the date here.
\end{center}

\vspace{0.2 cm}

\subsection*{Functions}

Program in MATLAB/Octave the following functions:

\begin{enumerate}[label=\alph*)]
    \itemsep2.5ex
    \item The function \matlab{duplica}
    \begin{itemize}
        \item \textbf{Receives} a single parameter, $x$.
        \item \textbf{Returns} twice the value of $x$.
        \item Example: \matlab{duplica(20)} $\implies$ \matlab{40}.
    \end{itemize}
    \item The function \matlab{sumar}
    \begin{itemize}
        \item \textbf{Receives} three parameters.
        \item \textbf{Returns} the sum of the three parameters.
        \item Example: \matlab{sumar(2,3,5)} $\implies$ \matlab{10}.
    \end{itemize}
    \item The function \matlab{holapueblo}
    \begin{itemize}
        \item \textbf{Receives} a single parameter, $x$.
        \item \textbf{Prints}: ``Hola'' \textbf{if} $x$ is divisible by 3
        \item \textbf{Prints}: ``Pueblo'' \textbf{if} $x$ is divisible by 5
        \item \textbf{Prints}: ``Hola Pueblo'' \textbf{if} $x$ is divisible by 3 and by 5
        \item \textbf{Prints}: ``No'' \textbf{if} $x$ does not fall in any other previous case.
        \item Example: \matlab{holapueblo(30)} $\implies$ \texttt{"Hola Pueblo"}
    \end{itemize}
\end{enumerate}

\bigskip

You should upload a \textbf{ZIP} compressed file containing \textbf{three MATLAB files} with \textbf{.m} extension:
\texttt{duplica.m}, \texttt{sumar.m} and \texttt{holapueblo.m}.
Each file should have the corresponding function and its documentation in this format:

\bigskip

\begin{lstlisting}[style=Matlab-editor]
function output = functionname(x)
% NAME: Arturo Gonzalez
% STUDENT ID: A01170065
% FUNCTIONNAME output = functionname (x)
% FUNCTIONNAME returns some magic
% You can use it like this
% functionname(parameters) = returnvalue
...
\end{lstlisting}

\pagebreak

{\Large Recommendations:}
\begin{itemize}
    \item Ensure to completely understand how \matlab{function}, \matlab{if}, \matlab{else}, \matlab{end} and \matlab{mod} commands work in MATLAB.
    \item You may need the \matlab{elseif} command. Use \matlab{help <functionname>} to learn how to use any command you need.
    \item To \textbf{print} a value, you can do that either with \matlab{sprintf} or \matlab{disp}. To delimit a whole phrase to print, use double quotes: \matlab{disp("Hola Pueblo Viejo y Sabio")}.
    \item Remember that if you need to save a value to use it later, you can assign it to a variable, like this: \matlab{out = x + 7}.
    \item Ensure you're running MATLAB in the same directory where your files are stored, take a look at the path in the top of your editor and the path in the top of the file explorer in the left.
    \item Make sure your files are named exactly as your functions.
    \item Use lowercase names for your files, with no spaces, symbols or accents.
    \item Don't forget to \textbf{document your functions} and include your name and Student ID in the specified format above.
\end{itemize}
\end{document}