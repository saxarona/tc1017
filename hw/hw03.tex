\documentclass[]{book}

%These tell TeX which packages to use.
\usepackage{array,epsfig}
\usepackage{amsmath}
\usepackage{amsfonts}
\usepackage{amssymb}
\usepackage{amsxtra}
\usepackage{amsthm}
\usepackage{mathrsfs}
\usepackage{color}
\usepackage[spanish, mexico]{babel}
\usepackage[utf8]{inputenc}
\usepackage{enumitem}
\usepackage{helvet}
\usepackage[framed]{matlab-prettifier}
\newcommand{\matlab}[1]{\lstinline[style=Matlab-pyglike]!#1!}

%Here I define some theorem styles and shortcut commands for symbols I use often
\theoremstyle{definition}
\newtheorem{defn}{Definition}
\newtheorem{thm}{Theorem}
\newtheorem{cor}{Corollary}
\newtheorem*{rmk}{Remark}
\newtheorem{lem}{Lemma}
\newtheorem*{joke}{Joke}
\newtheorem{ex}{Example}
\newtheorem*{sol}{Solution}
\newtheorem{prop}{Proposition}

\renewcommand{\familydefault}{\sfdefault}

\newcommand{\lra}{\longrightarrow}
\newcommand{\ra}{\rightarrow}
\newcommand{\surj}{\twoheadrightarrow}
\newcommand{\graph}{\mathrm{graph}}
\newcommand{\bb}[1]{\mathbb{#1}}
\newcommand{\Z}{\bb{Z}}
\newcommand{\Q}{\bb{Q}}
\newcommand{\R}{\bb{R}}
\newcommand{\C}{\bb{C}}
\newcommand{\N}{\bb{N}}
\newcommand{\M}{\mathbf{M}}
\newcommand{\m}{\mathbf{m}}
\newcommand{\MM}{\mathscr{M}}
\newcommand{\HH}{\mathscr{H}}
\newcommand{\Om}{\Omega}
\newcommand{\Ho}{\in\HH(\Om)}
\newcommand{\bd}{\partial}
\newcommand{\del}{\partial}
\newcommand{\bardel}{\overline\partial}
\newcommand{\textdf}[1]{\textbf{\textsf{#1}}\index{#1}}
\newcommand{\img}{\mathrm{img}}
\newcommand{\ip}[2]{\left\langle{#1},{#2}\right\rangle}
\newcommand{\inter}[1]{\mathrm{int}{#1}}
\newcommand{\exter}[1]{\mathrm{ext}{#1}}
\newcommand{\cl}[1]{\mathrm{cl}{#1}}
\newcommand{\ds}{\displaystyle}
\newcommand{\vol}{\mathrm{vol}}
\newcommand{\cnt}{\mathrm{ct}}
\newcommand{\osc}{\mathrm{osc}}
\newcommand{\LL}{\mathbf{L}}
\newcommand{\UU}{\mathbf{U}}
\newcommand{\support}{\mathrm{support}}
\newcommand{\AND}{\;\wedge\;}
\newcommand{\OR}{\;\vee\;}
\newcommand{\Oset}{\varnothing}
\newcommand{\st}{\ni}
\newcommand{\wh}{\widehat}

%Pagination stuff.
\setlength{\topmargin}{-.3 in}
\setlength{\oddsidemargin}{0in}
\setlength{\evensidemargin}{0in}
\setlength{\textheight}{9.in}
\setlength{\textwidth}{6.5in}
\setlength{\itemsep}{0.45in}
\pagestyle{empty}



\begin{document}

\begin{center}
{\huge Solución de Problemas con Programación (TC1017)}\\[1.5ex]
{\large \textbf{Tarea 03}\\[1.5ex] %You should put your name here
30.01.19} %You should write the date here.
\end{center}

\vspace{0.2 cm}

\subsection*{Ciclos y condiciones}

Programa en MATLAB/Octave las sumatorias y productos que vimos en clase:

\begin{enumerate}[label=\alph*)]
    \itemsep2.5ex
    \item {\Large $\sum \limits_{i=1}^{m=10} 2i + 3 =$}
    \item {\Large $\prod \limits_{i=1}^{k=6} i =$}
\end{enumerate}

\bigskip

Deberás entregar un archivo comprimido en \textbf{ZIP} que contenga \textbf{dos archivos} de MATLAB con extensión \textbf{.m}:
\texttt{cicloM.m} y \texttt{cicloK.m}.
Cada archivo debe tener la función y su documentación \textdf{(al inicio)} en el siguiente formato:

\bigskip

\begin{lstlisting}[style=Matlab-editor]
% ------------------------------------
% NAME: Arturo Gonzalez
% STUDENT ID: A01170065
% NAMEOFYOURSCRIPT.m
% Describe the module in some words
% What happens and what do we expect
% ------------------------------------
% Program starts here!
...
\end{lstlisting}

\bigskip

{\Large Recomendaciones:}
\begin{itemize}
    \item Si lo crees necesario, haz un diagrama de flujo que te ayude a guiarte en el proceso.
    \item Asegúrate de estar corriendo el MATLAB en el mismo lugar donde guardaste tus archivos.
    \item Asegúrate de que tus archivos tienen nombres en minúsculas, sin espacios ni acentos o símbolos. 
    \item No te olvides de \textbf{documentar tus scripts} e incluir tu nombre y número de matrícula en el formato especificado arriba.
\end{itemize}
\end{document}