\documentclass[]{book}

%These tell TeX which packages to use.
\usepackage{array,epsfig}
\usepackage{amsmath}
\usepackage{amsfonts}
\usepackage{amssymb}
\usepackage{amsxtra}
\usepackage{amsthm}
\usepackage{mathrsfs}
\usepackage{color}
\usepackage[spanish, mexico]{babel}
\usepackage[utf8]{inputenc}
\usepackage{enumitem}
\usepackage{helvet}
\usepackage[framed]{matlab-prettifier}
\newcommand{\matlab}[1]{\lstinline[style=Matlab-pyglike]!#1!}

%Here I define some theorem styles and shortcut commands for symbols I use often
\theoremstyle{definition}
\newtheorem{defn}{Definition}
\newtheorem{thm}{Theorem}
\newtheorem{cor}{Corollary}
\newtheorem*{rmk}{Remark}
\newtheorem{lem}{Lemma}
\newtheorem*{joke}{Joke}
\newtheorem{ex}{Example}
\newtheorem*{sol}{Solution}
\newtheorem{prop}{Proposition}

\renewcommand{\familydefault}{\sfdefault}

\newcommand{\lra}{\longrightarrow}
\newcommand{\ra}{\rightarrow}
\newcommand{\surj}{\twoheadrightarrow}
\newcommand{\graph}{\mathrm{graph}}
\newcommand{\bb}[1]{\mathbb{#1}}
\newcommand{\Z}{\bb{Z}}
\newcommand{\Q}{\bb{Q}}
\newcommand{\R}{\bb{R}}
\newcommand{\C}{\bb{C}}
\newcommand{\N}{\bb{N}}
\newcommand{\M}{\mathbf{M}}
\newcommand{\m}{\mathbf{m}}
\newcommand{\MM}{\mathscr{M}}
\newcommand{\HH}{\mathscr{H}}
\newcommand{\Om}{\Omega}
\newcommand{\Ho}{\in\HH(\Om)}
\newcommand{\bd}{\partial}
\newcommand{\del}{\partial}
\newcommand{\bardel}{\overline\partial}
\newcommand{\textdf}[1]{\textbf{\textsf{#1}}\index{#1}}
\newcommand{\img}{\mathrm{img}}
\newcommand{\ip}[2]{\left\langle{#1},{#2}\right\rangle}
\newcommand{\inter}[1]{\mathrm{int}{#1}}
\newcommand{\exter}[1]{\mathrm{ext}{#1}}
\newcommand{\cl}[1]{\mathrm{cl}{#1}}
\newcommand{\ds}{\displaystyle}
\newcommand{\vol}{\mathrm{vol}}
\newcommand{\cnt}{\mathrm{ct}}
\newcommand{\osc}{\mathrm{osc}}
\newcommand{\LL}{\mathbf{L}}
\newcommand{\UU}{\mathbf{U}}
\newcommand{\support}{\mathrm{support}}
\newcommand{\AND}{\;\wedge\;}
\newcommand{\OR}{\;\vee\;}
\newcommand{\Oset}{\varnothing}
\newcommand{\st}{\ni}
\newcommand{\wh}{\widehat}

%Pagination stuff.
\setlength{\topmargin}{-.3 in}
\setlength{\oddsidemargin}{0in}
\setlength{\evensidemargin}{0in}
\setlength{\textheight}{9.in}
\setlength{\textwidth}{6.5in}
\setlength{\itemsep}{0.45in}
\pagestyle{empty}



\begin{document}

\begin{center}
{\huge Solución de Problemas con Programación (TC1017)}\\[1.5ex]
{\large \textbf{Tarea 00b}\\[1.5ex] %You should put your name here
22.04.19} %You should write the date here.
\end{center}

\vspace{0.2 cm}

\subsection*{Estadísticas básicas}

\bigskip

Genera un \textbf{script} de MATLAB para, junto con tu archivo de datos, puedas encontrar la respuesta a las siguientes preguntas (80 \%).
Posteriormente, contéstalas como \textbf{comentarios}. Considera que para contestar \textbf{cada pregunta} es necesario \textbf{hacer un cálculo}.
Incluye las respuestas como comentarios antes o después de la \textbf{sección correspondiente en tu script} (20 \%)

No olvides que la mayor parte de las instrucciones necesarias están en tu \textbf{Actividad EX02}.

\bigskip

\begin{itemize}
    \item \textsc{Música}:
    \begin{enumerate}[label=\alph*)]
        \item ¿Cuántas canciones registraste? (10\%)
        \item ¿Cuántas canciones oíste el \textbf{último día} de tu registro? (10\%)
        \item De todas las registradas, ¿qué porcentaje de canciones te saltaste? (20\%)
        \item ¿Cuál fue el artista que más escuchaste? (20\%)
        \item ¿Cuál fue la canción que más escuchaste? (20\%)
    \end{enumerate}
    \item \textsc{Netflix}: 
    \begin{enumerate}[label=\alph*)]
        \item ¿Cuántas películas o episodios registraste? Cuenta ambos: tres películas y dos episodios son 5 en total (10\%)
        \item ¿Cuántas películas viste el \textbf{último día} de tu registro? (15\%)
        \item ¿Cuántos episodios viste el \textbf{primer día} de tu registro? (15\%)
        \item De todos tus datos, ¿qué porcentaje te saltaste el intro o los créditos (cualquiera de los dos)? (20\%)
        \item ¿Cuál es el título de la serie que más viste? (20\%)
    \end{enumerate}
    \item \textsc{Juegos}:
    \begin{enumerate}[label=\alph*)]
        \item ¿Cuántos juegos o partidas registraste? (10\%)
        \item ¿Cuántas partidas tuviste el \textbf{último día} de tu registro? (15\%)
        \item ¿Cuántas partidas tuviste el \textbf{primer día} de tu registro? (15\%)
        \item ¿Cuál es el nombre del juego que más jugaste? (20\%)
        \item De todas tus partidas, ¿qué porcentaje ganaste o empataste? (20\%)
    \end{enumerate}
\end{itemize}

Deberás entregar una carpeta comprimida en \textbf{ZIP} con tu script (\texttt{tumatricula.m}) y tu archivo de datos (tumatricula.csv).

\section*{Recomendaciones}
\begin{itemize}
    \item Si lo crees necesario, haz un diagrama de flujo que te ayude a guiarte en el proceso.
    \item Asegúrate de estar corriendo el MATLAB en el mismo lugar donde guardaste tus archivos. Esto es sumamente importante.
    \item Asegúrate de que tu archivo tiene nombre en minúsculas, sin espacios ni acentos o símbolos. 
    \item No te olvides de incluir tu nombre y número de matrícula en los archivos fuente.
\end{itemize}
\end{document}