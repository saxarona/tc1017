\documentclass[spanish, 10pt]{article}

\usepackage[table, xcdraw]{xcolor}
\usepackage[utf8]{inputenc}
\usepackage[spanish, mexico]{babel}
\usepackage{helvet}
\usepackage{fullpage}
\usepackage{graphicx}
\usepackage{enumitem}
\usepackage{tikz}
\usepackage{ulem}
\usepackage{url}
\usepackage{hyperref}
\usepackage[margin = 3 cm]{geometry}
\usepackage{amsmath}
\usepackage{amsfonts}

\usepackage{matlab-prettifier}
\usepackage{multicol}

\usetikzlibrary{arrows, shapes, trees, calc, decorations.pathreplacing, shapes.misc, positioning, automata}

\setlength\parindent{0pt}

\renewcommand{\familydefault}{\sfdefault}
\newcommand{\responserule}{{\large\rule{14 cm}{0.3mm}}}
\newcommand{\shortresponserule}{{\large\rule{5 cm}{0.3mm}}}
\newcommand{\veryshortresponserule}{{\large\rule{3 cm}{0.3mm}}}
\newcommand{\matlab}[1]{\lstinline[style=Matlab-pyglike]!#1!}


% Specifications for listing package
% \lstset{	
%     basicstyle = \scriptsize\ttfamily,
%     keywordstyle = \color{blue}\ttfamily,
%     stringstyle = \color{red}\ttfamily,
%    	commentstyle = \color{gray}\ttfamily,
%    	tabsize = 3,
%    	breaklines = true,
%    	stepnumber = 1,
%    	showtabs = false,
%    	showstringspaces = false,
%    	frame = none
% }

% Commands for true/false questions
% ----------------------------------------------------------------
\newcommand{\question}[1]{%
	\noindent
	\begin{minipage}[t]{0.15\linewidth}
	\centering		
		\textbf{[\hspace{1 cm}]}
	\end{minipage}%
	\begin{minipage}[t]{0.85\linewidth}
		#1
	\end{minipage}
	\smallskip
}
% ----------------------------------------------------------------

\setlength\parindent{0pt}

\begin{document}

\begin{center}
	{\Large \textbf{Solución de Problemas con Programación (TC-1017)}}
	
	\bigskip
	{\large \textbf{Actividad 03 -- Control de Flujo II}}
\end{center}

\bigskip
{\large \textbf{Nombre}: \rule{13.7 cm}{0.4mm}}

% \bigskip
% {\large \textbf{Matrícula}: \rule{5 cm}{0.4mm}}

% \bigskip
% {\large \textbf{Name}: \rule{14 cm}{0.4mm}}

\bigskip
{\large \textbf{Matrícula}: \rule{5 cm}{0.4mm}} \hfill {\large \textbf{Fecha}: \today}

\bigskip

{\footnotesize Nota: es probable que esta actividad nos asuste un poco al principio. Es perfectamente normal.
En efecto, es de mayor dificultad a las que hemos visto anteriormente y probablemente haya dudas.
Si hay algo que no entiendas, no te quedes sin preguntar.}

\section{Condiciones y Ciclos I}

Resuelve las operaciones y contesta correctamente. Puedes usar MATLAB/Octave para ayudarte.

\begin{enumerate}
    \itemsep2.5ex
    \item Describe el conjunto $A$ por extensión si $A = \{ n : n \in \mathbb{N}, n \leq 10 \}$: \\[3ex] \responserule
    \item Sea $x = \langle 2, 5, 45, 17, 10, 22, 121 \rangle$, y 
    $$f(x_i) =
        \begin{cases}
            2x_i, & \text{si } x_i \text{ es igual a } 2 \\
            3x_i, & \text{si } x_i < 11 \text{ y } x_i \mod 5 = 0 \\
            x_i^2, & \text{si } 11 < x_i < 20 \\
            x_i, & \text{si } 20 \leq x_i \leq 100 \text{ o bien si } x_i \geq 200 \\
            0, & \text{de lo contrario}
        \end{cases}$$
        \bigskip
    \begin{enumerate}
        \item $f(x_1) =$ \quad \; \shortresponserule
        \item $f(x_2) =$ \quad \; \shortresponserule
        \item $f(x_3) =$ \quad \; \shortresponserule
        \item $f(x_4) =$ \quad \; \shortresponserule
        \item $f(x_5) =$ \quad \; \shortresponserule
        \item $f(x_6) =$ \quad \; \shortresponserule
        \item $f(x_7) =$ \quad \; \shortresponserule
        \item $f(x_i=1) =$ \shortresponserule
    \end{enumerate}
    \item {\Large $\sum \limits_{i=1}^{n=100} i =$} \hfill \shortresponserule
    \item {\Large $\sum \limits_{i=1}^{n=10} 2i + 3 =$} \hfill \shortresponserule
    \item {\Large $\prod \limits_{i=1}^{n=6} i =$} \hfill \shortresponserule
\end{enumerate}

\section{Condiciones y Ciclos II}

Antes de implementar las instrucciones anteriores en MATLAB/Octave, hay que hacernos algunas preguntas:

\begin{itemize}
	\item Para el inciso 2 de la sección anterior:
	\begin{enumerate}
        \item ¿Cuántos parámetros tiene $f(x_i)$? \hfill \shortresponserule
        \item ¿Cuántos posibles resultados distintos tiene? \hfill \shortresponserule
        \item ¿Qué pasa si evaluamos $f(x_i=11)$ \hfill \shortresponserule
        \item ¿Qué pasa si evaluamos $f(x_i=255)$ \hfill \shortresponserule
    \end{enumerate}
	\item Para el inciso 3 de la sección anterior:
	\begin{enumerate}
        \item ¿Cuántos números estamos sumando? \hfill \shortresponserule
        \item ¿Habrá alguna manera más rápida de hacer este procedimiento? \hfill \veryshortresponserule
    \end{enumerate}
	\item Para el inciso 4 de la sección anterior:
	\begin{enumerate}
        \item ¿Cuántas veces debo repetir el procedimiento? \hfill \shortresponserule
        \item ¿Habrá alguna manera más rápida de hacerlo? \hfill \shortresponserule
    \end{enumerate}
    \item Para el inciso 5 de la sección anterior:
	\begin{enumerate}
        \item ¿Cuántos valores distintos toma la $i$? \hfill \shortresponserule
        \item ¿Habrá alguna manera más rápida de hacerlo? \hfill \shortresponserule
        \item ¿Puedo hacer el procedimiento si en lugar de darme valores para $i$ conozco una condición para detenerme?
        \hfill \shortresponserule
    \end{enumerate}
\end{itemize}

\section{Comandos}

Escribe los símbolos y comandos de MATLAB/Octave que consideres útiles para recordar lo visto en la sesión, y una descripción breve de cada uno de ellos:

\vfill

\textbf{Apegándome al Código de Ética de los Estudiantes del Tecnológico de Monterrey, me comprometo a que mi actuación en esta actividad esté regida por la honestidad académica.}

\end{document}