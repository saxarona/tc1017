\documentclass[spanish, 10pt]{article}

\usepackage[table, xcdraw]{xcolor}
\usepackage[utf8]{inputenc}
\usepackage[spanish, mexico]{babel}
\usepackage{helvet}
\usepackage{fullpage}
\usepackage{graphicx}
\usepackage{enumitem}
\usepackage{tikz}
\usepackage{ulem}
\usepackage{url}
\usepackage{hyperref}
\usepackage[margin = 3 cm]{geometry}
\usepackage{amsmath}
\usepackage{amsfonts}

\usepackage{matlab-prettifier}
\usepackage{multicol}

\usetikzlibrary{arrows, shapes, trees, calc, decorations.pathreplacing, shapes.misc, positioning, automata}

\setlength\parindent{0pt}

\renewcommand{\familydefault}{\sfdefault}
\newcommand{\responserule}{{\large\rule{14 cm}{0.3mm}}}
\newcommand{\shortresponserule}{{\large\rule{5 cm}{0.3mm}}}
\newcommand{\veryshortresponserule}{{\large\rule{3 cm}{0.3mm}}}
\newcommand{\matlab}[1]{\lstinline[style=Matlab-pyglike]!#1!}


% Specifications for listing package
% \lstset{	
%     basicstyle = \scriptsize\ttfamily,
%     keywordstyle = \color{blue}\ttfamily,
%     stringstyle = \color{red}\ttfamily,
%    	commentstyle = \color{gray}\ttfamily,
%    	tabsize = 3,
%    	breaklines = true,
%    	stepnumber = 1,
%    	showtabs = false,
%    	showstringspaces = false,
%    	frame = none
% }

% Commands for true/false questions
% ----------------------------------------------------------------
\newcommand{\question}[1]{%
	\noindent
	\begin{minipage}[t]{0.15\linewidth}
	\centering		
		\textbf{[\hspace{1 cm}]}
	\end{minipage}%
	\begin{minipage}[t]{0.85\linewidth}
		#1
	\end{minipage}
	\smallskip
}
% ----------------------------------------------------------------

\setlength\parindent{0pt}

\begin{document}

\begin{center}
	{\Large \textbf{Solución de Problemas con Programación (TC-1017)}}
	
	\bigskip
	{\large \textbf{Actividad EX 02 -- Problemas Reales II (Tablas) (+2)}}
\end{center}

\bigskip
{\large \textbf{Nombre}: \rule{13.7 cm}{0.4mm}}

% \bigskip
% {\large \textbf{Matrícula}: \rule{5 cm}{0.4mm}}

% \bigskip
% {\large \textbf{Name}: \rule{14 cm}{0.4mm}}

\bigskip
{\large \textbf{Matrícula}: \rule{5 cm}{0.4mm}} \hfill {\large \textbf{Fecha}: \today}

\bigskip

{\footnotesize Nota: esta actividad es de mayor dificultad a las que hemos visto anteriormente y probablemente haya dudas.
Si hay algo que no entiendas, no te quedes sin preguntar.}

\section{Repaso}

Contesta estas breves preguntas de repaso.

\begin{itemize}
    \item ¿Qué es un vector? \\[3ex] \responserule
    \item ¿Qué es una matriz? \\[3ex] \responserule
    \item ¿Qué es un arreglo? \\[3ex] \responserule
    \item ¿Cómo represento en MATLAB un vector? \\[3ex] \responserule
    \item ¿Cómo represento en MATLAB una matriz? \\[3ex] \responserule
    \item ¿Cómo represento en MATLAB un arreglo? \\[3ex] \responserule
\end{itemize}

\section{Reflexión}

\begin{itemize}
	\item ¿Cómo accedo al tercer elemento de un vector $x$? \hfill \veryshortresponserule
	\item ¿Cómo obtengo todas las columnas de la $i$-ésima fila de una matriz $A$? \hfill \veryshortresponserule
	\item ¿Cómo obtengo el tercer elemento de la segunda fila de un arreglo $C$? \hfill \veryshortresponserule
\end{itemize}

\pagebreak

\section{Comandos útiles de la sesión}

Escribe una descripción breve de cada uno de los símbolos y comandos de MATLAB/Octave que consideres útiles para recordar lo visto en la sesión. La lista debería incluir, al menos, \matlab{readtable, find, strfind, count} y \matlab{array2table}.

\vfill

\textbf{Apegándome al Código de Ética de los Estudiantes del Tecnológico de Monterrey, me comprometo a que mi actuación en esta actividad esté regida por la honestidad académica.}

\end{document}