\documentclass[spanish, 10pt]{article}

\usepackage[table, xcdraw]{xcolor}
\usepackage[utf8]{inputenc}
\usepackage[spanish, mexico]{babel}
\usepackage{helvet}
\usepackage{fullpage}
\usepackage{graphicx}
\usepackage{enumitem}
\usepackage{tikz}
\usepackage{ulem}
\usepackage{url}
\usepackage{hyperref}
\usepackage[margin = 3 cm]{geometry}
\usepackage{amsmath}
\usepackage{amsfonts}

\usepackage{matlab-prettifier}
\usepackage{multicol}

\usetikzlibrary{arrows, shapes, trees, calc, decorations.pathreplacing, shapes.misc, positioning, automata}

\setlength\parindent{0pt}

\renewcommand{\familydefault}{\sfdefault}
\newcommand{\responserule}{{\large\rule{14 cm}{0.3mm}}}
\newcommand{\shortresponserule}{{\large\rule{5 cm}{0.3mm}}}
\newcommand{\veryshortresponserule}{{\large\rule{3 cm}{0.3mm}}}
\newcommand{\matlab}[1]{\lstinline[style=Matlab-pyglike]!#1!}


% Specifications for listing package
% \lstset{	
%     basicstyle = \scriptsize\ttfamily,
%     keywordstyle = \color{blue}\ttfamily,
%     stringstyle = \color{red}\ttfamily,
%    	commentstyle = \color{gray}\ttfamily,
%    	tabsize = 3,
%    	breaklines = true,
%    	stepnumber = 1,
%    	showtabs = false,
%    	showstringspaces = false,
%    	frame = none
% }

% Commands for true/false questions
% ----------------------------------------------------------------
\newcommand{\question}[1]{%
	\noindent
	\begin{minipage}[t]{0.15\linewidth}
	\centering		
		\textbf{[\hspace{1 cm}]}
	\end{minipage}%
	\begin{minipage}[t]{0.85\linewidth}
		#1
	\end{minipage}
	\smallskip
}
% ----------------------------------------------------------------

\setlength\parindent{0pt}

\begin{document}

\begin{center}
	{\Large \textbf{Solución de Problemas con Programación (TC-1017)}}
	
	\bigskip
	{\large \textbf{In-class activity 03 -- Control Flow II}}
\end{center}

\bigskip
{\large \textbf{Name}: \rule{13.7 cm}{0.4mm}}

% \bigskip
% {\large \textbf{Matrícula}: \rule{5 cm}{0.4mm}}

% \bigskip
% {\large \textbf{Name}: \rule{14 cm}{0.4mm}}

\bigskip
{\large \textbf{Student ID}: \rule{5 cm}{0.4mm}} \hfill {\large \textbf{Date}: \today}

\bigskip

{\footnotesize Note: this activity may scare you at first sight. It's perfectly normal.
In fact, it is harder than any other activity we've done so far, and you may have questions.
If there's something not clear enough, please don't hesitate to ask.}

\section{Conditionals and Loops I}

Solve the operations and answer correctly. You can use MATLAB/Octave.

\begin{enumerate}
    \itemsep2.5ex
    \item Describe the set $A$ by extension, if $A = \{ n : n \in \mathbb{N}, n \leq 10 \}$: \\[3ex] \responserule
    \item Let $x = \langle 2, 5, 45, 17, 10, 22, 121 \rangle$, and 
    $$f(x_i) =
        \begin{cases}
            2x_i, & \text{si } x_i \text{ es igual a } 2 \\
            3x_i, & \text{si } x_i < 11 \text{ y } x_i \mod 5 = 0 \\
            x_i^2, & \text{si } 11 < x_i < 20 \\
            x_i, & \text{si } 20 \leq x_i \leq 100 \text{ o bien si } x_i \geq 200 \\
            0, & \text{de lo contrario}
        \end{cases}$$
        \bigskip
    \begin{enumerate}
        \item $f(x_1) =$ \quad \; \shortresponserule
        \item $f(x_2) =$ \quad \; \shortresponserule
        \item $f(x_3) =$ \quad \; \shortresponserule
        \item $f(x_4) =$ \quad \; \shortresponserule
        \item $f(x_5) =$ \quad \; \shortresponserule
        \item $f(x_6) =$ \quad \; \shortresponserule
        \item $f(x_7) =$ \quad \; \shortresponserule
        \item $f(x_i=1) =$ \shortresponserule
    \end{enumerate}
    \item {\Large $\sum \limits_{i=1}^{n=100} i =$} \hfill \shortresponserule
    \item {\Large $\sum \limits_{i=1}^{n=10} 2i + 3 =$} \hfill \shortresponserule
    \item {\Large $\prod \limits_{i=1}^{n=6} i =$} \hfill \shortresponserule
\end{enumerate}

\section{Conditionals and Loops II}

Before implementing in MATLAB/Octave the instructions of the previous section, we need to formulate some questions:

\begin{itemize}
	\item For exercise 2 of previous section:
	\begin{enumerate}
        \item How many parameter does $f(x_i)$ have? \hfill \shortresponserule
        \item How many different return values does it have? \hfill \shortresponserule
        \item What happens if we evaluate $f(x_i=11)$? \hfill \shortresponserule
        \item What happens if we evaluate $f(x_i=255)$? \hfill \shortresponserule
    \end{enumerate}
	\item For exercise 3 of previous section:
	\begin{enumerate}
        \item How many numbers are we adding? \hfill \shortresponserule
        \item Is there a faster way to do this procedure? \hfill \veryshortresponserule
    \end{enumerate}
	\item For exercise 4 of previous section:
	\begin{enumerate}
        \item How many times is the procedure repeated? \hfill \shortresponserule
        \item Is there a faster way to do this procedure? \hfill \shortresponserule
    \end{enumerate}
    \item For exercise 5 of previous section:
	\begin{enumerate}
        \item How many different values does $i$ take? \hfill \shortresponserule
        \item Is there a faster way to do this procedure? \hfill \shortresponserule
        \item Can I do this procedure using a condition to stop instead of using a certain number of values for $i$?
        \hfill \shortresponserule
    \end{enumerate}
\end{itemize}

\section{Commands}

Write the symbols and MATLAB/Octave commands that you consider useful to remember what we saw in class, and a brief description of each:
\vfill

\textbf{In accordance with the Tecnológico de Monterrey Student Code of Honor, my performance in this activity will be guided by academic honesty}.

\end{document}