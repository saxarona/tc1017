\documentclass[spanish, 10pt]{article}

\usepackage[table, xcdraw]{xcolor}
\usepackage[utf8]{inputenc}
\usepackage[spanish, mexico]{babel}
\usepackage{helvet}
\usepackage{fullpage}
\usepackage{graphicx}
\usepackage{enumitem}
\usepackage{tikz}
\usepackage{ulem}
\usepackage{url}
\usepackage{hyperref}
\usepackage[margin = 3 cm]{geometry}
\usepackage{amsmath}
\usepackage{amsfonts}

\usepackage{matlab-prettifier}
\usepackage{multicol}

\usetikzlibrary{arrows, shapes, trees, calc, decorations.pathreplacing, shapes.misc, positioning, automata}

\setlength\parindent{0pt}

\renewcommand{\familydefault}{\sfdefault}
\newcommand{\responserule}{{\large\rule{14 cm}{0.3mm}}}
\newcommand{\shortresponserule}{{\large\rule{5 cm}{0.3mm}}}
\newcommand{\veryshortresponserule}{{\large\rule{3 cm}{0.3mm}}}
\newcommand{\matlab}[1]{\lstinline[style=Matlab-pyglike]!#1!}


% Specifications for listing package
% \lstset{	
%     basicstyle = \scriptsize\ttfamily,
%     keywordstyle = \color{blue}\ttfamily,
%     stringstyle = \color{red}\ttfamily,
%    	commentstyle = \color{gray}\ttfamily,
%    	tabsize = 3,
%    	breaklines = true,
%    	stepnumber = 1,
%    	showtabs = false,
%    	showstringspaces = false,
%    	frame = none
% }

% Commands for true/false questions
% ----------------------------------------------------------------
\newcommand{\question}[1]{%
	\noindent
	\begin{minipage}[t]{0.15\linewidth}
	\centering		
		\textbf{[\hspace{1 cm}]}
	\end{minipage}%
	\begin{minipage}[t]{0.85\linewidth}
		#1
	\end{minipage}
	\smallskip
}
% ----------------------------------------------------------------

\setlength\parindent{0pt}

\begin{document}

\begin{center}
	{\Large \textbf{Solución de Problemas con Programación (TC-1017)}}
	
	\bigskip
	{\large \textbf{Actividad EX 01 -- Problemas Reales I}}
\end{center}

\bigskip
{\large \textbf{Nombre}: \rule{13.7 cm}{0.4mm}}

% \bigskip
% {\large \textbf{Matrícula}: \rule{5 cm}{0.4mm}}

% \bigskip
% {\large \textbf{Name}: \rule{14 cm}{0.4mm}}

\bigskip
{\large \textbf{Matrícula}: \rule{5 cm}{0.4mm}} \hfill {\large \textbf{Fecha}: \today}

\bigskip

{\footnotesize Nota: es probable que esta actividad nos asuste un poco al principio. Es perfectamente normal.
En efecto, es de mayor dificultad a las que hemos visto anteriormente y probablemente haya dudas.
Si hay algo que no entiendas, no te quedes sin preguntar.}

\section{Definición del problema}

Considera el modelo $SIR$ siguiente y escribe las definiciones siguientes.

\begin{itemize}
    \item ¿Qué significa la $S$ en $SIR$? \\[1ex] \textit{Grupo de personas Susceptibles}
    \item ¿Qué significa la $I$ en $SIR$? \\[1ex] \textit{Grupo de personas Infectadas}
    \item ¿Qué significa la $R$ en $SIR$? \\[1ex] \textit{Grupo de personas Recuperadas}
    \item ¿Qué simboliza $\Delta t$? \\[1ex] \textit{Un cambio en el tiempo}
    \item ¿Qué significa \textit{discretizar}? \\[1ex] \textit{Convertir una magnitud continua e infinita, en una magnitud discreta y finita}
\end{itemize}


\section{Ecuaciones}

Escribe las ecuaciones para definir el cambio en los bloques $S, I, R$ a continuación:

\begin{align*}
    S^{n+1} & = S^n - \beta \Delta t S^n I^n \\
    I^{n+1} & = I^n + \beta \Delta t S^n I^n - \gamma \Delta t I^n \\
    R^{n+1} & = R^n + \gamma \Delta I^n
\end{align*}

\begin{itemize}
    \item ¿Qué significa la $\beta$? \\[1ex] \textit{Qué tan fácil te enfermas}
    \item ¿Qué significa la $\mu$? \\[1ex] \textit{Probabilidad de encontrarse con alguien enfermo}
    \item ¿Qué significa la $\gamma$? \\[1ex] \textit{Qué tan fácil te recuperas}
\end{itemize}

\pagebreak

Escribe las ecuaciones diferenciales del sistema anterior:

\begin{align*}
S' & =- \beta S I \\
I' & =\beta S I - \gamma I \\
R' & =\gamma I
\end{align*}

Escribe la versión discretizada del sistema de ecuaciones diferenciales anterior:

\begin{align*}
    S'(t_n) & =- \beta S(t_n) I(t_n) \\
    I'(t_n) & =\beta S(t_n) I(t_n) - \gamma I(t_n) \\
    R'(t_n) & =\gamma I(t_n)
\end{align*}

\section{Comandos útiles de la sesión}

Escribe los símbolos y comandos de MATLAB/Octave que consideres útiles para recordar lo visto en la sesión, y una descripción breve de cada uno de ellos:

\vfill

\textbf{Apegándome al Código de Ética de los Estudiantes del Tecnológico de Monterrey, me comprometo a que mi actuación en esta actividad esté regida por la honestidad académica.}

\end{document}