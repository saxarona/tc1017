\documentclass[spanish, 10pt]{article}

\usepackage[table, xcdraw]{xcolor}
\usepackage[utf8]{inputenc}
\usepackage[spanish, mexico]{babel}
\usepackage{helvet}
\usepackage{fullpage}
\usepackage{graphicx}
\usepackage{enumitem}
\usepackage{tikz}
\usepackage{ulem}
\usepackage{url}
\usepackage{hyperref}
\usepackage[margin = 3 cm]{geometry}
\usepackage{amsmath}
\usepackage{matlab-prettifier}
\usepackage{multicol}

\usetikzlibrary{arrows, shapes, trees, calc, decorations.pathreplacing, shapes.misc, positioning, automata}

\setlength\parindent{0pt}

\renewcommand{\familydefault}{\sfdefault}
\newcommand{\responserule}{{\large\rule{14 cm}{0.3mm}}}
\newcommand{\shortresponserule}{{\large\rule{5 cm}{0.3mm}}}
\newcommand{\matlab}[1]{\lstinline[style=Matlab-pyglike]!#1!}

% Specifications for listing package
% \lstset{	
%     basicstyle = \scriptsize\ttfamily,
%     keywordstyle = \color{blue}\ttfamily,
%     stringstyle = \color{red}\ttfamily,
%    	commentstyle = \color{gray}\ttfamily,
%    	tabsize = 3,
%    	breaklines = true,
%    	stepnumber = 1,
%    	showtabs = false,
%    	showstringspaces = false,
%    	frame = none
% }

% Commands for true/false questions
% ----------------------------------------------------------------
\newcommand{\question}[1]{%
	\noindent
	\begin{minipage}[t]{0.15\linewidth}
	\centering		
		\textbf{[\hspace{1 cm}]}
	\end{minipage}%
	\begin{minipage}[t]{0.85\linewidth}
		#1
	\end{minipage}
	\smallskip
}
% ----------------------------------------------------------------

\setlength\parindent{0pt}

\begin{document}

\begin{center}
	{\Large \textbf{Solución de Problemas con Programación (TC-1017)}}
	
	\bigskip
	{\large \textbf{Actividad 02 -- Funciones y Control de Flujo I}}
\end{center}

\bigskip
{\large \textbf{Nombre}: \rule{13.7 cm}{0.4mm}}

% \bigskip
% {\large \textbf{Matrícula}: \rule{5 cm}{0.4mm}}

% \bigskip
% {\large \textbf{Name}: \rule{14 cm}{0.4mm}}

\bigskip
{\large \textbf{Matrícula}: \rule{5 cm}{0.4mm}} \hfill {\large \textbf{Fecha}: \today}

\section{Funciones I}

Usando el MATLAB/Octave para ayudarte, evalúa las siguientes funciones sobre el conjunto $A = \{0,1,2,3,4\}$.
Recuerda utilizar los operadores que vimos la clase pasada (*,+,-,/).
\begin{multicols}{2}
    \begin{enumerate}
        \item $f(x) = 2x + 3$
        \begin{enumerate}
            \item $f(0) =$
            \item $f(1) =$
            \item $f(2) =$
            \item $f(3) =$
            \item $f(4) =$
        \end{enumerate}
        \item $g(x,y) = 5x^2 + 3y + 5$
        \begin{enumerate}
            \item $g(0,4) =$
            \item $g(1,3) =$
            \item $g(2,2) =$
            \item $g(3,1) =$
            \item $g(4,0) =$
        \end{enumerate}
        
        \columnbreak

        \item $h(x) =
            \begin{cases}
                2x, & \text{si } x \text{ es par} \\
                3x, & \text{si } x \text{ es impar}
            \end{cases}$
        \begin{enumerate}
            \item $h(0) =$
            \item $h(1) =$
            \item $h(2) =$
            \item $h(3) =$
            \item $h(4) =$
        \end{enumerate}
    \end{enumerate}
\end{multicols}

\section{Funciones II y Control de Flujo I}

Antes de implementar las funciones anteriores en MATLAB/Octave, hay que hacernos algunas preguntas:

\begin{enumerate}
	\item $f(x) = 2x + 3$
	\begin{enumerate}
        \item ¿Cuántos parámetros tiene $f(x)$? \hfill \shortresponserule
        \item ¿Cuál o cuáles son esos parámetros? \hfill \shortresponserule
    \end{enumerate}
	\item $g(x,y) = 5x^2 + 3y + 5$
	\begin{enumerate}
        \item ¿Cuántos parámetros tiene $g(x,y)$? \hfill \shortresponserule
        \item ¿Cuál o cuáles son esos parámetros? \hfill \shortresponserule
    \end{enumerate}
	\item $h(x) =
    \begin{cases}
        2x, & \text{si } x \text{ es par} \\
        3x, & \text{si } x \text{ es impar}
    \end{cases}$
    \begin{enumerate}
        \item ¿Cuántos parámetros tiene $h(x)$? \hfill \shortresponserule
        \item ¿Cuál o cuáles son esos parámetros? \hfill \shortresponserule
        \item ¿Cómo hago para que regrese a veces $2x$ y a veces $3x$? \hfill \shortresponserule
        \item ¿Cómo hago para revisar si es par o impar? \hfill \shortresponserule
    \end{enumerate}
\end{enumerate}

\section{Comandos}

Escribe una descripción breve de cada comando. Si no te sabes alguno, prueba a usarlo en la ventana de comandos de MATLAB/Octave junto con \matlab{help}.

\vspace{3ex}

\begin{enumerate}[label=\alph*)]
	\large
	\item \matlab{function} \\[3ex] \responserule
	\item \matlab{end} \\[3ex] \responserule
	\item \matlab{if} \\[3ex] \responserule
	\item \matlab{else} \\[3ex] \responserule
	\item \matlab{mod} \\[3ex] \responserule
	\item \matlab{eq} \\[3ex] \responserule
	\item \matlab{==} \\[3ex] \responserule
\end{enumerate}

\vfill

\textbf{Apegándome al Código de Ética de los Estudiantes del Tecnológico de Monterrey, me comprometo a que mi actuación en esta actividad esté regida por la honestidad académica.}

\end{document}