\documentclass[spanish, 10pt]{article}

\usepackage[table, xcdraw]{xcolor}
\usepackage[utf8]{inputenc}
\usepackage[spanish, mexico]{babel}
\usepackage{helvet}
\usepackage{fullpage}
\usepackage{graphicx}
\usepackage{enumitem}
\usepackage{tikz}
\usepackage{ulem}
\usepackage{url}
\usepackage{hyperref}
\usepackage[margin = 3 cm]{geometry}
\usepackage{amsmath}
\usepackage{matlab-prettifier}

\usetikzlibrary{arrows, shapes, trees, calc, decorations.pathreplacing, shapes.misc, positioning, automata}

\setlength\parindent{0pt}

\renewcommand{\familydefault}{\sfdefault}
\newcommand{\responserule}{{\large\rule{14 cm}{0.3mm}}}
\newcommand{\matlab}[1]{\lstinline[style=Matlab-pyglike]!#1!}

% Specifications for listing package
% \lstset{	
%     basicstyle = \scriptsize\ttfamily,
%     keywordstyle = \color{blue}\ttfamily,
%     stringstyle = \color{red}\ttfamily,
%    	commentstyle = \color{gray}\ttfamily,
%    	tabsize = 3,
%    	breaklines = true,
%    	stepnumber = 1,
%    	showtabs = false,
%    	showstringspaces = false,
%    	frame = none
% }

% Commands for true/false questions
% ----------------------------------------------------------------
\newcommand{\question}[1]{%
	\noindent
	\begin{minipage}[t]{0.15\linewidth}
	\centering		
		\textbf{[\hspace{1 cm}]}
	\end{minipage}%
	\begin{minipage}[t]{0.85\linewidth}
		#1
	\end{minipage}
	\smallskip
}
% ----------------------------------------------------------------

\setlength\parindent{0pt}

\begin{document}

\begin{center}
	{\Large \textbf{Solución de Problemas con Programación (TC-1017)}}
	
	\bigskip
	{\large \textbf{Actividad 01 -- Primeros pasos}}
\end{center}

\bigskip
{\large \textbf{Nombre}: \rule{14 cm}{0.4mm}}

% \bigskip
% {\large \textbf{Matrícula}: \rule{5 cm}{0.4mm}}

% \bigskip
% {\large \textbf{Name}: \rule{14 cm}{0.4mm}}

\bigskip
{\large \textbf{Matrícula}: \rule{5 cm}{0.4mm}} \hfill {\large \textbf{Fecha}: \today}

\section{Aritmética I}

Sin usar tu calculadora, realiza las operaciones siguientes:

\begin{enumerate}[label=\alph*)]
	\item $\dfrac{8}{3} \div \dfrac{3}{4} =$
	\item $11 - 2 =$
	\item $2 - 6 + 1 \times 5 =$
	\item $\dfrac{1}{4} \div \dfrac{8}{9} =$
	\item $5 \times 7 + 10 \times 2 =$
	\item $\dfrac{7}{4} + \dfrac{4}{3} =$
	\item $\dfrac{3}{2} \times \dfrac{1}{2} =$
	\item $8^ 2 + (2+8) \times 5 - 9 =$
	\item $2 \div 6^2 + (7-8) \div 2 =$
	\item $(9 + 4) \div 8^2 + 3 \times 8 =$
\end{enumerate}

\section{Aritmética II}

Con ayuda de MATLAB/Octave, realiza las siguientes operaciones:

\begin{enumerate}[label=\alph*)]
	\item $\dfrac{782}{2123} \div \dfrac{919}{1333} =$
	\item $32935872 - 9920175 =$
	\item $\left( \dfrac{2}{7} \right)^{7} =$
	\item $12345 - 7675 \times 345 - 248 =$
	\item $2 \pi =$
	\item $3 \mathrm{e} =$
	\item $\sin 67 =$
	\item $\sqrt{2} =$
\end{enumerate}

\section{Comandos}

Escribe una descripción breve de cada comando. Si no te sabes alguno, prueba a usarlo en la ventana de comandos de MATLAB/Octave.

\vspace{3ex}

\begin{enumerate}[label=\alph*)]
	\large
	\item \matlab{help} \\[3ex] \responserule
	\item \matlab{sqrt} \\[3ex] \responserule
	\item \matlab{pi} \\[3ex] \responserule
	\item \matlab{e} \\[3ex] \responserule
	\item \matlab{format} \\[3ex] \responserule
	\item \matlab{atan} \\[3ex] \responserule
	\item \matlab{sprintf} \\[3ex] \responserule
	\item \matlab{disp} \\[3ex] \responserule
\end{enumerate}

\vfill

\textbf{Apegándome al Código de Ética de los Estudiantes del Tecnológico de Monterrey, me comprometo a que mi actuación en esta actividad esté regida por la honestidad académica.}

\end{document}