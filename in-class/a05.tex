\documentclass[spanish, 10pt]{article}

\usepackage[table, xcdraw]{xcolor}
\usepackage[utf8]{inputenc}
\usepackage[spanish, mexico]{babel}
\usepackage{helvet}
\usepackage{fullpage}
\usepackage{graphicx}
\usepackage{enumitem}
\usepackage{tikz}
\usepackage{ulem}
\usepackage{url}
\usepackage{hyperref}
\usepackage[margin = 3 cm]{geometry}
\usepackage{amsmath}
\usepackage{amsfonts}

\usepackage{matlab-prettifier}
\usepackage{multicol}

\usetikzlibrary{arrows, shapes, trees, calc, decorations.pathreplacing, shapes.misc, positioning, automata}

\setlength\parindent{0pt}

\renewcommand{\familydefault}{\sfdefault}
\newcommand{\responserule}{{\large\rule{14 cm}{0.3mm}}}
\newcommand{\shortresponserule}{{\large\rule{5 cm}{0.3mm}}}
\newcommand{\veryshortresponserule}{{\large\rule{3 cm}{0.3mm}}}
\newcommand{\matlab}[1]{\lstinline[style=Matlab-pyglike]!#1!}


% Specifications for listing package
% \lstset{	
%     basicstyle = \scriptsize\ttfamily,
%     keywordstyle = \color{blue}\ttfamily,
%     stringstyle = \color{red}\ttfamily,
%    	commentstyle = \color{gray}\ttfamily,
%    	tabsize = 3,
%    	breaklines = true,
%    	stepnumber = 1,
%    	showtabs = false,
%    	showstringspaces = false,
%    	frame = none
% }

% Commands for true/false questions
% ----------------------------------------------------------------
\newcommand{\question}[1]{%
	\noindent
	\begin{minipage}[t]{0.15\linewidth}
	\centering		
		\textbf{[\hspace{1 cm}]}
	\end{minipage}%
	\begin{minipage}[t]{0.85\linewidth}
		#1
	\end{minipage}
	\smallskip
}
% ----------------------------------------------------------------

\setlength\parindent{0pt}

\begin{document}

\begin{center}
	{\Large \textbf{Solución de Problemas con Programación (TC-1017)}}
	
	\bigskip
	{\large \textbf{Actividad 05 -- Estructuras de Datos II}}
\end{center}

\bigskip
{\large \textbf{Nombre}: \rule{13.7 cm}{0.4mm}}

% \bigskip
% {\large \textbf{Matrícula}: \rule{5 cm}{0.4mm}}

% \bigskip
% {\large \textbf{Name}: \rule{14 cm}{0.4mm}}

\bigskip
{\large \textbf{Matrícula}: \rule{5 cm}{0.4mm}} \hfill {\large \textbf{Fecha}: \today}

\bigskip

% {\footnotesize Nota: es probable que esta actividad nos asuste un poco al principio. Es perfectamente normal.
% En efecto, es de mayor dificultad a las que hemos visto anteriormente y probablemente haya dudas.
% Si hay algo que no entiendas, no te quedes sin preguntar.}

\section{Vectores y Matrices}

Resuelve las operaciones y contesta correctamente. Puedes usar MATLAB/Octave para ayudarte.

\begin{enumerate}
    \itemsep2.5ex
    \item Sean $\mathbf{x} =\begin{bmatrix}
        1 & 3 & 5 & 7
    \end{bmatrix}$ \quad y \quad
    $\mathbf{y} = \begin{bmatrix}
        2 \\
        4 \\
        6 \\
        8
    \end{bmatrix}$

        \bigskip

    \begin{enumerate}
        \item $\mathbf{x} + 2 =$
        \item $\mathbf{y} + 3 =$
        \item $\mathbf{x} \cdot 4 =$
        \item $\mathbf{y}  \cdot 5 =$
        \item $\mathbf{x} + \mathbf{y}^T =$
        \item $\mathbf{x} + \mathbf{y} =$
        \item $\mathbf{x} \cdot \mathbf{y} =$
        \item $\mathbf{x} \otimes \mathbf{y} =$
        \item $\mathbf{y} \otimes \mathbf{x} =$
    \end{enumerate}

    \item Sean $ A = \begin{bmatrix}
        1 & 2\\
        3 & 4
    \end{bmatrix}$ \quad y
    $B = \begin{bmatrix}
        3 & 3 & 1\\
        2 & 3 & 5
    \end{bmatrix}$
    
    \bigskip

    \begin{enumerate}
        \item $A + 2 =$
        \item $A \times 5 =$
        \item $A \times B =$
        \item $B^T = $
        \item $B^T \times A = $
    \end{enumerate}
\end{enumerate}

\pagebreak

\section{Comandos: operadores y matrices constantes}

Prueba los siguientes comandos en MATLAB/Octave y escribe una breve descripción de los mismos:

\begin{enumerate}[label=\alph*)]
	\large
	\item \matlab{*} \\[3ex] \responserule
	\item \matlab{.*} \\[3ex] \responserule
	\item \matlab{zeros(m,n)} \\[3ex] \responserule
	\item \matlab{ones(m,n)} \\[3ex] \responserule
	\item \matlab{eye(n)} \\[3ex] \responserule
	\item \matlab{rand(m,n)} \\[3ex] \responserule
	\item \matlab{randi(k,m,n)} \\[3ex] \responserule
	\item \matlab{triu(A)} \\[3ex] \responserule
	\item \matlab{tril(A)} \\[3ex] \responserule
\end{enumerate}

\vfill

\textbf{Apegándome al Código de Ética de los Estudiantes del Tecnológico de Monterrey, me comprometo a que mi actuación en esta actividad esté regida por la honestidad académica.}

\end{document}